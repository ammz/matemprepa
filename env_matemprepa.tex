\startenvironment env_matemprepa
\project matemprepa

\usepath[{1_alg_elemental,2_geometria,3_alg_intermedia,4_alg_avanzada,5_precalculo}]

\definefontfamily [mainface] [rm] [Cambria]
\definefontfamily [mainface] [ss] [Calibri]
\definefontfamily [mainface] [tt] [Consolas]
\definefontfamily [mainface] [mm] [Cambria Math]

\definefontfamily [office] [serif] [Times New Roman]
\definefontfamily [office] [sans]  [Arial]
\definefontfamily [office] [mono]  [Courier]
\definefontfamily [office] [math]  [TeX Gyre Termes Math]

\definefontfamily [dejavu] [serif] [DejaVu Serif]
\definefontfamily [dejavu] [sans]  [DejaVu Sans]
\definefontfamily [dejavu] [mono]  [DejaVu Sans Mono]
\definefontfamily [dejavu] [math]  [XITS Math] [scale=1.1]


\setupbodyfont[10pt,office]

% \setupbodyfont[11pt,xits]
% \setupbodyfont[11pt, stixtwo]

\definesymbol [vartriangle] [\textormathchar{"25B5}]

\mainlanguage[es]
% \setuplabeltext[es][%
  % table=Tabla ,
  % figure=Gráfico ,
  % ]

\usemodule[tikz]
\usemodule[pgf]
\usetikzlibrary[arrows]

\setupcolors[state=start]
\definecolor [fondo]     [r=0.95,g=0.95,b=0.95]
\definecolor [fondoejem] [r=0.98,g=0.97,b=0.93]
\definecolor [azulon]    [r=0.21,g=0.67,b=0.85]
\definecolor [ejemcolor] [darkred]
\definecolor [color_def] [azulon]
\definecolor [color_teo] [r=0.81,g=0.58,b=0.15]% anarajando

\definehighlight [obj][
  style=bolditalic]
\definehighlight [ini][color=ejemcolor]

\setuppapersize[A5][A5]
\setuplayout[%
  backspace=14mm,
  % cutspace=15mm,
  width=120mm,
  % leftmargindistance=3cm,
  leftmargin=0em,
  % leftedge=0em,
  % rightmargin=30mm,
  % rightedgedistance=2cm,
  topspace=10mm,
  header=10mm,
  footer=0mm,
  height=180mm,
]

% \setuppapersize[A4][A4]
% \setuplayout[%
%   backspace=23mm,
%   width=120mm,
%   % leftmargindistance=3cm,
%   leftmargin=0em,
%   % leftedge=0em,
%   rightmargin=50mm,
%   % rightedgedistance=2cm,
%   topspace=15mm,
%   header=20mm,
%   footer=0mm,
%   height=260mm,
% ]

\setupwhitespace[medium]

\setupreferenceformat[in][style=\it,color=red]
\definereferenceformat[ineq][%
  left=(,
  right=),
]

\setuppagenumbering[alternative=doublesided]%, location={header,right}]
\setupheader[text][after={\vskip 1pt \hrule}, style=\ss\tfx]
% \setupheader[text][after=]
% \setupheader[margin][after=]


\define[2]\MyChapterCommand%
       {\framed[%
           frame=off,
           bottomframe=off,
           topframe=off,
           top=\vss,
           bottom=\vss,
           height=3cm,
           % width= \dimexpr(\textwidth-3cm)\relax,
           width=\textwidth,
           align={stretch,nothyphenated,flushright},
           strut=no]
         {\headtext{chapter}#2}
         \quad
         \framed[%
           top=\vss,
           bottom=\vss,
           height=3cm,
           width=3cm]
                {\switchtobodyfont[32pt]#1}
       }

\setuphead[chapter][%
   % command=\MyChapterCommand,
   style=\ss\bfd,
   header=empty,
   after={\blank[10*big]}]


% \setuphead[chapter][alternative=margin]
% \definebodyfont[10pt,11pt,12pt][rm][tfe=Serif at 48pt]

% \setuplabeltext [es] [chapter=Tema~]

% \define[2]\Tema
%   {\framed[frame=off]{\framed[frame=on,height=2.5cm,width=2.5cm]{#1}\\#2\par}}

% \setuphead [chapter] [command=\Tema, numberstyle=\tfe]

\definefont [Fseccion] [SansBold sa 1.2]

\setuphead[section][%
  color=red,
  style=\Fseccion,
  % textstyle=\bf,
]


\setupitemgroup[itemize][%
  option={joinedup,intro},
]

\defineitemgroup[itemizer]
\setupitemgroup[itemizer][%
  placestopper=no,
  option={r,intro,joinedup,unpacked},
  style=\sl,
  right=),
  leftmargin=.7em,
  distance=.5em,
  after=,
]

\defineitemgroup[itemizep][itemizer][%
  option={r,intro,joinedup,intext,unpacked},
  leftmargin=0pt,
  distance=0pt,
]

\defineitemgroup[itemejem]
\setupitemgroup[itemejem:1][%
  option={intro},
  symbol=n,
  color=ejemcolor,
  placestopper=no,
  left=(,
  right=),
  distance=.7em,
  after={\blank[small]},
]

\startuseMPgraphic{mp:axiomframe}
  begingroup;
    for i=1 upto nofmultipars :
      % Draw the surrounding box
      path p;
      p := ( llcorner multipars[i]
             -- lrcorner multipars[i]
             -- urcorner multipars[i]
             -- ulcorner multipars[i]
             -- cycle )
             enlarged (0pt,0pt) ;
      fill p withcolor boxfillcolor ;
      linecap := butt;
      draw (p cutbefore point 3 of p cutafter point 4 of p)
            withpen pencircle scaled .5pt
            withcolor boxlinecolor ;
    endfor ;
  endgroup;
\stopuseMPgraphic

% \startuseMPgraphic{mp:marco}
%   begingroup;
%     for i=1 upto nofmultipars :
%       % Draw the surrounding box
%       path p;
%       p := ( llcorner multipars[i]
%              -- lrcorner multipars[i]
%              -- urcorner multipars[i]
%              -- ulcorner multipars[i]
%              -- cycle )
%              enlarged (0pt,0pt) ;
%       % fill p withcolor boxfillcolor ;
%       linecap := butt;
%       draw (p)
%             withpen pencircle scaled .5pt
%             withcolor boxlinecolor ;
%     endfor ;
%   endgroup;
% \stopuseMPgraphic

\definetextbackground[trecuadro][%
  % mp=mp:marco,
  background=none,
  before=, after=,
  framecolor=red,
  location=always,
  topoffset=3pt,
  leftoffset=13pt,
  rightoffset=10pt,
  bottomoffset=3pt,
  before={\startpostponingnotes\blank[big]\vbox\bgroup},
  after={\egroup\blank[big]\stoppostponingnotes},
]

\definetextbackground[tfondo][trecuadro][%
  mp=mp:axiomframe,
  backgroundcolor=fondo,
  before=, after=,
  framecolor=lightgray,
  before={\startpostponingnotes\blank[2*big]},
  after={\blank[2*big]\stoppostponingnotes},
]

\definetextbackground [tbteor][trecuadro][%
  framecolor=color_teo]

\definetextbackground [tbdemo][tfondo][%
  framecolor=color_teo,
  backgroundcolor=white]

\definetextbackground [tbdefi][trecuadro][%
  framecolor=color_def]

\definetextbackground [tbejem][tfondo][%
  backgroundcolor=white]

\definetextbackground [tbobser][tfondo][%
  backgroundcolor=white]


\defineframed[frecuadro][%
  toffset=3pt,
  loffset=13pt,
  roffset=10pt,
  boffset=3pt,
  before={\startpostponingnotes\blank[2*big]},
  after={\blank[2*big]\stoppostponingnotes},
]


\defineframed [fdefi] [frecuadro] [framecolor=color_def]


\setupenumerations [%
  alternative=serried,
  before=,after=,
  distance=.7em,
  width=broad,
  headstyle=normal,
  titlestyle=slanted,
  way=bytext,
  conversion=numbers,
]

\definecounter[definicion][way=bytext]
\defineenumeration[definicion][%
  text=DEFINICIÓN ,
  counter=definicion,
  align=flushleft,
  headstyle={\ss\tx\bf},
  headcolor=color_def,
  distance=1em,
  title=yes,
  titlecolor=black,
  titlestyle={\rm\bia}, % head y title están relacionados
  titledistance=.7em,
  % style=italic,
  before={\starttbdefi},
  after={\stoptbdefi},
]

\definecounter[axioma][way=bytext]
\defineenumeration[axioma][definicion][%
  text=AXIOMA ,
  counter=axioma,
]

\defineenumeration[axiomaextra][axioma][%
  number=no,
]

\definecounter[teorema][way=bytext]
\defineenumeration [teorema][definicion][%
  text=TEOREMA ,%Space after Theorem is deliberate
  right=:,
  counter=teorema,
  headcolor=color_teo,
  before={\starttbteor},
  after={\stoptbteor\blank[none]},
]

\defineenumeration[lema][teorema][%
  text=LEMA ,
  number=no,
]

\defineenumeration[corolario][lema][%
  text=COROLARIO ,
]

\defineenumeration [observacion][definicion] [%
  text=OBSERVACIÓN: ,
  headcolor=darkgray,
  style={\switchtobodyfont[9pt]},
  number=no,
  before={\starttbobser},
  after={\stoptbobser},
]

\defineenumeration [acuerdo][observacion] [%
  text=ACUERDO: ,
]

\defineenumeration [demo] [%
  text=PRUEBA: ,
  before={\blank[none]\starttbdemo},
  after={\blank[none]\stoptbdemo},
  inbetween={\blank[medium]},
  number=no,
  headstyle={\ss\tx\bf},
  headcolor=color_teo,
  distance=1em,
  title=no, %this is the default
  style=normal,
  closesymbol={\color[color_teo]{\mathematics{\square}}},
]

\defineenumeration [demop][demo][%
  alternative=top,
  % distance=2cm,
  closesymbol=,
]

\defineenumeration [demoejem][demo][%
  before=, after=,
  % before={\blank[none]\starttbejem},
  % after={\stoptbejem\blank[none]},
]

\defineenumeration [metodo][demo][%
 before=, after=,
 text=,
]

\setupnarrower[middle=1cm]
\defineparagraphs[ej][%
  rule=on,
  before={\blank\startnarrower},
  after={\stopnarrower\blank},
]

\defineenumeration [ejemplo] [%
  style={\switchtobodyfont[small]},
  text=Ejemplo: ,
  number=no,
  headstyle=\ss\tx,
  color=darkgray,
  before={\blank[medium]%
    \resetcounter[formula]\startnarrower},%\starttbejem},
  after={\resetcounter[formula]\stopnarrower}%\stoptbejem},
]

\defineenumeration [ejemplos][ejemplo] [%
  text=Ejemplos: ,
]

\definedescription [discusion] [%
  headstyle=\ss,
  headcolor=darkred,
  alternative=top,
  % command={\hskip-1cm},
  before={\blank[2*big]},
  after={\blank[2*big]},
]

\definedescription [descripcion] [%
  alternative=top,
  headcolor=darkred,
  hang=50,
  headstyle=\ss,
  % width=.4\textwidth,
  before={\blank[2*big]},
  after={\blank[2*big]},
]

\definedescription [comentario] [%
  style={\switchtobodyfont[small]},
  before={\starttbejem},
  after={\stoptbejem},
]


\setupformulas[prefix=no,% split=page
]

\defineformula[ejer] [%
  location=left,
  prefix=no,
  align=flushleft,
  distance=1em,
  numbercolor=ejemcolor,
  spaceafter=small,
]

\defineparagraphs [TwoColumns] [%
  n=2,
  rule=on,
  distance=.5cm,
]


% \define\PlaceFootnote
%   {\inmargin{\vtop{\placelocalnotes[footnote][before=,after=]}}}

% \setupnote [footnote][%
%   location=text,
%   bodyfont=small,
%   align={nothyphenated,right},
%   next=\PlaceFootnote,
% ]

% \setupnotation[footnote][%
%   alternative=serried,
% ]

\setupexternalfigures[directory=img]

% Una explicación para cada paso de la ecuación
\definemathalignment
  [doaligncomments]
  [m=3,distance=3em plus 1 fil]

\def\startaligncomments#1\stopaligncomments{%
   \def\Comment##1\NR{\NC\NC\text{##1}\dodoubleempty\doComment}
   %\qquad is for some space between the comment and the number
   \def\doComment[##1][##2]{%
    \iffirstargument \qquad \fi
    \NR[##1][##2]\NC\NC}
   \startdoaligncomments
   \NC\NC#1\stopdoaligncomments}

\definemathframed[graymath] [%
  frame=off,
  location=mathematics,
  background=color,
  backgroundcolor=lightgray,
   backgroundoffset=2pt,
]

\definemathfence[abs][bar][command=yes]


% \showframe
% \showlayout
\stopenvironment

%%% Local Variables:
%%% mode: context
%%% TeX-master: t
%%% End:
