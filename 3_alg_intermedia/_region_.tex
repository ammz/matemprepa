\message{ !name(c_sistemas_ecuaciones_lineales.tex)} %
\message{ !name(c_sistemas_ecuaciones_lineales.tex) !offset(-2) }
\startcomponent c_sistemas_ecuaciones_lineales
\project project_matemprepa
% \product prod_algebra_intermedia

\youtube{\from[AI56A]}
\startchapter[title={Sistemas de ecuaciones lineales}]

  \startsection[title={Conceptos fundamentales}]

    \startdefinicion
      Un arreglo de ecuaciones en la forma
      \startformula
        \startmathalignment[n=11,align={middle,middle,middle,middle,middle,middle,middle,middle,middle,middle,middle}]
          \NC a_{1\,1}\, x_1 \NC + \NC a_{1\,2}\, x_2 \NC + \NC a_{1\,3}\, x_3 \NC + \NC \cdots \NC + \NC a_{1\,n}\, x_n \NC = \NC b_1 \NR
          \NC a_{2\,1}\, x_1 \NC + \NC a_{2\,2}\, x_2 \NC + \NC a_{2\,3}\, x_3 \NC + \NC \cdots \NC + \NC a_{2\,n}\, x_n \NC = \NC b_2 \NR
          \NC a_{3\,1}\, x_1 \NC + \NC a_{3\,2}\, x_2 \NC + \NC a_{3\,3}\, x_3 \NC + \NC \cdots \NC + \NC a_{3\,n}\, x_n \NC = \NC b_3 \NR
          \NC \vdots \NC  \NC \vdots \NC  \NC \vdots \NC  \NC \ddots \NC  \NC \vdots \NC  \NC \vdots \NR
          \NC a_{m\,1}\, x_1 \NC + \NC a_{m\,2}\, x_2 \NC + \NC a_{m\,3}\, x_3 \NC + \NC \cdots \NC + \NC a_{m\,n}\, x_n \NC = \NC b_m, \NR
        \stopmathalignment
      \stopformula
      donde $x_1,\, x_2,\, x_3,\, \dots\,,\, x_n \in \reals$, se llama \obj{un sistema de ecuaciones lineales con esas $n$ incógnitas, escrito en (la) forma estándar}. En éste, $a_{1\,1},\, a_{1\,2},\, a_{1\,3},\, \dots\,,\, a_{m\,n} \in \reals$ fijos, llamados \obj{los coeficientes (numéricos) del sistema} y $b_1,\, b_2,\, b_3,\, \dots\,,\, b_m \in \reals$ fijos, llamados \obj{las (o los términos) constantes del sistema}.
    \stopdefinicion

    \startejemplos
      \startplaceformula
        \startejerformula
          \startmathalignment[n=5,align={right,right,right,right,middle}]
            \NC 3x \NC -2y \NC +z \NC = \NC -1 \NR[+]
            \NC x \NC +4y \NC -5z \NC = \NC 0 \NR
            \NC   \NC 2y \NC +6z \NC = \NC 3 \NR
          \stopmathalignment
        \stopejerformula
        Es un ssistema de tres ecuaciones lineales con tres incógnitas, escrito en forma estándar. Aquí, $a_{1\,1} = 3$, $a_{1\,2} = -2$, $a_{1\,3} = 1$, $a_{2\,1} = 1$, $a_{2\,2} = 4$, $a_{2\,3} = -5$, $a_{3\,1} = 0$, $a_{3\,2} = 2$, $a_{3\,3} = 6$; $b_1 = -1$, $b_2 = 0$, $b_3 = 3$; $x_1 = x$, $x_2 = y$, $x_3 = z$.
      \stopplaceformula

      \startplaceformula
        \startejerformula
          \startmathalignment[n=4,align={right,right,right,middle}]
            \NC 3x \NC -5 \NC = \NC 30y \NR[+]
            \NC 2y \NC -5x \NC = \NC 1 \NR
          \stopmathalignment
        \stopejerformula
        Es un sistema de ecuaciones lineales con dos incógnitas que no está escrito en la forma estándar. Lo reescribimos como tal
        \startformula
          \startmathalignment[n=4,align={right,right,right,middle}]
            \NC -30y \NC +3x \NC = \NC 5 \NR
            \NC 2y \NC -5x \NC = \NC 1 \NR
          \stopmathalignment
        \stopformula
        Identificamos cada elemento: $a_{1\,1} = -30$, $a_{1\,2} = 3$, $a_{2\,2} = -5$; $b_1 = 1$, $b_2 = 1$; $x_1 = y$, $x_2 = x$.
      \stopplaceformula
    \stopejemplos

    ¿Qué será la solución de una sistema de ecuaciones?

    Valores de las incógnitas del sistema que hacen ciertas a todas las ecuaciones de éste simultáneamente.

    Para hallar esos valores nos tienen que dar un conjunto de sustitución de cada incógnita del sistema, para de entre los elementos de ellos encontrar los valores que hacen ciertas a las ecuaciones del sistema.

    \startejemplos
      \startplaceformula
        \startejerformula[color=ejemcolor]
          \startmathalignment[n=4,align={right,right,right,middle}]
            \NC 4x \NC +3y \NC = \NC 7 \NR[+]
            \NC 4x \NC +y \NC = \NC 5 \NR
          \stopmathalignment
        \stopejerformula
        $S_1 = \{2, -1, 1, 4, 0\}$ es el conjunto de sustituciones de $x$ \\
        $S_2 = \{1, 0, -11, -3\}$ es el conjunto de sustituciones de $y$

        Recordamos el producto cartesiano de dos conjuntos:

        $S_1 \times\, S_2 = \{(2,1), (2,0), (2,-11), (2,-3), (-1,1), (-1,0), \dots \,, (0, -3)\}$

        Seleccionamos un subconjunto $S = \{(2,0), (4,-11), (1,1)\}$ que será el conjunto de sustituciones de nuestro sistema y comprobamos

        $x = 2,\; y = 0$

        \startformulas
          \startformula
            \startmathalignment[n=4, align={middle,middle,middle,left}]
              \NC 4 \cdot 2 \NC + 3 \cdot 0 \NC = \NC 7 \NR
              \NC 4 \cdot 2 \NC + 0         \NC = \NC 5 \NR
            \stopmathalignment
          \stopformula
          \startformula
            \Rightarrow
          \stopformula
          \startformula
            \startmathalignment[n=4, align={middle,middle,middle,left}]
              \NC 8 \NC + 0 \NC = \NC 7 \NR
              \NC 8 \NC + 0 \NC = \NC 5 \NR
            \stopmathalignment
          \stopformula
          \startformula
            \Rightarrow
          \stopformula
          \startformula
            \startmathalignment[n=3, align={middle,middle,left}]
              \NC 8  \NC \neq \NC 7 \NR
              \NC 8  \NC \neq \NC 5 \NR
            \stopmathalignment
          \stopformula
        \stopformulas

        $x = 4,\; y = -11$

        \startformulas
          \startformula
            \startmathalignment[n=4, align={middle,middle,middle,left}]
              \NC 4 \cdot 4 \NC + 3 (-11) \NC = \NC 7 \NR
              \NC 4 \cdot 4 \NC + (-11)         \NC = \NC 5 \NR
            \stopmathalignment
          \stopformula
          \startformula
            \Rightarrow
          \stopformula
          \startformula
            \startmathalignment[n=4, align={middle,middle,middle,left}]
              \NC 16 \NC -33 \NC = \NC 7 \NR
              \NC 1 \NC - 11 \NC = \NC 5 \NR
            \stopmathalignment
          \stopformula
          \startformula
            \Rightarrow
          \stopformula
          \startformula
            \startmathalignment[n=3, align={right,middle,left}]
              \NC -17  \NC \neq \NC 7 \NR
              \NC 5  \NC = \NC 5 \NR
            \stopmathalignment
          \stopformula
        \stopformulas

        $x = 1,\; y = 1$

        \startformulas
          \startformula
            \startmathalignment[n=4, align={middle,middle,middle,left}]
              \NC 4 \cdot 1 \NC + 3 \cdot 1 \NC = \NC 7 \NR
              \NC 4 \cdot 1 \NC + 1         \NC = \NC 5 \NR
            \stopmathalignment
          \stopformula
          \startformula
            \Rightarrow
          \stopformula
          \startformula
            \startmathalignment[n=4, align={middle,middle,middle,left}]
              \NC 4 \NC +3 \NC = \NC 7 \NR
              \NC 4 \NC +1 \NC = \NC 5 \NR
            \stopmathalignment
          \stopformula
          \startformula
            \Rightarrow
          \stopformula
          \startformula
            \startmathalignment[n=3, align={right,middle,left}]
              \NC 7  \NC = \NC 7 \NR
              \NC 5  \NC = \NC 5 \NR
            \stopmathalignment
          \stopformula
        \stopformulas
      \stopplaceformula
      La solución del sistema es \inframed{$x = 1, \, y = 1$} o \inframed{$(1,1)$} o como conjunto solución \inframed{$X = \left\{(1,1)\right\}$}.
    \stopejemplos

    \startdefinicion
      Sean $S_1, S_2, \dots\,, S_n,\; n$ conjuntos no vacíos y considere a $s_1 \in S_1, s_2 \in S_2, \dots\,, s_n \in S_n$. \obj{Un n-tuplo ordenado} es el arreglo $(s_1, s_2, \dots\,, s_n)$, que vea que será uno de los elementos del producto cartesiano $S_1 \times\, S_2 \times\, \dots\,  \times\, S_n$.
    \stopdefinicion

    \startejemplo
      \ini{Determine la solución de este sistema
      \startformula
        \startmathalignment[n=5]
          \NC 3x \NC -y  \NC +2z \NC = \NC 9 \NR
          \NC 2x \NC +y  \NC  -z \NC = \NC 7 \NR
          \NC  x \NC +2y \NC -3z \NC = \NC 4 \NR
        \stopmathalignment
      \stopformula
      si suponemos que el conjunto de sustitución es $S = \{(2, -1, 0), (3, 2, 1), (-1, -2, 1)\}$. Identifique un posible conjunto de sustitución para cada una de las tres incógnitas del sistema.}

      Los posibles conjuntos des sustituciones para cada una de las ecuaciones son: $x : S_1 = \{2,3-1\};\; y: S_2 = \{(-1,2,-2)\};\; z: S_3 = \{(0,1)\}$.

      \youtube{\from[AI56B]}
      Ahora vamos sustituyendo hasta encontrar un trío que satisfaga las ecuaciones del sistema

      $x =2, \, y = -1, \, z = 0$

      \startformulas
        \startformula
          \startmathalignment[n=5, align={middle,middle,middle,middle,left}]
            \NC 3 \cdot 2 \NC - (-1)  \NC 2 \cdot 0  \NC = \NC 9 \NR
            \NC 2 \cdot 2 \NC + (-1)  \NC - 0        \NC = \NC 7 \NR
            \NC 2         \NC + 2(-1) \NC -3 \cdot 0 \NC = \NC 4 \NR
          \stopmathalignment
        \stopformula
        \startformula
          \Rightarrow
        \stopformula
        \startformula
          \startmathalignment[n=5, align={middle,iddle,middle,middle,left}]
            \NC 6 \NC + 1 \NC +0 \NC = \NC 9 \NR
            \NC 4 \NC - 1 \NC -0 \NC = \NC 7 \NR
            \NC 2 \NC - 2 \NC -0 \NC = \NC 4 \NR
          \stopmathalignment
        \stopformula
        \startformula
          \Rightarrow
        \stopformula
        \startformula
          \startmathalignment[n=3, align={middle,middle,left}]
            \NC 7  \NC \neq \NC 9 \NR
          \stopmathalignment
        \stopformula
      \stopformulas

      $x =3, \, y = 2, \, z = 1$

      \startformulas
        \startformula
          \startmathalignment[n=5, align={middle,middle,middle,middle,left}]
            \NC 3 \cdot 3 \NC - 2  \NC + 2 \cdot 1  \NC = \NC 9 \NR
            \NC 2 \cdot 3 \NC + 2  \NC - 1           \NC = \NC 7 \NR
            \NC 3         \NC + 2\cdot 2 \NC -3 \cdot 1 \NC = \NC 4 \NR
          \stopmathalignment
        \stopformula
        \startformula
          \Rightarrow
        \stopformula
        \startformula
          \startmathalignment[n=5, align={middle,iddle,middle,middle,left}]
            \NC 9 \NC - 2 \NC +2 \NC = \NC 9 \NR
            \NC 6 \NC + 2 \NC -1 \NC = \NC 7 \NR
            \NC 3 \NC + 4 \NC -3 \NC = \NC 4 \NR
          \stopmathalignment
        \stopformula
        \startformula
          \Rightarrow
        \stopformula
        \startformula
          \startmathalignment[n=3, align={middle,middle,left}]
            \NC 9  \NC = \NC 9 \NR
            \NC 7  \NC = \NC 7 \NR
            \NC 4  \NC = \NC 4 \NR
          \stopmathalignment
        \stopformula
      \stopformulas
      $\therefore$  \inframed{$x =3,\, y = 2,\, z = 1$} o \inframed{$(3, 2, 1)$}

      Pero aún debemos comprobar el último tuplo.

      $x = -1, \, y = -2, \, z = 1$

      \startformulas
        \startformula
          \startmathalignment[n=5, align={middle,middle,middle,middle,left}]
            \NC 3 (-1) \NC - (-2)  \NC + 2 \cdot 1  \NC = \NC 9 \NR
            \NC 2 (-1) \NC + 2 (-2)  \NC - 1        \NC = \NC 7 \NR
            \NC -1     \NC + 2 (-2) \NC -3 \cdot 1 \NC = \NC 4 \NR
          \stopmathalignment
        \stopformula
        \startformula
          \Rightarrow
        \stopformula
        \startformula
          \startmathalignment[n=5, align={middle,iddle,middle,middle,left}]
            \NC -3 \NC + 2 \NC +2 \NC = \NC 9 \NR
            \NC -2 \NC - 2 \NC -1 \NC = \NC 7 \NR
            \NC -1 \NC - 4 \NC -3 \NC = \NC 4 \NR
          \stopmathalignment
        \stopformula
        \startformula
          \Rightarrow
        \stopformula
        \startformula
          \startmathalignment[n=3, align={middle,middle,left}]
            \NC 1  \NC \neq \NC 9 \NR
          \stopmathalignment
        \stopformula
      \stopformulas

       Por tanto, \inframed{$X = \{(1,2,3)\}$} será el conjunto solución.
    \stopejemplo
    Este método, como se ha visto, es muy engorroso, por lo que debemos buscar otros métodos alternativos.
  \stopsection

  \youtube{\from[AI57A]}
  \startsection[title={Solución por sustitución}]
    \startejemplos
      Resuelva los siguientes sisitemas por sustitución.
      \startplaceformula
        \startejerformula[color=ejemcolor]
          \startmathalignment[n=4,align={right,right,right,middle}]
            \NC 4x \NC +3y \NC = \NC 7 \NR[+]
            \NC 4x \NC +y \NC = \NC 5 \NR
          \stopmathalignment
        \stopejerformula
        Lo primero es buscar la ecuación  más fácil de resolver para una incógnita y despejar.
        \startformula
          y = 5 - 4x \leftarrow \text{ecuación auxiliar}
        \stopformula
        Ahora sustituimos en la otra ecuación
        \startformula
          4x + 3(5 -4x) = 7
        \stopformula
        Y resolvemos
        \startformula
          4x + 15 - 12x = 7
        \stopformula
        \startformula
          -8x = -8
        \stopformula
        \startformula
          x = \dfrac{-8}{8} = -1
        \stopformula
        Ahora obtenemos la otra incógnita
        \startformula
          y = 5 - 4 \cdot 1 = 5 - 4
        \stopformula
        \startformula
          y = 1
        \stopformula
        $\therefore$ \inframed{$x = 1,\, y = 1$} o \inframed{$(1,1)$} o \inframed{$X = \{(1,1)\}$}
      \stopplaceformula

      \startplaceformula
        \startejerformula[color=ejemcolor]
          \startmathalignment[n=2, align={middle,middle}]
            \NC 2x = 14  -5y \NC\NR[+]
            \NC -2y + 3x +17 = 0 \NC\NR
          \stopmathalignment
        \stopejerformula
        La primera ecuación está casi resuelta para $x$,
        \startformula
          x = \dfrac{14 - 5y}{2} \leftarrow \text{ecuación auxiliar}
        \stopformula
        Ahora sustituimos en la otra ecuación
        \startformula
          -2y + 3\dfrac{14 - 5y}{2} + 17 = 0
        \stopformula
        Resolvemos. Primero eliminamos las fracciones
        \startformula
          \left\[-2y + 3\dfrac{14 - 5y}{2} + 17 = 0\right\] 2
        \stopformula
        \startformula
          - 4y + 3(14 - 5y) + 34 = 0
        \stopformula
        \startformula
          -4y + 42 - 15y  = -34
        \stopformula
        \startformula
          -19y = - 34 -42
        \stopformula
        \startformula
          -19y = -76
        \stopformula
        \startformula
          y = \dfrac{-76}{-19}
        \stopformula
        \startformula
          y = 4
        \stopformula
        Obtenemos la otra incógnita
        \startformula
          x = \dfrac{14 - 5(4)}{2} = \dfrac{14-20}{2} = \dfrac{-6}{2}
        \stopformula
        \startformula
          x = -3
        \stopformula
        $\therefore$ \inframed{$x = -3,\, y = 4$} o \inframed{$(-3,4)$} o \inframed{$X = \{(-3,4)\}$}
      \stopplaceformula

      \ \youtube{\from[AI57B]}
      \startplaceformula
        \startejerformula[color=ejemcolor]
          \startmathalignment[n=3, align={middle,middle,left}]
            \NC 3x -y + 2z  \NC = \NC 9 \NR[+]
            \NC 2x + y + -z \NC = \NC 7 \NR
            \NC x + 2y - 3z \NC = \NC 4 \NR
          \stopmathalignment
        \stopejerformula

        \startformula
          \[-z = 7 -2x -y\](-1)
        \stopformula
        \startformula
          z = 2x +y -7 \leftarrow \text{ecuación auxiliar}
        \stopformula

        \startformulas
          \startformula
            \startmathalignment[n=3, align={middle,middle,left}]
              \NC 3x - y + 2(2x + y -7) \NC = \NC 9 \NR
              \NC x + 2y -3(2x + y -7) \NC = \NC 4 \NR
            \stopmathalignment
          \stopformula
          \startformula
            \Rightarrow
          \stopformula
          \startformula
            \startmathalignment[n=3, align={middle,middle,left}]
              \NC 3x - y + 4x + 2y -14 \NC = \NC 9 \NR
              \NC x + 2y -6x -3y + 21 \NC = \NC 4 \NR
            \stopmathalignment
          \stopformula
          \startformula
            \Rightarrow
          \stopformula
        \stopformulas
        \startformulas
          \startformula
            \startmathalignment[n=3, align={middle,middle,left}]
              \NC 7x + y  \NC = \NC 9 + 14 \NR
              \NC -5x - y \NC = \NC 4 - 21\NR
            \stopmathalignment
          \stopformula
          \startformula
            \Rightarrow
          \stopformula
          \startformula
            \startmathalignment[n=3, align={middle,middle,left}]
              \NC 7x + y  \NC = \NC 23  \NR
              \NC -5x - y \NC = \NC -17 \NR
            \stopmathalignment
          \stopformula
          \startformula
            .
          \stopformula
        \stopformulas
        Ahora tenemos un sistema de dos ecuaciones con dos incógnitas. Tendremos que aplicar de nuevo el método de sustitución para resolverla.
        \startformula
          y = 23 - 7x \leftarrow\text{segunda ecuación auxiliar}
        \stopformula
        \startformula
          -5x - (23 - 7x) = -17
        \stopformula
        \startformula
          -5x -23 + 7x = -17
        \stopformula
        \startformula
          2x = -17 + 23
        \stopformula
        \startformula
          2x = 6
        \stopformula
        \startformula
          x = \dfrac{6}{2} = 3
        \stopformula
        Ahora obtenemos las otras incógnitas
        \startformula
          y = 23 - 7(3) =23 - 21
        \stopformula
        \startformula
          y = 2
        \stopformula
        \startformula
          z = 2(3) + 2 - 7 = 6 + 2 - 7
        \stopformula
        \startformula
          z = 1
        \stopformula
        $\therefore$ \inframed{$x = 3,\, y = 2,\, z = 1$} o \inframed{$(3,2,1)$} o \inframed{$X =\{(3,2,1)\}$}
      \stopplaceformula

      \startplaceformula
        \startejerformula[color=ejemcolor]
          \startmathalignment[n=2, align={middle,left}]
            \NC \text{Resuelva para } (x,y) \text{ el sistema:} \NC\NR[+]
            \NC 3ax + 2ay = 7  \NC\NR
            \NC 4ax + 3ay = 10 \NC\NR
          \stopmathalignment
        \stopejerformula

        \startformula
          2ay = 7 - 3ax
        \stopformula
        \startformula
          y = \dfrac{7- 3ax}{2a} \leftarrow \text{ecuación auxiliar}
        \stopformula
        \startformula
          4ax + 3a\left(\dfrac{7- 3ax}{2a}\right) = 10
        \stopformula
        \startformula
          \left\[4ax + 3a\left(\dfrac{7- 3ax}{2a}\right) = 10\right\] 2a
        \stopformula
        \startformula
          8a^2x + 3a (7 - 3ax) = 20a
        \stopformula
        \startformula
          8a^2x + 21a - 9a^2x = 20a
        \stopformula
        \startformula
          -a^2x = 20a - 21a
        \stopformula
        \startformula
          -a^2x = -a
        \stopformula
        \startformula
          x = \dfrac{-a}{-a^2}
        \stopformula
        \startformula
          x = \dfrac{1}{a}
        \stopformula
        \startformula
          y = \dfrac{7 -3a \left(\dfrac{1}{a}\right)}{2a} = \dfrac{7 -3}{2a} = \dfrac{4}{2a}
        \stopformula
        \startformula
          Y = \dfrac{2}{a}
        \stopformula
        $\therefore$ \inframed{$x = \dfrac{1}{a},\, y =\dfrac{2}{a}$} o \inframed{$\left(\dfrac{1}{a},\,\dfrac{2}{a}\right)$} o \inframed{$\left\{\left(\dfrac{1}{a},\,\dfrac{2}{a}\right)\right\}$}
      \stopplaceformula
    \stopejemplos
  \stopsection

  \youtube{\from[AI58A]}
  \startsection[title={Solución por suma y resta}]
    \startejemplos
      \startplaceformula
        \startejerformula[color=ejemcolor]
          \startmathalignment[n=3, align={middle,middle,left}]
            \NC 3x -y  \NC = \NC 5 \NR[+]
            \NC 3x + 2y \NC = \NC 8 \NR
          \stopmathalignment
        \stopejerformula
      \stopplaceformula

      \startcenteraligned
        \starttabulate [|mr|mr|mc|mr|]
          \NC 3x     \NC -y   \NC = \NC 5 \NR
          \NC (-) 3x \NC + 2y \NC = \NC 8 \NR
          \HL
          \NC        \NC -3y \NC = \NC -3 \NR
        \stoptabulate
      \stopcenteraligned
      \startformula
        -3y = -3
      \stopformula
      \startformula
        y = 1
      \stopformula
      Ahora sustituimos en cualquiera de las ecuciones del sistema original. Normalmente escojeremos la que tenga los coeficientes más simples. En este ejemplo será la primera.
      \startformula
        3x - (1) = 5
      \stopformula
      \startformula
        3x = 6
      \stopformula
      \startformula
        x = \dfrac{6}{3} = 2
      \stopformula
      $\therefore X = \{(2, 1)\}$

      \startplaceformula
        \startejerformula[color=ejemcolor]
          \startmathalignment[n=3, align={middle,middle,right}]
            \NC 2x -6y  \NC = \NC 26 \NR[+]
            \NC -5x + 8y \NC = \NC -44 \NR
          \stopmathalignment
        \stopejerformula
      \stopplaceformula

      Como no hay coeficientes iguales en valor absoluto en ninguna de las ecuaciones, vamos a multiplicar ambas ecuaciones por algún número de manera que obtengamos la igualdad de los coeficientes en valor absoluto.

      \startcenteraligned
        \starttabulate [|mr|mr|mc|mr|]
          \NC 5(2x     \NC -6y   \NC = \NC 26) \NR
          \NC 2(-5x \NC + 8y \NC = \NC -44) \NR
        \stoptabulate
      \stopcenteraligned

      \startcenteraligned
        \starttabulate [|mr|mr|mc|mr|]
          \NC  10x \NC -30y \NC = \NC 130 \NR
          \NC -10x \NC +16y \NC = \NC -88 \NR
          \HL
          \NC      \NC -14y \NC = \NC  42 \NR
        \stoptabulate
      \stopcenteraligned
      \startformula
        y = \dfrac{42}{-14} = -3
      \stopformula
      Sustituyendo en la primera ecuación
      \startformula
        2x -6 (-3) = 26
      \stopformula
      \startformula
        2x = 26 - 18
      \stopformula
      \startformula
        x = \dfrac{8}{2} = 4
      \stopformula

      $\therefore X = \{(4, -3)\}$
      \youtube{\from[AI58B]}

      \startplaceformula
        \startejerformula[color=ejemcolor]
          \startmathalignment[n=3, align={middle,middle,right}]
            \NC 2x +y -z  \NC = \NC -6 \NR[+]
            \NC -3x -4y +2z \NC = \NC 15 \NR
            \NC x +3y +5z \NC = \NC 3 \NR
          \stopmathalignment
        \stopejerformula
      \stopplaceformula

      Escojemos dos ecuaciones y eliminamos una de las incógnitas

      \startcenteraligned
        \starttabulate [|mr|mr|mr|mc|mr|]
          \NC 2x \NC  +y  \NC  -z \NC = \NC -6  \NR
          \NC 2(x \NC +3y \NC +5z \NC = \NC  3) \NR
        \stoptabulate
      \stopcenteraligned

      \startcenteraligned
        \starttabulate [|mr|mr|mr|mc|mr|]
          \NC 2x    \NC  +y \NC   -z \NC = \NC  -6 \NR
          \NC(-) 2x \NC +6y \NC +10z \NC = \NC   6 \NR
          \HL
          \NC       \NC -5y \NC -11z \NC = \NC -12 \NR
        \stoptabulate
      \stopcenteraligned

      Ahora seleccionamos la ecuación que quedaba y alguna otra ecuación y eliminamos la misma incógnita que en el caso anterior.

      \startcenteraligned
        \starttabulate [|mr|mr|mr|mc|mr|]
          \NC 3(2x \NC  +y \NC -z\NC = \NC -6) \NR
          \NC 2(-3x \NC -4y \NC +2z \NC = \NC 3) \NR
        \stoptabulate
      \stopcenteraligned

      \startcenteraligned
        \starttabulate [|mr|mr|mr|mc|mr|]
          \NC  6x \NC +3y \NC -3z \NC = \NC -18 \NR
          \NC -6x \NC -8y \NC +4z \NC = \NC  30 \NR
          \HL
          \NC     \NC -5y \NC  +z \NC = \NC 12 \NR
        \stoptabulate
      \stopcenteraligned

      Formamos un nuevo sistema de ecuaciones lineales con las dos ecuaciones resultantes.

      \startcenteraligned
        \starttabulate [|mr|mr|mc|mr|]
          \NC -5y \NC -11z \NC = \NC -12 \NR
          \NC(-) -5y \NC  +z \NC = \NC 12 \NR
          \HL
          \NC     \NC  -12z \NC = \NC -24 \NR
        \stoptabulate
      \stopcenteraligned

      \startformula
        -12z = -24
      \stopformula
      \startformula
        z = \dfrac{-24}{-12} = 2
      \stopformula

      Sustituyendo en la primera ecuación

      \startformula
        -5y = 11 \cdot 2 = -12
      \stopformula
      \startformula
        -5y - 22 = -12
      \stopformula
      \startformula
        -5y = -12 + 22
      \stopformula
      \startformula
        y = \dfrac{10}{-5} = -2
      \stopformula

      Sustituimos ahora en la tercera ecuación

      \startformula
        x + 3(-2) + 5 \cdot 2 = 3
      \stopformula
      \startformula
        x - 6 + 10 = 3
      \stopformula
      \startformula
        x = 3 -4
      \stopformula
      \startformula
        x = -1
      \stopformula
      $\therefore X = \{(-1, -2, 2)\}$

      \startplaceformula
        \startejerformula[color=ejemcolor]
          \startmathalignment[n=2, align={middle,middle,right}]
            \NC \text{Resuelva para } (x,y) \text{ el sistema:} \NC\NR[+]
            \NC ax -by = a^2 + b^2 \NC \NR
            \NC bx +ay = a^2 + b^2 \NC \NR
          \stopmathalignment
        \stopejerformula
      \stopplaceformula

      \startcenteraligned
        \starttabulate [|mr|mr|mc|mr|]
          \NC a(ax \NC -by \NC = \NC a^2 + b^2) \NR
          \NC b(bx \NC +ay \NC = \NC a^2 + b^2) \NR
        \stoptabulate
      \stopcenteraligned

      \startcenteraligned
        \starttabulate [|mr|mr|mc|ml|]
          \NC a^{2}x \NC -aby \NC = \NC a^3 +a b^2 \NR
          \NC b^{2}x \NC +aby \NC = \NC a^2 b + b^3 \NR
          \HL
          \NC a^2x + b^{2}x \NC  \NC = \NC a^3 + a b^2 + a^2 b + b^3) \NR
        \stoptabulate
      \stopcenteraligned

      \startformula
        x(a^2 + b^2) = a^3 + a b^2 + a^2 b + b^3
      \stopformula
      \startformula
        x = \dfrac{a^3 + a b^2 + a^2 b + b^3}{a^2 + b^2}
      \stopformula
      \startformula
        x = \dfrac{a(a^2 + b^2) + b(a^2 + b^2)}{a^2 + b^2}
      \stopformula
      \startformula
        x = \dfrac{(a^2 + b^2)(a+b)}{a^2 + b^2}
      \stopformula
      \startformula
        x = a + b
      \stopformula

      Sustituimos en una de las ecuaciones

      \startformula
        b(a + b) + ay = a^2 + b^2
      \stopformula
      \startformula
        ab + b^2 + ay = a^2 + b^2
      \stopformula
      \startformula
        ay = a^2 + b^2 - ab - b^2
      \stopformula
      \startformula
        ay = a (a-b)
      \stopformula
      \startformula
        y = \dfrac{a(a-b)}{a} = a - b
      \stopformula

      $\therefore X =\{(a+b, a-b)\}$
    \stopejemplos

    \youtube{\from[AI59A]}
    \startteorema
    Sean $a,b,c,d \in \reals$
    \startitemizer
      \startitem
        $a = b \wedge c = d \rightarrow a + c = b + d \wedge c + a = d + b$
      \stopitem
      \startitem
        $a = b \wedge c = d \rightarrow a - c = b - d$
      \stopitem
      \startitem
        $a = b \wedge c = d \rightarrow ac = db \wedge ca = bd$
      \stopitem
      \startitem
        $a = b \wedge c = d; \; c,d \neq 0, \rightarrow a \div c = b \div d$
      \stopitem
    \stopitemizer
  \stopteorema

  \startdemo
    $i)$ Por hipótesis, $a = b$. Entonces, por la ley de suma de las igualdades
    \placeformula
    \startformula
      a + c = b + c
    \stopformula
    Por hipótesis, también, $c = d$. Luego, según la ley de sustitución,
    \startformula
      a + c = b + d.
    \stopformula
    Como la suma es conmutativa, la igualdad anterior se puede escribir como
    \startformula
      c + a = d + b
    \stopformula

    Los otros casos quedan como ejercicios
  \stopdemo
  \stopsection

  \startsection[title={Solución por igualación}]
    \startejemplos
      Resuelva los siguientes sistemas por igualación

      \startplaceformula
        \startejerformula[color=ejemcolor]
          \startmathalignment[n=4, align={right,right,middle,left}]
            \NC  2x \NC +y \NC = \NC 5  \NR[+]
            \NC -2x \NC -y \NC = \NC -1 \NR
          \stopmathalignment
        \stopejerformula
      \stopplaceformula

      Despejamos $y$ en ambas ecuaciones

      \startformula
        y = 5 + 2x \leftarrow \text{1 ecuación auxiliar}
      \stopformula
      \startformula
        y = 1  - 2x \leftarrow \text{2 ecuación auxiliar}
      \stopformula

      Podemos forma la siguiente ecuación
      \startformula
        5 + 2x = 1 - 2x
      \stopformula
      \startformula
        2x + 2x = 1 - 5
      \stopformula
      \startformula
        4x = -4
      \stopformula
      \startformula
        x = \dfrac{-4}{4} = -1
      \stopformula

      Ahora sustuitmos en cualquiera de las ecuaciones

      \startformula
        y = 5 + 2(-1)
      \stopformula
      \startformula
        y = 5 - 2 = 3
      \stopformula
      $\therefore X = \{(-1, 3)\}$

      \startplaceformula
        \startejerformula[color=ejemcolor]
          \startmathalignment[n=3, align={right,middle,left}]
            \NC 3x -y +2z \NC = \NC 9 \NR[+]
            \NC 2x +y \NC = \NC 7 + z \NR
            \NC 2y +x \NC = \NC 3z +4 \NR
          \stopmathalignment
        \stopejerformula
      \stopplaceformula

      Lo primero será reescribir el sistema en la forma estándar.

      \startformula
        \startmathalignment[n=5, align={right,right,right,middle,left}]
          \NC 3x \NC  -y \NC +2z \NC = \NC 9 \NR
          \NC 2x \NC  +y \NC  -z \NC = \NC 7 \NR
          \NC  x \NC +2y \NC -3z \NC = \NC 4 \NR
        \stopmathalignment
      \stopformula

      Despejamos para la $y$.

      \resetnumber[formula]
      \placeformula
      \startformula
        y = 3x + 2z -9
      \stopformula
      \placeformula
      \startformula
        y = 7-2x + z
      \stopformula
      \placeformula
      \startformula
        y = \dfrac{4 -x +3z}{2}
      \stopformula

      Podemos igualar (1) y (2) y formamos una nueva ecuación. Y también (3) y (2).

      \startformula
        3x + 2z - 9 = 7 -2x +z
      \stopformula
      \startformula
        \dfrac{4 -x +3z}{2} = 7 -2x +z
      \stopformula

      Reordenando tendremos

      \startformula
        3x + 2z + 2x - z = 7 + 9
      \stopformula
      \startformula
        4 -x +3z = 14 - 4x + 2z
      \stopformula

      Seguimos simplificando
      \startformula
        5x + z = 16
      \stopformula
      \startformula
        -x + 3z +4x -2z = 14 -4
      \stopformula
      Resultará un nuevo sistema con dos incógnitas
      \startformula
        5x + z = 16
      \stopformula
      \startformula
        3x + z = 10
      \stopformula

      \youtube{\from[AI59B]}
      Para resolver el sistema despejamos la $z$
      \startformula
        z = 16 - 5x \leftarrow \text{1 ecuación auxiliar}
      \stopformula
      \startformula
        z = 10 - 3x \leftarrow \text{2 ecuación auxiliar}
      \stopformula
      Por la ley transitiva de la igualdad tenemos
      \startformula
        16 -5x = 10 - 3x
      \stopformula
      \startformula
        -5x +3x = 10 -16
      \stopformula
      \startformula
        -2x = -6
      \stopformula
      \startformula
        x = \dfrac{-6}{-2} = 3
      \stopformula
      Ahora sustituimos en alguna de las ecuaciones auxiliares
      \startformula
        z = 10 - 3 (3) = 10 - 9 = 1
      \stopformula
      Ahora podemos sustituir en (2), por tener los coeficientes más pequeños

      \startformula
        y = 7 - 2(3) + (1) = 7 - 6 + 1
      \stopformula
      \startformula
        y = 2
      \stopformula
      $\therefore X = \{(3, 2, 1)\}$

      \startplaceformula
        \startejerformula[color=ejemcolor]
          \startmathalignment[n=3, align={right,middle,left}]
            \NC -3x -2y \NC = \NC 1 \NR[+]
            \NC 3x -2y +3z \NC = \NC -1 \NR
            \NC 2x +4y +2z \NC = \NC 3 \NR
          \stopmathalignment
        \stopejerformula
      \stopplaceformula

      Como la primera ecuación ya tiene solo dos incógnitas (la $x$ y la $y$), resolvemos las otas dos ecuaciones del sistema para la incógnita $z$. De esa forma la segunda ecuación del sistema \quote{reducido} (donde la primera ecuación de ese sistema reducido es la primera ecuación del sistema original), no tendrá a la $z$.

      \startformula
        3z = -1 -3x +2y
      \stopformula
      \startformula
        z = \dfrac{-1 -3x +2y}{3} \leftarrow \text{1 ecuación auxiliar}
      \stopformula
      \startformula
        z = \dfrac{3 -2x -4y}{2} \leftarrow \text{2 ecuación auxiliar}
      \stopformula
      El sistema \quote{reducido} será
      \startformula
        -3x -2y = 1
      \stopformula
      \startformula
        \left[\dfrac{-1 -3x +2y}{3} = \dfrac{3 -2x -4y}{2}\right]6
      \stopformula
      Que resulta en
      \startformula
        -3x -2y = 1
      \stopformula
      \startformula
        2(-1-3x+2y) = 3(3-2x-4y)
      \stopformula
      Y continuamos simplificando
      \startformula
        -3x -2y = 1
      \stopformula
      \startformula
        -2 -6x +4y = 9 -6x -12y
      \stopformula
      Seguimos simplificando
      \startformula
        -3x -2y = 1
      \stopformula
      \startformula
        -6x +4y +6x -12y= 9 +2
      \stopformula
      Finalmente tenemos el sistema
      \startformula
        -3x -2y = 1
      \stopformula
      \startformula
        16y = 11
      \stopformula
      La incógnita $y$ se puede obtener fácilmente
      \startformula
        y = \dfrac{11}{16}
      \stopformula
      Y ahora podemos sustituir en la primera ecuación
      \startformula
        -3x -2(\dfrac{11}{16}) = 1
      \stopformula
      \startformula
        -3x -\dfrac{11}{8} = 1
      \stopformula
      \startformula
        \left[-3x -\dfrac{11}{8} = 1\right]8
      \stopformula
      \startformula
        -24x -11 = 8
      \stopformula
      \startformula
        -24x = 8 + 11 = 19
      \stopformula
      \startformula
        x = -\dfrac{19}{24}
      \stopformula
      Nos falta obtener la $z$. Podemos utilizar las ecuaciones auxiliares. Sustituimos en la primera por tener los coeficientes más pequeños.
      \startformula
        z = \dfrac{-1-3(-\dfrac{19}{24})+2(\dfrac{11}{16})}{3} = \dfrac{-1 + \dfrac{19}{8} + \dfrac{11}{8}}{3} = \dfrac{-1 + \dfrac{18}{4}}{3}
      \stopformula
    \stopejemplos
  \stopsection




  \startsection[title={Solución gráfica}]

  \stopsection

  \startsection[title={Observaciones interesantes}]

  \stopsection

  \startsection[title={Sistemas en los recíprocos de las incógnitas}]

  \stopsection

  \startsection[title={Problemas verbales}]

  \stopsection
\stopchapter
\stopcomponent
%%% Local Variables:
%%% mode: context
%%% TeX-master: t
%%% End:

\message{ !name(c_sistemas_ecuaciones_lineales.tex) !offset(-1055) }
