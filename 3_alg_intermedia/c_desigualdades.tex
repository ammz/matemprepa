\startcomponent c_desigualdades
\project project_matemprepa
% \product prod_algebra_intermedia

\youtube{\from[AI88A]}
\startchapter[title={Desigualdades}]
  \startsection[title={Intervalos}]

    Recordamos que una \obj{desigualdad} es cualquier enunciado matemático que contiene cualquiera o varios de los signos $<$, $>$, $\geq$ o $\leq$.

    Consideremos $a, b, c \in \reals$, localizados en una recta numérica.

    dibujo

    Por lo que estudiamos referente al orden de los números reales en el Álgebra Elemental
    \startformula
      a < b \quad\vee\quad b < c
    \stopformula
    Abreviamos lo anterior con el símbolo

    \resetnumber[formula]
    \setupformulas[align=middle]
    \placeformula
    \startformula
      a < b < c
    \stopformula
    También
    \startformula
      a > b \quad\vee\quad b > a
    \stopformula
    lo abreviamos con el símbolo
    \placeformula
    \startformula
      c > b > a
    \stopformula

    \startdefinicion
      Sean $a, b \in \reals$ con $a<b$. Los siguientes conjuntos se denominan \obj{intervalos acotados}:
      \startitemizer
        \startitem
          {\bf cerrado}:
          \startformula
            \{x \mid a \leq x \leq b\} \equiv \{x \mid b \geq x \geq a\} \equiv \[a, b\]
          \stopformula
        \stopitem
        \startitem
          {\bf abierto}:
          \startformula
            \{x \mid a < x < b\} \equiv \{x \mid b > x > a\} \equiv (a, b)
          \stopformula
        \stopitem
        \startitem
          {\bf semiabierto o semicerrado}:
          \startformula
            \{x \mid a \leq x < b\} \equiv \{x \mid b > x \geq a\} \equiv \[a, b)
          \stopformula
          \startformula
            \{x \mid  a < x \leq b\} \equiv \{x \mid b \geq x > a\} \equiv (a, b\]
          \stopformula
        \stopitem
      \stopitemizer
    \stopdefinicion
    \startdefinicion
      Sean $a, b \in \reals$. Los siguientes conjuntos se llaman \obj{intervalos no acotados}:
      \startitemize[i][placestopper=no, style=\sl, right=),]
        \startitem
          {\bf  cerrado}:
          \startformula
            \{x \mid x \geq b \} \equiv \{x \mid b \leq x \} \equiv \{x \mid b \leq x < \infty \} \equiv \{x \mid \infty > x \geq b\} \equiv \[a, \infty)
          \stopformula
          \startformula
            \{x \mid x \leq a \} \equiv \{x \mid a \geq x \} \equiv \{x \mid a \geq x > -\infty \} \equiv \{x \mid -\infty < x \leq a\} \equiv (-\infty, a \]
          \stopformula
        \stopitem
        \startitem
          {\bf  abierto}:
          \startformula
            \{x \mid x > b \} \equiv \{x \mid b < x \} \equiv \{x \mid b < x < \infty \} \equiv \{x \mid \infty > x > b\} \equiv (b, \infty)
          \stopformula
          \startformula
            \{x \mid x < a \} \equiv \{x \mid a > x \} \equiv \{x \mid -\infty < x < a \} \equiv \{x \mid a > x > -\infty \} \equiv (a, \infty)
          \stopformula
        \stopitem
      \stopitemize
    \stopdefinicion

    \startobservacion
      Vea que los intervalos son conjuntos de números reales que se localizan en segmentos de la recta numérica.
    \stopobservacion

    \startsubsection[title={Representación de intervalos en la recta numérica}]
      Graficos

      \youtube{\from[AI88B]}
      \startejemplos
        Clasifique, represente de otras formas e ilustre, gráficamente, a los siguientes intervalos
        \startitemejem
          \startitem
            \ini{$A = \{x \mid -2 \leq x \leq 2 \}$}

            Se trata de un intervalo acotado cerrado.

            $\equiv \{x \mid 2 \geq x \geq -2 \} \equiv \[-2, 2\]$

            Gráficos
          \stopitem
          \startitem
            \ini{$B = \left(-\dfrac{2}{5}, 3\right)$}

            Se trata de un intervalo acotado abierto.

            $\equiv \left\{x \mid -\dfrac{2}{5} < x < 3\right\} \equiv \left\{x \mid 3 > x > -\dfrac{2}{5}\right\}$

            Gráficos
          \stopitem
          \startitem
            \ini{$\left[-5 \dfrac{1}{3}, -.5\right]$}
          \stopitem
        \stopitemejem
      \stopejemplos

    \stopsubsection
  \stopsection
\stopchapter
\stopcomponent
