\startcomponent c_radicales
\project project_matemprepa
% \product prod_algebra_intermedia

\youtube{\from[AI1A]}
\startchapter[title={Radicales}]

\startsection[title={Raíces}]
    
  \startdefinicion
    \startitemizer
      \startitem
        Sea $a \in \reals^+ \cup \{0\}$ y sea $n \in \naturalnumbers$ par. \obj{La enésima raíz principal de $a$} es aquel número $r \in \mathbb{R^+} \cup \{0\}$ de modo que $r^n = a$.
      \stopitem
      \startitem
        Sea $a \in \reals$ y $n \in\naturalnumbers$, impar, $n \geq 3$. \obj{La enésima raíz principal de $a$} es aquel número $r \in \reals$ de manera que $r^n = a$.
      \stopitem
    \stopitemizer
  \stopdefinicion

  En cualquiera de los dos casos representamos a la enésima raíz principal de $a$ con el símbolo $\sqrt[n]{a}$. En éste, al signo $\sqrt{\phantom{x}}$ se le llama \obj{un radical}, al número $a$ se le llama \obj{el radicando} y al número $n$ se le llama \obj{el índice del radical} y se dice que da el \obj{orden} de éste. Si $n=2$, no se escribe, esto es, $\sqrt[2]{a} \equiv \sqrt{a}$.

  \startejemplos
    \startitemejem
      \startitem
        $\ini{\sqrt[3]{27} = r;\quad} r = 3 \;\text{ ya que }\; 3^3 = 27$
      \stopitem
      \startitem
        $\ini{\sqrt[5]{-32} = r;\quad} r = -2 \;\text{ ya que }\; (-2)^5 = -32$
      \stopitem
      \startitem
        $\ini{\sqrt{36} = r \in \reals^{+} \cup {0};\quad} r = 6 \;\text{ puesto que }\; 6^2 = 36$
      \stopitem
      \startitem
        $\ini{\sqrt[4]{81} = r \in \reals^{+} \cup {0};\quad} r = 3 \;\text{ ya que }\; 3^4 = 81$
      \stopitem
      \startitem
        $\ini{\sqrt[6]{-8}\quad}$ no está definido pues $n=6$, un número par, y $-8 \in \reals^{+}$
      \stopitem
    \stopitemejem
  \stopejemplos

  \startobservacion
    Hemos visto que $\sqrt{36} = 6$ puesto que $6^2 = 36$; pero también observamos que $(-6)^2 = 36$. Decimos, entonces, que -6 también es raíz cuadrada de 36, pero no es la raíz cuadrada principal.

    Igualmente, $\sqrt[4]{81} = 3,\; 3^4 = 81;$ pero $(-3)^4 = 81$. Luego, -3 también es raíz cuarta de 81, pero no es la raíz cuarta principal.
  \stopobservacion

  \youtube{\from[AI1B]}
  En el caso de una raíz de orden par hay dos raíces, una positiva y otra negativa, de las cuales es la positiva la que es la principal. 

  La situación anterior no ocurre si el índice $n$ de un radical es impar. Habrá una sola enésima raíz impar principal positiva, si el radicando es positivo, y una enésima raíz impar negativa, si el radicando es negativo.

  \startacuerdo
    Si $n$ es par, la enésima raíz principal de un número $a \in \reals \cup \{0\}$ se representará como $\sqrt[n]{a}$. Si quisiéramos la negativa escribiremos $-\sqrt[n]{a}$; y si quisieramos ambas escribiremos $\pm\sqrt[n]{a}$.
  \stopacuerdo

  \startteorema
    Sean $x,y \in \reals$ y $n \in \naturalnumbers$, de modo que $\sqrt[n]{x}$ y $\sqrt[n]{y}$ están definidas \footnote {Esto supone que si el índice $n$ es par, $x$ e $y$ tienen que ser positivos o cero. Si $n$ fuera impar $x$ e $y$ pueden ser cualquier número $\reals$.}. Entonces:
    \startitemizer
      \startitem
        $\left(\sqrt[n]{x}\right)^n = x$
      \stopitem
      \startitem
        $\sqrt[n]{y^n} = y$, si $n$ es impar.
      \stopitem
    \stopitemizer
  \stopteorema

  \startdemo
    $(i)$ Sea $r = \sqrt[n]{x}$. Entonces, por definición de enésima raíz principal de $a$, $r^n = x$. Si sustituimos por $r$ en esta última igualdad, obtenemos que $\left(\sqrt[n]{x}\right)^n = x$.

    $(ii)$ Sea $y = \sqrt[n]{a}$. Entonces, por definición de la enésima raíz principal, $y^n = a$. Entonces, si sustituimos por $a$, en la primera de estas dos igualdades, $y = \sqrt[n]{y^n}$, que puede reescribirse, según la ley simétrica de la igualdad, $\sqrt[n]{y^n} = y$. Observe que $\sqrt[n]{a}$ está definida ya que, por hipótesis, $n$ es impar.
  \stopdemo

  \startejemplos
    \startitemejem[columns,joinedup]
      \startitem
        $\ini{\left(\sqrt{5}\right)^2} = 5$
      \stopitem
      \startitem
        $\ini{\left(\sqrt[5]{-6}\right)^5} = -6$
      \stopitem
      \startitem
        $\ini{\sqrt[3]{(-6)^3}} = -6$
      \stopitem
      \startitem
        $\ini{\sqrt[61]{3^{61}}} = 3$
      \stopitem
      \startitem
        $\ini{\left(\sqrt[4]{2}\right)^4} = 2$      
      \stopitem
    \stopitemejem
    \startitemejem[start=6]
      \startitem
        $\ini{\left(\sqrt[6]{-3}\right)^6}$ no se puede aplicar $i)$, ya que $\sqrt[6]{-3}$ no está definida.      
      \stopitem
    \stopitemejem
  \stopejemplos

  \startteorema 
    \youtube[method=top]{\from[AI2A]}
    Sea $x \in \reals$ y $n \in \naturalnumbers$ par. Entonces $\sqrt[n]{x^n} = |x|$.
  \stopteorema

  \startdemo
    Si $x \geq 0$, entonces $|x| = x$. Luego, por la ley de sustitución, $|x|^n = x^n$. Sabemos, del álgebra elemental, que $|x| \in \mathbb{R^+} \cup \{0\},\, \forall\, x \in \reals$. En consecuencia, por definición de enésima raíz principal par, $\sqrt[n]{x^n} = |x|$.

    Si $x < 0$, entonces, por definición de valor absoluto, $|x| = -x$. Luego,
    
    \startformula \startalign
      \NC |x|^n \NC= (-x)^n = \underbrace{(-x)(-x)(-x) \dots (-x)}_{n-\text{factores}} \NR
      \NC \NC = \underbrace{(-x)(-x)(-x)(-x) \dots (-x)}_{2m-\text{factores, } m \in \naturalnumbers} \NR
      \NC \NC = \underbrace{[(-x)(-x)][(-x)(-x)] \dots [(-x)(-x)]}_{m-\text{factores } [(-x)(-x)]} \NR
      \NC \NC = \underbrace{[x^2][x^2] \dots [x^2]}_{m-\text{factores } x^2} = [x^2]^m = x^{2m} = x^n.\NR
    \stopalign \stopformula

    $\therefore |x|^n = x^n$, por la ley transitiva de la igualdad.

    Recordamos, otra vez, que $|x| \in \mathbb{R^+} \cup \{0\}$. En consecuencia, por la definición de enésima raíz principal par, $\sqrt[n]{x^n} = |x|$.
  \stopdemo

  \startejemplos
    \startitemejem[columns,joinedup]
      \startitem
        $\ini{\sqrt{5^2}} = |5| = 5$
      \stopitem
      \startitem
        $\ini{\sqrt[4]{(-8)^2}} = |-8| = 8$
      \stopitem
      \startitem
        $\ini{\sqrt{c^2}} = |\,c\,|$
      \stopitem
      \startitem
        $\ini{\sqrt[4]{d^4}} = |\,d\,|$
      \stopitem
      \startitem
        $\ini{\sqrt[8]{^8}} = |\,f\,|$
      \stopitem
    \stopitemejem
  \stopejemplos

\stopsection

\startsection[title={Simplificación de radicales}]

  \startteorema {Propiedades de los radicales}
    \youtube{\from[AI2B]}
    Sean $x, y \in \reals$ y $m, n \in \naturalnumbers, m, n \geq 2$, de manera que las siguientes raíces están definidas. Entonces:
    \startitemizer
      \startitem
        $\sqrt[n]{x y} = \sqrt[n]{x}\, \sqrt[n]{y}$
      \stopitem
      \startitem
        $\sqrt[n]{\dfrac{x}{y}} = \dfrac{\sqrt[n]{x}}{\sqrt[n]{y}}$, si $y \neq 0$
      \stopitem
      \startitem
        \obj{(radicales compuestos)} $\sqrt[mn]{x} = \sqrt[m]{\sqrt[n]{x}} = \sqrt[n]{\sqrt[m]{x}}$
      \stopitem
    \stopitemizer
  \stopteorema

  \startdemop
    % \comentario{$\sqrt[n]{a} = r; \; r^{n}=a$  \\ $(ab)^{n} = a^{n} b^{n}$ \\ $\left(\sqrt[n]{x}\right)^{n}=x$}
    $(i)$ Bastará ver que $\left(\sqrt[n]{x}\; \sqrt[n]{y}\right)^n = xy$.

    \startformula
      \left(\sqrt[n]{x} \; \sqrt[n]{y}\right)^n = \left(\sqrt[n]{x}\right)^n \left(\sqrt[n]{y}\right)^n = xy
    \stopformula

    \startformula
      \therefore \left(\sqrt[n]{x} \;\sqrt[n]{y}\right)^n = xy
    \stopformula

    $(ii)$ Esta prueba queda pendiente.

    % \comentario{$(a^m)^{n} = a^{mn}$}
    $(iii)$ Bastará ver que $\left(\sqrt[m]{\sqrt[n]{x}}\right)^{mn} = x$

    \startformula
      \left(\sqrt[m]{\sqrt[n]{x}}\right)^m = \sqrt[n]{x}
    \stopformula
    
    \startformula
      \left\[\left(\sqrt[m]{\sqrt[n]{x}}\right)^m\right\]^n = (\sqrt[n]{x})^n
    \stopformula

    \startformula
      \therefore \left(\sqrt[m]{\sqrt[n]{x}}\right)^{mn} = x
    \stopformula
  \stopdemop

  \startejemplos
    \ini{Simplifique (reduzca el radicando y/o el índice) de los siguientes radicales.}
    \startitemejem
      \startitem
        $\ini{\sqrt{125}} = \sqrt{5^3} = \sqrt{5^2 \cdot 5} = \sqrt{5^2} \cdot \sqrt{5} = |\,5\,
        | \sqrt{5} = 5\sqrt{5 }$
      \stopitem
      \startitem
        $\ini{\sqrt[3]{72}} = \sqrt[3]{2^3\cdot 9} = 2 \sqrt[3]{9}$
      \stopitem
      \startitem
        $\ini{\sqrt{72}} = \sqrt{3^2 \cdot 8} = 3\sqrt{8} = 3 \sqrt{4 \cdot 2} = 3 \cdot 2 \sqrt{2} = 6 \sqrt{2}$
      \stopitem
    \stopitemejem
  \stopejemplos

  \youtube{\from[AI3A]}
  \startejemplos
    Simplifique
    \startitemejem
      \startitem
        $\ini{\sqrt[3]{54 a^2 b^3 c^5 d^7}} = \sqrt[3]{3^3 \cdot 2 a^2 b^3 c^5 d^7} = 3bcd^2\,\sqrt[3]{2 a^2 c^2 d}$
      \stopitem
      \startitem
        $\ini{\sqrt[4]{25x^2y^6z^4}} = |\,y\,|\,|\,z\,|\,\sqrt[{2 \cdot 2}]{25x^2y^2} = |\,y\,|\,|\,z\,|\, \sqrt{\sqrt{25x^2y^2}} = |\,y\,|\,|\,z\,|\,\sqrt{5\,|\,x\,|\,|\,y\,|}$
      \stopitem
      \youtube{\from[AI3B]}
      \startitem
        % \comentario{
        %   $64 = 2^6 = \left(2^2\right)^3 = \left(2^3\right)^2;$\\
        %   $m^6 = \left(m^2\right)^3 = \left(m^3\right)^2;$\\
        %   $y^3 = (y)^3;\quad p^9 = \left(p^3\right)^3$
        % }
        $\ini{-\sqrt[12]{64m^6y^3p^4}, m,y,p \geq 0}$\\
        $= - \sqrt[12]{\left(4m^2yp^3\right)^3} = \sqrt[{4 \cdot 3}]{\left(4m^2yp^3\right)^3} = - \sqrt[4]{\sqrt[3]{\left(4m^2yp^3\right)^3}} = - \sqrt[4]{4m^2yp^3}$
      \stopitem
      \startitem
        $\ini{\sqrt[3]{2x^2y}\;\sqrt[3]{3x^2y^2}} = \sqrt[3]{\left(2x^2y\right)\left(2x^2y^2\right)} = \sqrt[3]{6x^4y^3} = xy\,\sqrt[3]{6x}$
      \stopitem
      \startitem
        $\ini{\dfrac{\sqrt[5]{20x^3y^4}}{\sqrt[5]{4x^2y}}} = \sqrt[5]{\dfrac{20x^3y^4}{4x^2y}} = \sqrt[5]{5xy^3}$
      \stopitem
    \stopitemejem
  \stopejemplos

  \startejemplos
    Simplifique
    \startitemejem
      \startitem
        $\ini{\sqrt{\sqrt[4]{b}}} = \sqrt[8]{b}$
      \stopitem
      \startitem
        $\ini{\sqrt[5]{\sqrt[7]{32}}} = \sqrt[7]{\sqrt[5]{32}} = \sqrt[7]{\sqrt[5]{2^5}} = \sqrt[7]{2}$
      \stopitem
    \stopitemejem
  \stopejemplos

\stopsection

\startsection[title={Exponentes fraccionarios}]

  Suponga que $b \in \reals$ y $n \in \naturalnumbers$. ¿Qué significará la expresión $b^{\frac{1}{n}}$?

  Suponga que la ley de exponentes, $\left(x^m\right)^n = x^{mn}$, es aplicable a exponentes fraccionarios, entonces $\left(b^{\frac{1}{n}}\right)^n = b^{\frac{1}{n}\cdot n} = b$. Sabemos que $\left(\sqrt[n]{b}\right)^n = b$ siempre que $\sqrt[n]{b}$ esté definido.

  \startdefinicion
    Sean $b \in \reals$ y $n \in \naturalnumbers$. Entonces $b^{\frac{1}{n}} = \sqrt[n]{b}$, siempre que $\sqrt[n]{b}$ esté definida.
  \stopdefinicion

  \startejemplos
    \startitemejem
      \startitem
        $\ini{2^{\frac{1}{2}}} = \sqrt{2}$
      \stopitem
      \startitem
        $\ini{3^{\frac{1}{4}}} = \sqrt[4]{3}$
      \stopitem
      \startitem
        $\ini{(-8)^{\frac{1}{3}}} = \sqrt[3]{-8} = -2$
      \stopitem
      \startitem
        $\ini{(-6)^{\frac{1}{8}}} \neq \sqrt[8]{-6}$, ya que $\sqrt[8]{-6}$ no está definida.
      \stopitem
    \stopitemejem
  \stopejemplos

  Suponga que $b \in \reals$ y $m,n \in \naturalnumbers$. ¿Qué siginificará la expresión $b^{\frac{m}{n}}$. Si volvemos a suponer que la ley de exponentes es aplicable a exponentes fraccionarios, $b^{\frac{m}{n}} = b^{\frac{1}{n}\cdot m} = \left(b^{\frac{1}{n}}\right)^m = \underbrace{b^{\frac{1}{n}} \cdot b^{\frac{1}{n}} \cdot b^{\frac{1}{n}} \;\cdots\; b^{\frac{1}{n}}}_{m-\text{factores}} = \underbrace{\sqrt[n]{b} \cdot \sqrt[n]{b} \cdot \sqrt[n]{b} \;\cdots\; \sqrt[n]{b}}_{m-\text{factores}} = \left(\sqrt[n]{b}\right)^m = \sqrt[n]{\underbrace{b \cdot b \cdot b \cdots b}_{m-\text{factores}}} = \sqrt[n]{b^m}$ 

  \startdefinicion
    Sean $b \in \reals$ y $m, n \in \naturalnumbers$. Entonces $b^{\frac{m}{n}} = \sqrt[n]{b^m} = \left(\sqrt[n]{b}\right)^m$, siempre que estén definidas.
  \stopdefinicion


  \startejemplos
    \youtube[method=top]{\from[AI4B]}
    \startitemejem
      \startitem
        \ini{Demuestre que si $x \in \reals$ y $m, n \in \rationals^+$, entonces $(x^m)^n = x^{mn}$.}

        \startdemoejem
          Sean $m = \frac{p}{q}$ y $n = \frac{r}{s}$, donde $p, q, r, s \in \naturalnumbers$.  Luego, $(x^m)^n
          = (x^{\frac{p}{q}})^{\frac{r}{s}}
          = \sqrt[s]{\left(x^{\frac{p}{q}}\right)^r}
          = \sqrt[s]{\left(\sqrt[q]{x^p}\right)^r}
          = \sqrt[s]{\sqrt[q]{\left(x^p\right)^r}}
          = \sqrt[s]{\sqrt[q]{x^{pr}}}
          = \sqrt[q]{\sqrt[s]{x^{pr}}}
          =  \sqrt[qs]{x^{pr}}
          = x^{\frac{pr}{qs}}
          = x^{\frac{p}{q}\cdot\frac{r}{s}}
          = x^{mn}$. Luego, por la ley transitiva de la igualdad, $(x^m)^n = x^{mn}$.
        \stopdemoejem
      \stopitem
      \startitem
        $\ini{32^{\frac{2}{5}}} = \sqrt[5]{32^2} = \left(\sqrt[5]{32}\right)^2 = 2^2 = 4$
      \stopitem
      \startitem
        $\ini{27^{-\frac{4}{3}}} = \dfrac{1}{27^{\frac{4}{3}}} = \dfrac{1}{\left(\sqrt[3]{27}\right)^4} = \dfrac{1}{3^4} = \dfrac{1}{81}$
      \stopitem
      \startitem
        $\ini{\sqrt{\left(\dfrac{4}{25}\right)^3}} = \left(\sqrt{\dfrac{4}{25}}\right)^3 = \left(\dfrac{\sqrt{4}}{\sqrt{25}}\right)^3 = \left(\dfrac{2}{5}\right)^3 = \dfrac{2^3}{5^3} = \dfrac{8}{125}$
      \stopitem
      \startitem
        $\ini{\sqrt[3]{\sqrt{9} + \sqrt{25}}} = \left(\sqrt{9} + \sqrt{25}\right)^{\frac{1}{3}} = (3 + 5)^{\frac{1}{3}} = 8^{\frac{1}{3}} = \sqrt[3]{8} = 2$
      \stopitem
    \stopitemejem
  \stopejemplos

  \startejemplos
    \youtube[method=top]{\from[AI5A]}
    Simplifique

    \startitemejem
      \startitem
        $\ini{\left(q^7r^{12}\right)^{1/3}} = \left(q^7\right)^{\frac{1}{3}}\left(r^{12}\right)^{\frac{1}{3}} = q^{\frac{7}{1}\frac{1}{3}} r^{\frac{12}{1}\frac{1}{3}} = q^{\frac{7}{3}} r^4$
      \stopitem
      \startitem
        $\ini{\left(\dfrac{2x^{\frac{2}{3}}}{y^{\frac{1}{3}}}\right)^2 \left(\dfrac{3x^{-\frac{5}{4}}}{y^{\frac{1}{5}}}\right)} = \dfrac{\left(2x^{\frac{2}{3}}\right)^2}{\left(y^{\frac{1}{3}}\right)^2} \left(\dfrac{3}{x^{\frac{5}{6}}y^{\frac{1}{3}}}\right) = \dfrac{2^2\left(x^{\frac{2}{3}}\right)^2}{y^{\frac{1}{3}\frac{3}{1}}}\cdot \dfrac{3}{x^{\frac{5}{6}}y^{\frac{2}{3}}} = \dfrac{4x^{\frac{2}{3}\frac{2}{1}}}{y} \cdot \dfrac{3}{x^{\frac{5}{6}}y^{\frac{1}{3}}} = \dfrac{4x^{\frac{2}{3}\frac{2}{1}}}{y} \cdot \dfrac{3}{x^{\frac{5}{6}}y^{\frac{1}{3}}} = \dfrac{4x^{\frac{4}{3}}}{y} \cdot                      \dfrac{3}{x^{\frac{5}{6}}y^{\frac{1}{3}}} = \dfrac{4^{\frac{4}{3}}(3)}{y\left(x^{\frac{5}{6}}y^{\frac{1}{3}}\right)} = \dfrac{12x^{\frac{4}{3}}}{x^{\frac{5}{6}}y^{1+\frac{1}{3}}} = \dfrac{12x^{\frac{8}{6}}}{x^{\frac{5}{6}}y^{\frac{3}{3}+\frac{1}{3}}} = \dfrac{12x^{\frac{8}{6}-\frac{5}{6}}}{y^{\frac{4}{3}}} = \dfrac{12x^{\frac{3}{6}}}{y^{\frac{4}{3}}} = \dfrac{12x^{\frac{1}{2}}}{y^{\frac{4}{3}}}$
      \stopitem
    \stopitemejem

  \stopejemplos

  \youtube{\from[AI5B]}
  Suponga que nos piden simplificar el radical $\sqrt[14]{x^{59}}$. Sabemos resolverlo aplicando el teorema $\sqrt[n]{xy} = \sqrt[n]{x} \sqrt[n]{y}$. Así,

  $ = \sqrt[14]{x^{14} x^{14} x^{14} x^{14} x^3} = |\,x\,|\, |\,x\,|\, |\,x\,|\, |\,x\,|\, \sqrt[14]{x^3} =
  x \cdot x \cdot x \cdot x \,\sqrt[14]{x^3} = x^4\, \sqrt[14]{x^3}$

  También aplicando la definición $\sqrt[n]{b^m} = b^{\frac{m}{n}}$ tenemos 

  % \comentario{donde vemos que 56 es divisible entre 14}
  $= \sqrt[14]{x^{56} \cdot x^3} = x^4 \, \sqrt[14]{x^3}$

  \startejemplos
    \startitemejem
      \startitem
        $\ini{\sqrt[2]{27x^4y^3z^{10}}} = \sqrt[3]{27x^3y^3z^9xz} = 3xyz\, \sqrt[3]{xz}$
      \stopitem
      \startitem
        $\ini{\sqrt[5]{96x^2y^5z^{14}}} = \sqrt[5]{2^5y^5z^{10}\cdot 3x^2z^4} = 2yz^2\, \sqrt[5]{2x^2z^4}$
      \stopitem
    \stopitemejem
  \stopejemplos

  \youtube[method=top]{\from[AI6A]}
  \startejemplos
    \startitemejem
      \startitem
        $\ini{-\sqrt[4]{3x^3y^2}\;\sqrt[6]{8x^3y^3}} = -\left(3x^3y^2\right)^{\frac{1}{4}} \left(8x^2y^3\right)^{\frac{1}{6}} = -\left(3x^3y^2\right)^{\frac{3}{12}} \left(8x^2y^3\right)^{\frac{2}{12}} =$

        $ -\sqrt[12]{\left(3x^3y^2\right)^3}\; \sqrt[12]{\left(8x^2y^3\right)^3} = - \sqrt[12]{27x^9y^6\cdot 64x^4y^6} = - \sqrt[12]{1728x^{13}y^{12}} =$
        
        $-|\,x\,|\,|\,y\,|\,\sqrt[12]{2^63^3x} = - xy\sqrt[12]{1728x}$ 
      \stopitem
      \youtube[method=top]{\from[AI6B]}
      \startitem
        $\ini{\dfrac{\sqrt{6xyz}}{\sqrt[4]{3xy^2}}} = \dfrac{(6xyz)^{\frac{1}{2}}}{\left(3xy^2\right)^{\frac{1}{4}}} = \dfrac{(6xyz)^{\frac{2}{4}}}{\left(3xy^2\right)^{\frac{1}{4}}} = \dfrac{\sqrt[4]{(6xyz)^2}}{\sqrt[4]{3xy^2}} = \sqrt[4]{\dfrac{36x^2y^2z^2}{3xy^2}} = \sqrt[4]{12xz^2}$
      \stopitem
    \stopitemejem
  \stopejemplos

  \startteorema {de reducción de índices} 

    Sean $x \in \reals$ y $m, n, r \in \naturalnumbers$. Entonces $\sqrt[nr]{x^{mr}} = \sqrt[n]{x^m}$, siempre que esta raíz esté definida.
  \stopteorema

  \startdemo
    $\sqrt[nr]{x^{mr}} = x^{\frac{mr}{nr}} = x^{\frac{m}{n}} = \sqrt[n]{x^m}$. En consecuencia, debido a la ley transitiva de la igualdad, $\sqrt[nr]{x^{mr}} = \sqrt[n]{x^m}$
  \stopdemo

  \startejemplos
    \startitemejem
      \startitem
        $\ini{\sqrt[8]{x^6}} = \sqrt[4]{x^3}$
      \stopitem
      \startitem
        $\ini{\sqrt[4]{25x^2y^6z^4}} = \sqrt[4]{y^4z^4 \cdot 25x^2y^2} = |\,y\,|\;|\,z\,|\;\sqrt[4]{(5xy)^2} = |\,y\,|\;|\,z\,|\;\sqrt{5xy}$
      \stopitem
    \stopitemejem
  \stopejemplos

  \youtube[method=top]{\from[AI7A]}
  \startejemplos
    Simplifique
    \startitemejem
      \startitem
        $\ini{-\sqrt[12]{4m^6y^3p^4}} = -\sqrt[12]{\left(4m^2y^3\right)^3} = -\sqrt[4]{4m^2y^3} $
      \stopitem
      \startitem
        Vea que $\sqrt[4]{(-3)^2} \neq \sqrt{-3}$, ya que $\sqrt{-3}$ no está definida. Sin embargo, $\sqrt[4]{(-3)^2} = \sqrt[4]{9} = \sqrt[4]{3^2} = \sqrt{3}$.
      \stopitem
    \stopitemejem
  \stopejemplos

\stopsection

\startsection[title={Operaciones con radicales}]

  \startejemplos
    % \comentario{Note que en una combinación de términos con radicales, procederemos como en una combinación de términos semejantes donde ahora los términos semejantes estarán determinados por aquéllos que contengan radicales iguales (con el misno radicando y el mismo índice).}

    \startitemejem
      \startitem
        $\ini{3\sqrt{x} + 5\sqrt{x}} = (3 + 5) \sqrt{x} = 8\sqrt{x }$
      \stopitem
      \startitem
        $\ini{\sqrt{3} + 2\sqrt[3]{3} -\sqrt{2} + 5 - 7\sqrt{3} - \sqrt{9} -6\sqrt{2}} = -6\sqrt{3} + 2\sqrt[3]{3} -7\sqrt{2} + 5 -3 = -6\sqrt{3} + 2\sqrt[3]{3} -7\sqrt{2} + 2$
      \stopitem
      \youtube{\from[AI7B]}
      \startitem
        $\ini{\sqrt{32} + \sqrt[3]{24} -\sqrt{2} +\ + 5\sqrt[3]{3} + 4\sqrt{8} -3\sqrt[3]{81}}$

        $= \sqrt{16 \cdot 2} + \sqrt[3]{8 \cdot 3} - \sqrt{2} + 1 + 5\sqrt[3]{3} + 4\sqrt{4 \cdot 2} - 3\sqrt[3]{27 \cdot 3}$

        $= 4\sqrt{2} + 2\sqrt[3]{3} -\sqrt{2} + 1 + 5\sqrt[3]{3} + 4 \cdot 2\sqrt{2} $

        $= 3\sqrt{2} + 7\sqrt[3]{3} + 1 + 8\sqrt{2} - 9\sqrt[3]{3} = 11\sqrt{2} -2\sqrt[3]{3} + 1$
      \stopitem
      % \comentario{Recuerda los productos especiales o notables}
      \startitem
        $\ini{(2\sqrt{x} -3\sqrt{y})(\-\sqrt{x} -4\sqrt{y})} = -2\sqrt{x^2} -5\sqrt{xy} + 12\sqrt{y^2} = -2\;|\,x\,| - 5\sqrt{xy} + 12\;|\,y\,| = -2x -5\sqrt{xy} +12y$ 
      \stopitem
      \startitem
        $\ini{(3\sqrt{5} -4)(2\sqrt{5}+1)} = 6\sqrt{25} -5\sqrt{5} -4 = 6 \cdot 5 -5\sqrt{5} -4 =30 -5\sqrt{5} -4 = 26 - 5\sqrt{5}$
      \stopitem
    \stopitemejem
  \stopejemplos

  \youtube{\from[AI8A]}
  \startejemplos
    \startitemejem
      \startitem
        $\ini{\left(2\sqrt{3} - 5\sqrt{6}\right)^3} = 8\sqrt{3^3} - 60\sqrt{3^2 \cdot 6} + 150\sqrt{6^2 \cdot 3} - 125\sqrt{6^3} $

        $ = 8\sqrt{3^2 \cdot 3} - 60 \cdot 3\sqrt{6} + 150 \cdot 6 \sqrt{3} - 125\sqrt{6^2 \cdot 6}$

        $ = 8\cdot 3\sqrt{3} - 180\sqrt{6} + 900\sqrt{3} - 125\cdot 6\sqrt{6}$


        $ = 24 \sqrt{3} - 180\sqrt{6} + 900\sqrt{3} - 750\sqrt{6}$
        
        $ = 924\sqrt{3} - 930 \sqrt{6} $
      \stopitem
      \startitem
        \ini{$\left(\sqrt{5} +3\sqrt{3} - 7\right)\left(\sqrt{10} - 2\sqrt{6} + \sqrt{3}\right)$}

        \startcenteraligned
          \starttabulate[|mr|mr|mr|mr|mr|mr|mr|mr|]
            \NC \sqrt{5}  \NC +3\sqrt{3}  \NC -7          \NC             \NC             \NC            \NC            \NC           \NC\NR
            \NC \sqrt{10} \NC -2\sqrt{6}  \NC +\sqrt{3}   \NC             \NC             \NC            \NC            \NC           \NC\NR
            \HL      
            \NC \sqrt{50} \NC +3\sqrt{30} \NC -7\sqrt{10} \NC             \NC             \NC            \NC            \NC            \NC\NR
            \NC           \NC -2\sqrt{30} \NC             \NC -6\sqrt{18} \NC +14\sqrt{6} \NC            \NC            \NC            \NC\NR
            \NC           \NC             \NC             \NC             \NC             \NC +\sqrt{15} \NC +3\sqrt{9} \NC -7\sqrt{3} \NC\NR
            \HL      
            \NC \sqrt{25 \cdot 2} \NC +\sqrt{30}  \NC -7\sqrt{10} \NC -6\sqrt{9 \cdot 2} \NC +14\sqrt{6} \NC +\sqrt{15} \NC +3\cdot 3 \NC -7\sqrt{3} \NC\NR
          \stoptabulate
        \stopcenteraligned
        
        $=5\sqrt{2} +\sqrt{30} -7\sqrt{10} -6\cdot 3\sqrt{2} +14\sqrt{6} +\sqrt{15}  +3\cdot 3 -7\sqrt{3}$

        $=5\sqrt{2} +\sqrt{30} -7\sqrt{10} -18\sqrt{2} +14\sqrt{6} +\sqrt{15} +9 -7\sqrt{3}$

        $=-13\sqrt{2} +\sqrt{30} -7\sqrt{10} +14\sqrt{6} +\sqrt{15} +9 -7\sqrt{3}$
      \stopitem
      
    \stopitemejem
  \stopejemplos

  Decimos que \obj{racionalizamos el denominador de una fracción común} si eliminamos cualquier radical que haya en éste. 

  Esta técnica se basa en los teoremas que nos indican $\left(\sqrt[n]{x}\right)^n = x;\quad \sqrt[n]{x^n} =
  \startcases
    \NC x,   \NC $n$ inpar \NR
    \NC |x|, \NC $n$ par   \NR
  \stopcases$

  \startejemplos
    \startitemejem
      \startitem
        $\ini{\dfrac{1}{\sqrt{3}}} = \dfrac{1}{\sqrt{3}}\cdot\dfrac{\sqrt{3}}{\sqrt{3}} = \dfrac{\sqrt{3}}{\left(\sqrt{3}\right)^2} = \dfrac{\sqrt{3}}{3}$
      \stopitem
      \startitem
        $\ini{\sqrt{\dfrac{2}{5}}} = \dfrac{\sqrt{2}}{\sqrt{5}} = \dfrac{\sqrt{2}}{\sqrt{5}} \cdot \dfrac{\sqrt{5}}{\sqrt{5}} = \dfrac{\sqrt{10}}{5}$
      \stopitem
      \startitem
        $\ini{-\dfrac{3}{\sqrt{6}}} = -\dfrac{3}{\sqrt{6}} \cdot \dfrac{\sqrt{6}}{\sqrt{6}} = -\dfrac{3\sqrt{6}}{6} = -\dfrac{\sqrt{6}}{2}$
      \stopitem
      \startitem
        $\ini{\dfrac{5x^2y}{\sqrt[3]{2x^2yz^6}}} = \dfrac{5x^2y}{z^2\sqrt[3]{2x^2}} \cdot \dfrac{\sqrt[3]{2^2xy^2}}{\sqrt[3]{2^2xy^2}} = \dfrac{5x^2y\,\sqrt[3]{4xy^2}}{z^2\sqrt[3]{2^3x^3y^3}} = \dfrac{5x^2y\,\sqrt[3]{4xy^2}}{z^22xy} = \dfrac{5x\,\sqrt[3]{4xy^2}}{2z^2}$
      \stopitem
      \startitem
        $\ini{\dfrac{3ab^2}{\sqrt[5]{27bc^4d^2}}} = \dfrac{3ab^2}{\sqrt[5]{27bc^4d^2}} \cdot \dfrac{\sqrt[5]{9b^4cd^3}}{\sqrt[5]{3^2b^4cd^3}} = \dfrac{3ab^2\,\sqrt[5]{9b^4cd^3}}{\sqrt[5]{3^5b^5c^5d^5}} = \dfrac{3ab^2\,\sqrt[5]{9b^4cd^3}}{3bcd} = \dfrac{ab\sqrt[5]{9b^4cd^3}}{cd}$
      \stopitem
    \stopitemejem
  \stopejemplos

  \youtube{\from[AI9A]}
  \startejemplo
    \ini{Simplifique $2x\sqrt{x} - 5b\sqrt{\dfrac{a}{b}} - 6a\sqrt{\dfrac{b}{a}} + \dfrac{12x^3}{\sqrt{2x}}$}

    $ = 2x\sqrt{x} - 5b\dfrac{\sqrt{a}}{\sqrt{b}} - 6a\dfrac{\sqrt{b}}{\sqrt{a}} + \dfrac{12x^2}{\sqrt{2x}} \cdot \dfrac{\sqrt{2x}}{\sqrt{2x}}$

    $ = 2x\sqrt{x} - 5b\dfrac{\sqrt{a}}{\sqrt{b}} \cdot \dfrac{\sqrt{b}}{\sqrt{b}} - 6a \dfrac{\sqrt{b}}{\sqrt{a}} \cdot \dfrac{\sqrt{a}}{\sqrt{a}} + \dfrac{12x^2 \sqrt{2x}}{2x}$

    $ = 2x\sqrt{x} - \dfrac{5b}{1} \cdot \dfrac{\sqrt{ab}}{b} - \dfrac{6a}{1} \cdot \dfrac{\sqrt{ab}}{a} + 6x \sqrt{2x}$

    $ = 2x\sqrt{x} - 5\sqrt{ab} - 6\sqrt{ab} + 6x \sqrt{2x}$

    $ = 2x\sqrt{x} - 11\sqrt{ab} + 6x \sqrt{2x}$
  \stopejemplo

\stopsection

\startdefinicion 
  Una expresión algebraica en cualquiera de las formas:

  \startformula
    a\sqrt{x} + b \quad\text{o}\quad a + b\sqrt{y} \quad\text{o}\quad a\sqrt{x} + b\sqrt{y}
  \stopformula

  se llama {\obj un binomio surdo de orden 2} (o {\obj de segundo orden}).
\stopdefinicion

\startobservacion
  Vea que un binomio surdo de orden 2 contiene una raíz cuadrada en, al menos, uno de sus términos.
\stopobservacion

% \comentario{Dado
%   $\left(x^m + y^n\right)\left(x^m-y^n\right) = \left(x^m\right)^2 - \left(y^n\right)^2$
% tendremos que
%   $\left(a\sqrt{x} + b\sqrt{y}\right)\left(a\sqrt{x} - b \sqrt{y}\right) = \left(a\sqrt{x}\right)^2 - \left(b\sqrt{y}\right)^2 = a^2x - b^2y$}

  \startejemplos
    \ini{Recionalice los denominadores}
    \startitemejem
      \startitem
        $\ini{\dfrac{1}{2+\sqrt{3}}} = \dfrac{1}{2 +\sqrt{3}} \cdot \dfrac{2-\sqrt{3}}{2 - \sqrt{3}} = \dfrac{2 - \sqrt{3}}{4 - 3} = \dfrac{2 - \sqrt{3}}{1} = 2 - \sqrt{3}$
      \stopitem
      \youtube{\from[AI9B]}
      \startitem
        $\ini{\dfrac{\sqrt{x} + y\sqrt{z}}{3\sqrt{x}-2y\sqrt{z}}} = \dfrac{\sqrt{x} + y\sqrt{z}}{3\sqrt{x}-2y\sqrt{z}} \cdot \dfrac{3\sqrt{x}+2y\sqrt{z}}{3\sqrt{x}+2y\sqrt{z}} = \dfrac{3\sqrt{x^2} + 5y\sqrt{xz} + 2y^2\sqrt{z^2}}{9x - 4y^2z}$

        $ = \dfrac{3\,|\,x\,|\,+ 5y\sqrt{xz} + y^2\,|\,z\,|}{9x-4y^2z} = \dfrac{3x + 5y\sqrt{xz} + 2y^2z}{9x-4y^2z}$
      \stopitem
      \startitem
        $\ini{\dfrac{15}{\sqrt{10} - \sqrt{5}}} = \dfrac{15}{\sqrt{10} - \sqrt{5}} \cdot \dfrac{\sqrt{10} - \sqrt{5}}{\sqrt{10} - \sqrt{5}} = \dfrac{15\big(\sqrt{10} + \sqrt{5}\big)}{10 - 5} = \dfrac{15\big(\sqrt{10} + \sqrt{5}\big)}{5} $

        $= 3\sqrt{10} + 3\sqrt{5}$
      \stopitem
      \startitem
        $\ini{\dfrac{\sqrt{3-2x}}{\sqrt{2+x}}} = \dfrac{\sqrt{3-2x}}{\sqrt{2+x}} \cdot \dfrac{\sqrt{2+x}}{\sqrt{2+x}} = \dfrac{\sqrt{(3-2x)(2+x)}}{2+x} = \dfrac{\sqrt{6 - x - 2x^2}}{2+x}$
      \stopitem
      \startitem
        $\ini{\dfrac{3}{\sqrt{3} + 1 - \sqrt{5}}} = \dfrac{3}{\big(\sqrt{3} + 1\big) - \sqrt{5}} \cdot \dfrac{\big(\sqrt{3} + 1\big) + \sqrt{5}}{\big(\sqrt{3} + 1\big) + \sqrt{5}}$

        $ = \dfrac{3\Big[\big(\sqrt{3} + 1\big) + \sqrt{5}\Big]}{\big(\sqrt{3} + 1\big)^2 - 5} = \dfrac{3\big(\sqrt{3} + 1 + \sqrt{5}\big)}{3 + 2\sqrt{3} + 1 - 5} = \dfrac{3\sqrt{3} + 3 + \sqrt{5}}{2\sqrt{3} - 1} \cdot \dfrac{2\sqrt{3} + 1}{2\sqrt{3} + 1} =$

        $ \dfrac{\big(3\sqrt{3} + 3 + \sqrt{5}\big) \big(2\sqrt{3} + 1\big)}{4 \cdot 3 - 1} = \dfrac{\big(3\sqrt{3} + 3 + \sqrt{5}\big) \big(2\sqrt{3} + 1\big)}{11} = \dfrac{21+9\sqrt{3}+6\sqrt{15}+3\sqrt{5}}{11}$
      \stopitem
    \stopitemejem
  \stopejemplos

% \startsection[title={Resumen}]

%   \startformula
%     \color[blue]{\left(\sqrt[n]{x}\right)^n} = x
%   \stopformula

%   \startformula
%     \color[blue]{\sqrt[n]{x^n}} = \startcases[distance=1em]
%       \NC x,   \NC $n$ inpar \NR
%       \NC |x|, \NC $n$ par   \NR
%     \stopcases
%   \stopformula

%   \startformula
%     \color[blue]{\sqrt[n]{xy}} = \sqrt[n]{x} \sqrt[n]{y}
%   \stopformula

%   \startformula
%     \color[blue]{\sqrt[n]{\frac{x}{y}}} = \frac{\sqrt[n]{x}}{\sqrt[n]{y}},\quad y \neq 0
%   \stopformula

%   \startformula
%     \color[blue]{\sqrt[mn]{x}} = \sqrt[m]{\sqrt[n]{x}} = \sqrt[n]{\sqrt[m]{x}}
%   \stopformula

%   \startformula
%     \color[blue]{b^{\frac{1}{n}}} = \sqrt[n]{b}
%   \stopformula

%   \startformula
%     \color[blue]{b^{\frac{m}{n}}} = \sqrt[n]{b^m} = \left(\sqrt[n]{b}\right)^m
%   \stopformula
  
%   \startformula
%     \color[blue]{|a|} = \startcases 
%       \NC a,  \MC a \geq 0 \NR
%       \NC -a, \MC a < 0    \NR
%     \stopcases
%   \stopformula
% \stopsection

\stopchapter
\stopcomponent
