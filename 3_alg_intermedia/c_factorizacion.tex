\startcomponent c_factorizacion
\project project_matemprepa
%\product prod_algebra_intermedia


\margindata[youtube]{\from[AI10A]}
\startchapter[title={Factorización}]

  Factorizar en este tema consiste en escribir expresiones algebraicas, que son multinomios (sumas o rectas de términos) como una multiplicación de otras dos expresiones algebraicas.


  \startsection[title={El factor común}]

    Esta técnica de factorización se fundamenta en las leyes distributivas:

    \startformulas
      \startformula
        \startalign
          \NC xy + xz \NC = x(y + z) \NR
          \NC xz + yz \NC = (x + y)z \NR
        \stopalign
      \stopformula

      \startformula
        \startalign
          \NC xy - xz \NC = x(y - z) \NR
          \NC xz - yz \NC = (x - y)z \NR
        \stopalign
      \stopformula
    \stopformulas

    Así, al observar un factor que aparezca en todos los términos de un multinomio, podremos extraer a éste como \obj{un factor común}, dejando dentro de un segundo factor a la suma o resta del resto de los términos consistente del resto de los factores que aparecían en cada uno de ellos inicialmente.

    \startejemplos
      Factorice completamente (o primamente)

      \startitemejem
        \startitem
          $\ini{3x + 6y} = 3(x + 2y)$
        \stopitem

        \startitem
          $\ini{30x + 25y^2 - 10z^3} = 5\big(6x + 5y^2 - 2z^3\big)$
        \stopitem

        \startitem
          $\ini{-6x^2 + 9y^3x} = 3x\big(-2x +3y^3\big)$
        \stopitem

        \startitem
          $\ini{8x^3y^2- 6x^2y -4x^2z +12x^5yz^2} = 2x^2\big(4xy^2  -3y -2z +6x^3yz^2\big)$
        \stopitem

        \startitem
          $\ini{-2(x+y) + 5z^2(x+y)} = (x + y)\big(-2 + 5z^2\big)$
        \stopitem
      \stopitemejem
    \stopejemplos

    \startobservacion
      Note que el factor variable común consiste de aquella letra que aparezca elevada al menor de los exponentes entre todos los términos del multinomio original.
    \stopobservacion

  \stopsection

  \margindata[youtube]{\from[AI10B]}
  \startsection[title={Factorización por agrupación}]

    Cualquier multinomio de cuatro o más términos se puede tratar de factorizar mediante la técnica de factorización por agrupación. Ésta se fundamenta en la ley asociativa de la suma de números reales. Se procede a agrupar varios términos para ser si de entre ellos se puede extraer algún factor común, o si los términos agrupados pueden ser factorizados por cualquier otro método de factorización que conozacamos. A veces, aplicando la ley conmutativa de la suma de números reales, puede ser conveniente cambiar el orden de los térnimos en el multinomio antes de proceder a efectuar la agrupación. Una vez extraído el posible factor común de entre los términos agrupados, entonces se ve si se puede extraer otro factor común que puede contener más de un término.

    Un último comentario, antes de proceder a ver ejemplos, es que siempre lo primero que hay que hacer al intentar factorizar un multinomio es ver si entre los términos de éste hay algún factor común.

    \startejemplos
      Factorice completamente (o primamente)

      \startitemejem
        \startitem
          $\ini{3x + 3y + ax + ay} = 3(x + y) + a(x + y) = (x + y)(3 + a)$
        \stopitem

        \startitem
          $\ini{3x^2 + x^3 + 3 + x} = x^2(3 + x) + (3 + x) = (3 + x)\big(x^2 + 1\big)$
        \stopitem

        \head
        $\ini{3a^2x^2 - 18bc^3y + 36bc^3xz + 12a^2xz + 9bc^3x^2 - 6a^2y}$
        
        sacamos factor común

        $= 3\big(a^2x^2 - 6bc^3y + 12bc^3xz + 4a^2xz + 3bc^3x^2 - 2a^2y\big)$

        Reordenamos los monomios

        $= 3\big(a^2x^2 + 4a^2xz - 2a^2y -6bc^3y + 12bc^3xz + 3bc^3x^2\big)$
        
        $= 3\Big[a^2\big(x^2 + 4xz - 2y\big) - 3bc^3\big(2y - 4xz - x^2\big)\Big]$

        cambiamos el signo del último factor

        $= 3 \Big[a^2\big(x^2 + 4xz - 2y\big) + 3bc^3\big(x^2 + 4xz - 2y\big)\Big]$

        $= 3 \Big[\big(x^2 + 4xz - 2y\big) \big(a^2 +3bc^3\big)\Big]$

        $= 3 \big(x^2 + 4xz - 2y\big) \big(a^2 +3bc^3\big)$
      \stopitemejem
    \stopejemplos

  \stopsection

  
  \margindata[youtube]{\from[AI11A]}
  \startsection[title={Factorización de binomios}]

    \startsubsection[title={Binomio que es la diferencia de dos cuadrados perfectos}]

      Recordemos que, por un producto especial,
      \startformula
        x^{2m} - y^{2n} = (x^m + y^n)(x^m - y^n) = (x^m)^2 - (y^n)^2
      \stopformula

      \startejemplos
        Factorice primamente

        \startplaceformula
          \startejerformula
            \ini{\frac{9}{x^4} - .16y^6} = \bigg(\frac{3}{x^2} + .4y^3\bigg)\bigg(\frac{3}{x^2} - .4y^3\bigg)
          \stopejerformula

          Puesto que $\quad\sqrt{\frac{9}{x^4}} = \frac{3}{x^2}; \quad\sqrt{.16y^6} = .4y^3$
        \stopplaceformula
      
        \startplaceformula
          \startejerformula
            \startalign
              \NC \NC \ini{3x^2y^4 - 27x^4z^8}\NR[+]
              \NC \NC = 3x^2\big(y^4 - 9x^2z^8\big)\NR
              \NC \NC \text{puesto que el segundo factor es diferencia de cuadrados perfectos}\NR
              \NC \NC = 3x^2 \big(y^2 + 3xz^4\big)\big(y^2 - 3xz^4\big) \NR
            \stopalign
          \stopejerformula
        \stopplaceformula

        \startplaceformula
          \startejerformula
            \startalign
              \NC \NC \ini{2xy^6 - 8x^3z^6a^2} \NR[+]
              \NC \NC = 2x\big(y^2 - 4x^2z^6a^2\big) \NR
              \NC \NC \text{puesto que el segundo factor es diferencia de cuadrados perfectos}\NR
              \NC \NC = 2x\big(y^2 + 2xz^3a\big)\big(y^2 - 2xz^3a\big)
            \stopalign
          \stopejerformula
        \stopplaceformula
      \stopejemplos

      \startobservacion
        En general, el binomio que es la suma de dos cuadrados no es factorizable.
      \stopobservacion
      
    \stopsubsection

    \startsubsection[title={Binomio que es la suma o la resta de dos cubos perfectos}]

      \startteorema
        Sean $m, n \in \integers$ y $x, y \in \reals$. Entonces:

        \startitemizer
          \startitem
            $x^{3m} + y^{3n} = \big(x^m + y^n\big)\big(x^{2m} - x^my^n + y^{2n}\big)$
          \stopitem
          \startitem 
            $x^{3m} - y^{3n} = \big(x^m - y^n\big)\big(x^{2m} + x^my^n + y^{2n}\big)$
          \stopitem
        \stopitemizer
      \stopteorema

      \margindata[youtube][method=top]{\from[AI11B]}
      \startdemo
        El apartado {\it (i)} queda como ejercicio.
        
        {\it (ii)} Comenzamos reescribiendo la primera expresión de forma conveniente, introduciendo dos parejas de térnimos formadas por una expresión algebraica y su inverso aditivo, que al sumarse dan cero.

        \startformula
          \startalign[n=3]
            \NC x^{3m} - y^{3n} = \NC x^{3m} + \Big[ \overbrace{x^{2m}y^n+\big( -x^{2m}y^n\big)}^{0} \Big] + \Big[ \overbrace{x^{m}y^{2n}+\big( -x^{m}y^{2n} \big)}^{0} \Big] - y^{3n} \NR
            \NC \NC \text{Por la ley asociativa de } \reals \text{ y la equivalencia de la resta} \NR
            \NC = \NC x^{3m} + x^{2m}y^n - x^{2m}y^n + x^{m}y^{2n} - x^{m}y^{2n} - y^{3n} \NR
            \NC \NC \text{Por la ley conmutativa de la suma} \NR
            \NC = \NC x^{3m} + x^{2m}y^n + x^{m}y^{2n} - x^{2m}y^n - x^{m}y^{2n} - y^{3n} \NR
            \NC \NC \text{Aplicando la técnica de factorización por agrupación} \NR
            \NC = \NC x^m \big( x^{2m} + x^my^n + y^{2n} \big) - y^n \big( x^{2m} + x^my^n + y^{2n} \big) \NR
            \NC \NC \text{Sacamos factor común y aplicamos la ley conmutativa del producto} \NR
            \NC = \NC \big( x^{2m} + x^my^n + y^{2n} \big)\big( x^m - y^n \big) = \big( x^m - y^n \big)\big( x^{2m} + x^my^n + y^{2n} \big) \NR
          \stopalign
        \stopformula

        Por la ley transitiva de la igualdad

        $\therefore\; x^{3m} - y^{3n} = \big( x^m - y^n \big)\big( x^{2m} + x^my^n + y^{2n} \big)$
      \stopdemo
      
      \startejemplos
        Factorice completamente.

        \startplaceformula
          \startejerformula
            \ini{8x^3 + 27y^6}
          \stopejerformula              

          Comprobamos si fuera una suma de cubos perfectos. Puesto que $\sqrt[3]{8x^3} = 2x$ y $\sqrt[3]{27y^6} = 3y^2$, podemos poner

          \startformula
            = \big( 2x + 3y^2 \big)\big( 4x^2 -  6xy^2 + 9y^4 \big).
          \stopformula
        \stopplaceformula

        \startplaceformula
          \startejerformula
            \ini{x^{12} - 64y^9}
          \stopejerformula

          Comprobamos si fuera una diferencia de cubos perfectos. Puesto que $\sqrt[3]{x^{12}} = x^4$ y $\sqrt[3]{64y^9} = 4y^3$, podemos poner

          \startformula
            = \big( x^4 - 4y^3 \big)\big( x^8 + 4x^4y^3 + 16y^6 \big).
          \stopformula
        \stopplaceformula
      \stopejemplos

      \margindata[youtube]{\from[AI12A]}
      \startejemplos

        Factorice primamente:
        
        \startplaceformula
          \startejerformula
            \ini{3x - 24x^4y^6} = 3x\big(1 - 8x^3y^6\big) = 3x\big( 1 - 2xy^2 \big)\big( 1 +2xy^2 + 4x^2y^4 \big)
          \stopejerformula
        \stopplaceformula

        \startplaceformula
          \startejerformula
            \ini{2a^3 + 54b^{15}} = 2\big(a^3 + 27b^{15}\big) = 2\big(a + 3b^5 \big)\big( a^2 - 3ab^5 + 9b^{10} \big)
          \stopejerformula
        \stopplaceformula

        \startplaceformula
          \startejerformula
            \startalign
              \NC \ini{x^{12} - y^6} = \NC \big( x^6 + y^3 \big) \big( x^6 - y^3 \big) \NR[+]
              \NC = \NC \big( x^2 + y \big)\big( x^4 - x^2y + y^2 \big) \big( x^2 - y \big)\big( x^4 + x^2y + y^2 \big)\NR
          \stopalign
          \stopejerformula


          Lo hemos factorizado como una diferencia de cuadrados perfectos. Pero observamos que también son una diferencia de cubos perfectos. Si lo factorizamos como diferencia de cubos perfectos tendríamos

          \startejerformula
            = \big( x^4 - y^2\big) \big(x^8 + x^4y^2 + y^4 \big) = \big( x^2 - y \big)\big( x^2 + y \big)\big(x^8 + x^4y^2 + y^4 \big)
          \stopejerformula

          Cuando se trata como una diferencia de cuadrados perfectos salen más factores que cuando lo tratamos como una diferencia de cubos perfectos. Para factorizar primamente siempre que tengamos una diferencia de dos términos que sean a la vez diferencia de cuadrados perfectos y de cubos perfectos, lo factorizaremos como diferencia de cuadrados perfectos.
        \stopplaceformula
      \stopejemplos

    \stopsubsection

    \margindata[youtube][method=top]{\from[AI12B]}
    \startsubsection[title={Factorización de un binomio que tiene dos términos elevados a una potencia}]

      Presentamos ahora un resumen de las fórmulas que tenemos para factorizar binomios que son sumas o diferencias de cubos

      \startformula
        \big( x^m \pm y^n\big)\underbracket{\big(x^{2m} \mp x^my^n + y^{2n}\big)} = \ini{x^{3m} \pm y^{3n}} = \big(x^m\big)^3 \pm \big(y^n\big)^3
      \stopformula

      Vemos que podemos pasar de un binomio que tiene dos términos elevados a una potencia a un binonio que tiene esos mismos dos términos elevados al cubo multiplicando ese binomio por un trinomio como el marcado arriba.
      
      Supongamos que tenemos un denominador de una fracción común con binomio surdo de orden tres, $a\sqrt[3]{x} \pm b\sqrt[3]{y}$. ¿Cómo lo racionalizaríamos? Pues, vemos que, teniendo en cuenta lo señalado anteriormente, si multiplicamos por el trinomio $\big( a\sqrt[3]{x} \big)^2 \mp ab\sqrt[3]{xy} + \big(a\sqrt[3]{x}\big)$ obtenemos:

      \startformula
        \big(a\sqrt[3]{x} \pm b\sqrt[3]{y}\big)\Big(a^2\sqrt[3]{x^2} \mp ab\sqrt[3]{xy} + b^2\sqrt[3]{y^2}\Big) = \big( a\sqrt[3]{x}\big)^3 \pm  \big(b\sqrt[3]{y} \big)^3 = a^3x \pm b^3y,
      \stopformula

      una expresión sin radicales.

      \startejemplos
        Racionalice los denominadores de las siguientes fracciones.

        \startplaceformula
          \startejerformula
            \startalign
              \NC \ini{\frac{20}{\sqrt[3]{9} - \sqrt[3]{4}}} =
              \NC \frac{20}{\sqrt[3]{9} - \sqrt[3]{4}} \cdot \frac{\sqrt[3]{9^2} + \sqrt[3]{9\cdot4} + \sqrt[3]{4^2}}{\sqrt[3]{9^2} + \sqrt[3]{9\cdot4} +  \sqrt[3]{4^2}} \NR[+]

              \NC =
              \NC \frac{20\Big(\sqrt[3]{81} + \sqrt[3]{36} + \sqrt[3]{16} \Big)}{\Big( \sqrt[3]{9}\Big)^3 - \Big( \sqrt[3]{4}\Big)^3} = \frac{20\Big(\sqrt[3]{27 \cdot 3} + \sqrt[3]{36} + \sqrt[3]{8 \cdot 2} \Big)}{9 - 4}\NR

              \NC =
              \NC \frac{20\Big(3\sqrt[3]{3} + \sqrt[3]{36} + 2\sqrt[3]{2} \Big)}{5} = 12\sqrt[3]{3} + 4\sqrt[3]{36} + 8\sqrt[3]{2} \NR
            \stopalign
          \stopejerformula
        \stopplaceformula

        \startplaceformula
          \startejerformula
            \startalign
              \NC \ini{\frac{1}{1 + 2\sqrt[3]{a^2}}} = \NC \frac{1}{1 + 2\sqrt[3]{a^2}} \cdot \frac{1 - 2\sqrt[3]{a^2} + 4\sqrt[3]{a^4}}{1 - 2\sqrt[3]{a^2} + 4\sqrt[3]{a^4}}\NR[+]
              \NC = \NC \frac{1 - 2\sqrt[3]{a^2} + 4\sqrt[3]{a^3 a}}{1 + 8a^2} = \frac{1 - 2\sqrt[3]{a^2} + 4a\sqrt[3]{a}}{1 + 8a^2}\NR
            \stopalign
          \stopejerformula
        \stopplaceformula
      \stopejemplos

    \stopsubsection

  \stopsection

  \margindata[youtube][method=top]{\from[AI13A]}
  \startsection[title={Factorización de trinomios}]

    \startsubsection[title={El trinomio cuadrado perfecto}]
      \startdefinicion
        Un trinomio en la forma $ax^{2m} + bx^my^n + cy^{2n}$, donde $m,n \in \integers$ fijos y $a,b,c \in \reals$ fijos, de modo que $\sqrt{a}, \sqrt{c} \in \rationals$ (o sea son cuadrados perfectos) y donde $|b| = 2\sqrt{a}\sqrt{c}$, se llama \obj{un trinomio cuadrado perfecto}.
      \stopdefinicion

      \startdiscusion{Características del trinomio cuadrado perfecto}
        \startitemize[n]
          \startitem
            El primero y el tercer término son cuadrados perfectos. Esto es, se les puede extraer la raíz cuadrada exacta.
          \stopitem
          \startitem
            El término del medio sin contar su signo, es el doble del producto de la raíz cuadrada del primero por la raíz cuadrada del tercero.
          \stopitem
        \stopitemize
      \stopdiscusion

      \startejemplos
        Los siguientes trinomios son cuadrados perfectos.
        \margindata[comentario]{En los ejemplos que siguen vamos a suponer que las variables representan números reales positivos}

        \startplaceformula
          \startejerformula
            \ini{4x^2 - 16xy^3 + 16y^6}
          \stopejerformula

          Tenemos que ver que el primer y tercer términos tiene raíces cuadradas exactas.

          \startformula
            \sqrt{4x^2} = 2x, \quad \sqrt{16y^6} = 4y^3
          \stopformula

          Además, hay que ver que el término de enmedio sin contar el signo es dos veces el producto de esas dos raíces cuadradas.

          \startformula
            2(2x)(4y^3) = 16xy^3 = \big|-16xy^3\big| 
          \stopformula
        \stopplaceformula

        \startplaceformula
          \startejerformula
            \ini{x^6y^2 + 6x^3yz^4 + 9z^8}
          \stopejerformula

          Comprobamos el primer y tercer término.

          \startformula
            \sqrt{x^6y^2} = x^3y, \quad \sqrt{9x^8} = 3z^4
          \stopformula

          Veamos ahora el término del medio.

          \startformula
            2\big( x^3y \big)\big( 3z^4 \big) = 6x^3yz^4 = \big|6x^3yz^4\big|
          \stopformula
        \stopplaceformula
      \stopejemplos

      \startejemplos
        Los siguientes trinomios no son cuadrados perfectos.
        \startitemejem
          \head
          \ini{$3x^4 - 6x^2y + 9y^2$}

          Vemos que $\sqrt{3x^4}$ no es exacta.

          \head
          \ini{$9a^2 - 10a + 4$}

          Tenemos que $\sqrt{9a^2} = 3a, \quad \sqrt{4} = 2$. Sin embargo, la segunda condición no se cumple, ya que $2(3a)(2) = 12a \neq |-10a|$.

          \head
          \ini{$4p^2 + 20pq^3 + 25q^5$}

          Resulta que $\sqrt{4p^2} = 2p$, pero $\sqrt{25q^5}$ no es exacta.

          \head
          \ini{$x^6 - 12x^2y^4 + 36y^8$}

          Comprobamos que $\sqrt{x^6}=x^3, \quad \sqrt{36y^8} = 6y^4$. Pero, $2\big(x^3\big)\big(6y^4\big) = 12x^3y^4 \neq \big|-12x^2y^4\big|$.
        \stopitemejem
      \stopejemplos

      \margindata[youtube][method=top]{\from[AI13B]}
      \startejemplo
        \ini{Si $4x^2 + 9y^6$ son el primero y tercer término de un trinomio cuadrado perfecto, ¿cuál podrá ser el término del medio?}

        Primeramente, verificamos que estos dos términos son cuadrados perfectos:

        \startformula
          \sqrt{4x^2} = 2x \;\text{ y }\; \sqrt{9y^6} = 3y^3.
        \stopformula

        Luego, por la segunda característica de un trinomio cuadrado perfecto, el término del medio podrá ser: $2(2x)\big(3y^3\big) = 12xy^3$ o $-12xy^3$, recordando que éste puede ser positivo o negativo.
      \stopejemplo

      \startdiscusion{Cálculo del coeficiente numérico del tercer término de un trinomio cuadrado perfecto}
        
        Consideremos, ahora, los primeros dos términos de un trinomio cuadrado perfecto: $ax^{2m} + bx^my^n$. ¿Cuál será el tercer término?

        En cuanto a coeficiente literal vemos que $x^m$ corresponderá a $\sqrt{x^{2m}}$, la raíz cuadrada del coeficiente literal del primer término. Luego, $y^n$ tiene que corresonder a la raíz cuadrada del coeficiente literal del tercer término. Esto es, $y^n = \sqrt{y^{2n}}$, por lo que $y^{2n}$ es el coeficiente literal del tercer término que estamos buscando.

        En cuanto al coeficiente numérico del tercer término, por la definición de trinomio cuadrado perfecto,

        \startformula
          |b| = 2\sqrt{a}\sqrt{c}
        \stopformula

        donde $c$ es el coeficiente numérico del tercer término. De esta igualdad, veamos cómo podemos obtener $c$.

        \startformula
          \frac{|b|}{2\sqrt{a}}  = \frac{\sqrt{a}\sqrt{c}}{2\sqrt{a}}
        \stopformula
        \startformula
          \frac{|b|}{2\sqrt{a}}  = \sqrt{c}
        \stopformula
        \startformula
          \left(\frac{|b|}{2\sqrt{a}} \right)^2  = c
        \stopformula
        \startformula
          \frac{|b|^2}{\big(2\sqrt{a}\big)^2}  = c 
        \stopformula
        \startformula
          \frac{b^2}{4a}  = c 
        \stopformula

        Por la ley transitiva de la igualdad tenemos

        \startformula
          c = \frac{b^2}{4a} 
        \stopformula

        que se conoce como la \obj{fórmula para hallar el coeficiente numérico del tercer término de un trinomio cuadrado perfecto}.
      \stopdiscusion

      \startejemplos
        \startitemejem
          \startitem
            \ini{Si $4x^4 -5x^2$ son los primeros dos términos de un trinomio cuadrado perfecto, ¿cuál será el tercer término?}

            Como en el segundo término sólo hay una variable y ésta corresponde a la raíz cuadrada del coeficiente literal del primer término, entonces el tercer término no tiene variables.

            Para determinar el coeficiente numérico del tercer término, hacemos uso de la fórmula

            \startformula
              c = \frac{b^2}{4a}, \;\text{ donde }\; b = -5 \text{ y } a = 4
            \stopformula

            \startformula
              c = \frac{(-5)^2}{4 \cdot 4} = \frac{25}{16}
            \stopformula

            Luego, el trinomio que buscamos es

            \startformula
              4x^4 -5x^2 + \frac{25}{16}
            \stopformula

            Podemos verificar si el resultado es correcto, comprobando si el resultado cumple con las caracteríticas de un trinomio cuadrado perfecto.

            Vemos que $\displaymath{\sqrt{4x^4} = 2x^2}$ y $\sqrt{\frac{25}{16}} = \frac{\sqrt{5}}{\sqrt{4}} = \frac{5}{4}$.

            Además, $2\big(2x^2\big)\Big(\frac{5}{4}\Big) = 5x^2 = \big|-5x^2\big|$
          \stopitem

          \margindata[youtube][method=top]{\from[AI14A]}
          \startitem
            \ini{Si $c^2 + 2xy^3$ son los primeros dos términos de un trinomio cuadrado perfecto determine el tercer término.}

            Recordemos que en el segundo término tiene que aparecer el doble de las raíces cuadradas del primer término y del tercero. Vemos que $x = \sqrt{x^2}$. Luego, $y^3$ tiene que ser la raíz cuadrada de la variable que va a aparecer en el tercer término. Por tanto, si la cuadramos 

            \startformula
              (y^3)^2 = y^6
            \stopformula 

            obtenemos el coeficiente literal del tercer término.

            Para obtener el coeficiente numérico, utilizamos la fórmula: $c = \dfrac{b^2}{4a},\;$ con $b=2$, $a=1$. Resultando
            \startformula
              c = \frac{2^2}{4 \cdot 1} =\frac{4}{4} = 1
            \stopformula

            Luego, el término cuadrado perfecto que buscamos es:

            \startformula
              x^2 + 2xy^3 + y^6
            \stopformula
          \stopitem
        \stopitemejem
      \stopejemplos

      \startteorema
        Sea $ax^{2m} \pm bx^ny^n + cy^{2n}$, un trinomio cuadrado perfecto. Entonces
        \startformula
          ax^{2m} \pm bx^ny^n + cy^{2n} = \big( \sqrt{a}\,x^m \pm \sqrt{c}\,y^n \big)^2.
        \stopformula
      \stopteorema

      \startdemo
        Por el producto especial del cuadrado de un binomio,
        \startformula
          \startalign
            \NC \big( \sqrt{a}\,x^m \pm \sqrt{c}\,y^n \big)^2 =
            \NC \big( \sqrt{a}\,x^m \big)^2 \pm 2\sqrt{a}\sqrt{c}\,x^m y^n + \big(\sqrt{c}\,y^n\big)^2 \NR

            \NC =
            \NC ax^{2m} \pm bx^my^n + cy^{2n},\NR
        \stopalign

        \stopformula
        pues, por definición del trinomio cuadrado perfecto, $b=2\sqrt{a}\sqrt{c}$.

        Finalmente, por las leyes transitiva y simétrica de la igualdad, tenemos

        \startformula
          ax^{2m} \pm b x^m y^n + c y^{2n} = \big(\sqrt{a}\,x^m \pm \sqrt{c}\, y^n\big)^2
        \stopformula
      \stopdemo

      \startejemplos
        Factorice primamente
        \startitemejem
          \startitem
            $\ini{x^2 + 4x + 4} = (x + 2)^2$
          \stopitem
          \startitem
            $\ini{72a^2 + x^6 y^4 -24x^3y^2za^2 + 2a^2z^2}$\par
            $= 2a^2\big(36x^6y^4 - 12x^3y^2z + z^2\big)$\par
            $= 2a^2\big(6x^3y^2 - z\big)^2$
          \stopitem

        \stopitemejem
      \stopejemplos
    \stopsubsection
    \margindata[youtube]{\from[AI14B]}
    \startsubsection[title={El trinomio cuadrático o de tipo cuadrático mónico y no mónico}]

      \startdefinicion
        Un trinomio en la forma $ax^{2m} + bx^my^n + cy^{2n}$, donde $m,n \in \integers$ fijos y $a,b,c \in \reals$ fijos, que no cumplen, necesariamente, con las condiciones de un trinomio cuadrado perfecto, se llama \obj{un trinomio de tipo cuadrático}. Si $m, n <= 1$, el trinomio se llama \obj{cuadrático}. Si $|a| = 1$ o si $|c| = 1$, el trinomio cuadrático o de tipo cuadrático se llama \obj{mónico}, de lo contrario se llama \obj{no mónico}.
      \stopdefinicion

      \startobservacion
        \startitemize[packed][before=,after=]
          \startitem
            Vea que un trinomio cuadrado perfecto es un trinomio de tipo cuadrático o cuadrático, mónico o no mónico.
          \stopitem

          \startitem
            La característica fundamental de un trinomio cuadrático o de tipo cuadrático es que sus variables aparecen elevadas a exponentes pares en el primer y tercer términos (aunque pueden ser en cualquiera dos de los tres términos) y que esas mismas variables aparecen en el término del medio elevadas a exponentees que son la mitad de aquéllos a los que aparecen elevadas en el primero y tercer términos (en general, en los otros dos términos).
          \stopitem
        \stopitemize
      \stopobservacion

      \startdiscusion{Factorización del trinomio cuadrático o de tipo cuadrático}
        \margindata[comentario]{Por un producto especial del álgebra elemental sabemos que si multiplicamos dos binomios cuyos términos correspondientes son semejantes el resultado tiene tres términos}

        \startformula
          \startalign
            \NC x^{2m} + b x^m y^n +c y^{2n} \NC = (x^m + d y^n)(x^m + e y^n) \NR
            \NC \NC = x^{2m} + e x^m y^n + d x^m y^n + d e y^n \NR
            \NC \NC = x^{2m} + (e+d) x^m y^n + d e y^n \NR
            \NC \NC = x^{2m} + (d+e) x^m y^n + d e y^n \NR
          \stopalign
        \stopformula

        Resumiendo:
        \startformula
          \startalign
            \NC c \NC = d e \NR
            \NC b \NC = d + e \NR
          \stopalign
        \stopformula

        \margindata[youtube]{\from[AI15A]}
        
        \startejemplos
          Factorice primamente
          \startitemejem
            \startitem
              $\ini{x^2 + 5x + 6} = (x + 3)(x + 2)$

              Vemos que no tiene ningún factor común. Tampoco se trata de un trinomio cuadrado perfecto. Se trata de un trinomio cuadrático, ya que la potencia más alta es 2, luego $n = 1$ y además la misma variable aparece en el segundo término elevada a 1.

              Recordemos que $de = 6$ y $d + e = 5$. Estudiamos el signo y vemos que ambos tienen que ser positivos. Vemos los factores del 6 hasta ver que ellos suman 5. Al final los número que buscamos son 3 y 2.
            \stopitem

            \startitem
              $\ini{5x^2y + x^4 + 4y^2} = x^4 + 5x^2y + 4y^2 = (x^2 + 4y)(x^2 + y)$

              En este caso lo primero que hacemos es ordenar el trinomio.
            \stopitem

            \margindata[youtube]{\from[AI15B]}
            \startitem
              $\ini{-40z^4 + 3x^4 y^3 z^2 + x^8 y^ 6} = x^8 y^6 + 3x^4 y^3 z^2 - 40z^4 = (x^4 y^3 + 8z^2)(x^4 y^3 - 5z^2)$

              Se trata de un trinomio de tipo cuadrático mónico.
            \stopitem

            \startitem
              $\ini{x^6 - 10x^3 y^5 - 24y^{10}} = (x^3 + 2y^5)(x^3 - 12y^5)$
            \stopitem
          \stopitemejem
        \stopejemplos

      \stopdiscusion


      \margindata[youtube]{\from[AI16A]}
      \startdiscusion[title={El trinomio cuadrático o de tipo cuadrático no mónico}]

        \placeformula[eq1]
        \startformula
          \startalign
            \NC a x^{2m} + b x^m y^n - c y^{2n} \NC = (px^m +qy^n)(r x^m + s y^n)\NR
            \NC \NC = prx^{2m} + psx^my^n +qry^nx^m + q s y^{2n} \NR
            \NC \NC = prx^{2m} + (ps + qr) x^my^n + q s y^{2n} \NR[+]
          \stopalign
        \stopformula

        \startformula
          \therefore pr = a,\; ps + qr = b,\; qs = c
        \stopformula

        Si ahora consideramos el producto
        \startformula
          P = ac = (pr)(qs) = prqs = psqr = (ps)(qr),
        \stopformula
        y la suma
        \startformula
          S = b = ps + qr,
        \stopformula
        y nos preguntamos, ¿qué dos números tienen como producto a: $ac = (ps)(qr)$; y como suma a: $b = ps+qr$?, la respuesta es, obviamente, $ps$ y $pr$.

        Vea que podemos reescribir a \ineq[eq1] como:

        \startformula
          pr x^{2m} + \underbracket{ps} x^m y^n + \underbracket{qr} x^m y^n + qs y^{2n},
        \stopformula

        y, además, observamos que ambos números están en los dos términos marcados.

        Si ahora aplicamos factorización por agrupación al multinomio anterior, obtenemos:

        \startformula
          \startalign
            \NC = \NC p x^m\left(r x^m + s y^n\right) + q y^n\left(r x^m + s y^n \right) \NR
            \NC = \NC \left(r x^m + s y ^n \right) \left(p x^m + q y^n \right) \NR
            \NC = \NC \left(p x^m + q y ^n \right) \left(r x^m + s y^n \right) \NR 
          \stopalign
        \stopformula

        Resumiendo: dado el trinomio cuadrático o de tipo cuadrático no mónico $ax^{2m} + bx^m y^n + c y^{2n}$, bucamos qué dos números multiplicados producen
        \startformula
          P = ac
        \stopformula
        y que al sumarlos totalizan
        \startformula
          S = b.
        \stopformula

        Vimos que eran los números que hemos identificado como $ps$ y $qr$.

        Reescribimos el trinomio cuadrático o de tipo cuadrático no mónico dado, $ax^{2m} + bx^m y^n + c y^{2n}$ en la forma:

        \startformula
          a x^{2m} + ps x^m y^n + qr x^m y^n + c y^{2n}
        \stopformula

        y procedemos a aplicar factorización por agrupación a este tetranomio obtenido.

        \margindata[youtube]{\from[AI16B]}
        \startejemplo
          Factorice completamente

          \startplaceformula
            \startejerformula
              \startalign
                \NC \ini{15x^4 - 13x^2y^3 - 6y^6} = \NC 15x^4 + 5x^2 y^3 - 18x^2 y^3 - 6y^6\NR[+]
                \NC = \NC 5x^2\left(3x^2 + y^3 \right) - 6y^3\left(3x^2 + y^3 \right)\NR
                \NC = \NC \left(3x^2 + y^3\right) \left(5x^2 - 6y^3\right)\NR
              \stopalign
            \stopejerformula
          \stopplaceformula

          Se trata de un trinomio de tipo cuadrático no mónico.
          Partimos de:
          \startformula
            \startalign
              \NC P = \NC 15(-6) = -90 = ac \NR
              \NC S = \NC -13 = b \NR
            \stopalign
          \stopformula
          y hallamos $a = 5$ y $c = -18$
        \stopejemplo
        
        \margindata[youtube]{\from[AI17A]} 
        \startejemplos

          \startplaceformula
            \startejerformula
              \startalign
                \NC \ini{12x^2 - 28x + 15} = \NC 12x^2 - 18x -10x + 15\NR[+]
                \NC = \NC 6x (2x - 3) - 5(2x - 3)\NR
                \NC = \NC (2x - 3) (6x - 5)\NR
              \stopalign
            \stopejerformula
          \stopplaceformula

          Se trata de un trinomio cuadrático no mónico.

          Partimos de:
          \startformula
            \startalign
              \NC P = \NC 12(15) = 180 = ac \NR
              \NC S = \NC -28 = b \NR
            \stopalign
          \stopformula
          y hallamos $a = -18$ y $c = -10$

          \startplaceformula
            \startejerformula
              \startalign
                \NC \ini{20x^2a^2 + 18xa^2y^4 - 126y^8a^2} = \NC\NR[+] 
              \stopalign
            \stopejerformula
          \stopplaceformula
          \startformula
            \startalign
              \NC = \NC 2a^2\left(10x^2 + 9xy^4 - 63y^8\right)\NR
              \NC = \NC 2a^2\left(10x^2 -21x y^4 + 50x y^4 -63y^8 \right)\NR
              \NC = \NC 2a^2\left[x\left(10x - 21y^4\right) + 3y^4\left(10 x - 21y^4\right)\right] \NR
              \NC = \NC 2a^2\left(10x - 21y^4\right) \left(x + 3y^4\right) \NR
            \stopalign
          \stopformula

          Después de sacar factor común aparece un trinomio de tipo cuadrático no mónico.

          Partimos de:
          \startformula
            \startalign
              \NC P = \NC 10(-63) = -630 = ac \NR
              \NC S = \NC 9 = b \NR
            \stopalign
          \stopformula
          y hallamos $a = 30$ y $c = -21$

          \margindata[youtube][method=top]{\from[AI17B]}
          \startplaceformula
            \startejerformula
              \startalign
                \NC \ini{3z + 20xyz + 12x^2 y^2 z} = \NC z\left(3 + 20xy + 12x^2 y^2\right) \NR[+]
                \NC = \NC z\left(3 + 2xy +18xy + 12x^2 y^2\right) \NR
                \NC = \NC z\left[(3 + 2xy) + 6xy (3 + 2xy)\right] \NR
                \NC = \NC z (3 + 2xy)(1+ 6xy) \NR
              \stopalign
            \stopejerformula
          \stopplaceformula

          Después de sacar factor común aparece un trinomio de tipo cuadrático no mónico.

          Partimos de:
          \startformula
            \startalign
              \NC P = \NC 3(12) = 36 = ac \NR
              \NC S = \NC  = 20 \NR
            \stopalign
          \stopformula
          y hallamos $a = 2$ y $c = 18$

          \startplaceformula
            \startejerformula
              \startalign
                \NC \ini{4x^2 + 12xy + 9y^2} = \NC (2x + 3y)^2 \NR[+]
              \stopalign
            \stopejerformula
          \stopplaceformula

          Se trata de un trinomio cuadrático no mónico. Sin embargo, comprobamos que también se trata de un trinomio cuadrado perfecto y lo factorizamos como tal.

          \startplaceformula
            \startejerformula
              \startalign
                \NC \ini{3p^2 + 22p + 8} \NC \NR[+]
              \stopalign
            \stopejerformula
          \stopplaceformula

          Se trata de un trinomio cuadrático no mónico.

          Partimos de:
          \startformula
            \startalign
              \NC P = \NC 3(8) = 24 = ac \NR
              \NC S = \NC b = 22 \NR
            \stopalign
          \stopformula
          pero no existen valores de $a$ y $c$ que cumplan las restricciones.

          Decimos, entonces, que el trinomio dado \obj{no es factorizable, es primo o es irreducible}.

        \stopejemplos
      \stopdiscusion
    \stopsubsection

  \stopsection

  \margindata[youtube]{\from[AI18A]}
  \startsection[title={Otras técnicas de factorización}]

    Se trata de variantes o combinaciones de las técnicas que hemos visto hasta ahora.

    \startejemplos
      \startplaceformula
        \startejerformula
          \startalign
            \NC \ini{16x^6 - 25y^2 + 9 z^4 - 24x^3z^2} = \NC 16x^6 - 24x^3z^2 + 9z^4 - 25y^2\NR[+]
            \NC = \NC \left(4x^3 - 3z^2\right)^2 - 25y^2 \NR
            \NC = \NC \left[\left(4x^3 - 3z^2\right) + 5y\right]\left[\left(4x^3 - 3z^2\right) - 5y\right] \NR
            \NC = \NC \left(4x^3 - 3x^2 + 5y\right) - \left(4x^3 - 3z^2 - 5y\right) \NR
          \stopalign
        \stopejerformula
      \stopplaceformula

      \startplaceformula
        \startejerformula
          \startalign
            \NC \ini{x^2 + 2xy + y^2 - u^2 + 2uv - v^2} = \NC \left(x^2 + 2xy + y^2\right) - \left(u^2 - 2uv + v^2\right) \NR[+]
            \NC = \NC (x + y)^2 - (u - v)^2 \NR
            \NC = \NC [(x + y) + (u -v)][(x + y) - (u - v)] \NR
            \NC = \NC (x + y + u - v)(x + y - u + v) \NR
          \stopalign
        \stopejerformula
      \stopplaceformula

      \startplaceformula
        \startejerformula
          \startalign
            \NC \ini{a^2 + ab - 2b^2 + 2a - 2b} = \NC (a + 2b)(a - b) + 2(a - b)\NR[+]
            \NC = \NC (a - b)(a + 2b +2)\NR
          \stopalign
        \stopejerformula
      \stopplaceformula

      Los primeros tres términos forman un trinomio cuadrático mónico. Para resolverlo partimos de:
      \startformula
        \startalign
          \NC P = \NC -2  \NR
          \NC S = \NC  1 \NR
        \stopalign
      \stopformula
      Obtenemos los valores 2 y -1.
    \stopejemplos

    \margindata[youtube]{\from[AI18B]}
  \stopsection

  \startsection[title={Factorización del tetranomio cúbido}]
    Consideremos el tetranomio de tercer grado $ax^3 + bx^2 + cx + d$. Se puede demostrar que si los coeficientes numéricos $a, b, c, d \in \integers$, el polinomio indicado puede factorizar en la forma

    \startplaceformula[eq2]
      \startformula
        (ex + f)(gx^2 + hx + j),
      \stopformula
    \stopplaceformula

    donde $e,f,g,h,j \in \integers$ fijos, también. A su vez el trinomio cuadrático $gx^2 + bx + j$ podrá no ser factorizable como el producto de otros dos binomios lineales con términos correspondientes semejantes.

    Si efectuamos la multiplicación larga indicada en \ineq[eq2] obtenemos

    \starttabulate[|ml|mc|mc|mc|mc|mc|mc|]
      \NC gx^2  \NC + \NC hx           \NC + \NC j          \NC   \NC    \NR
      \NC       \NC   \NC ex           \NC + \NC f          \NC   \NC    \NR
      \HL 
      \NC egx^3 \NC + \NC ehx^2        \NC + \NC ejx        \NC   \NC    \NR
      \NC       \NC   \NC fgx^2        \NC + \NC fhx        \NC + \NC fj \NR
      \HL
      \NC egx^3 \NC + \NC (eh + jg)x^2 \NC + \NC (ej + fh)x \NC + \NC fj \NR
    \stoptabulate

    de donde al compararlo con el polinomio dado inicialmente, vemos que:
    \startformula
      eg = a,\; eh + fg = b,\; ej + fh = c\; \text{ y }\; fj = d.
    \stopformula
    
    Si simplifiamos el resultado de la multiplicación anterior, obtenemos $egx^3 + ehx^2 + fgx^2 + ejx + fhx + fj$, que puede ser reescrito en la siguiente forma
    \startformula
      egx^3 + fgx^2 + ehx^2 + fhx + ejx + fj.
    \stopformula
    al aplicar la ley conmutativa de la suma. Si lo factorizamos, agrupando cada dos términos, obtenemos
    \startformula
      \startalign
        \NC = \NC gx^2 (ex + f) + hx(ex + f) + j(ex + f) \NR
        \NC = \NC (ex + f) (gx^2 + hx + j), \NR
      \stopalign
    \stopformula
    obteniendo la factorización del trinomio original ya vista en \ineq[eq2].

    Observamos, entonces, que para obtener esa factorización del tretanomio original de tercer grado, $ax^3 + bx^2 + cx + d$, lo tenemos que reescribir estirando los dos términos $bx^2$ y $cx$, del siguiente modo:
    \startformula
      ax^3 + fgx^2 + ehx^2 + fhx + ejx + d,
    \stopformula
    donde los cuatro términos nuevos que hemos escrito ($e,f,g,h$) deben ser tales que haya factores comunes cada dos términos que agrupemos entonces.

    \comentario{Factorización del tetranomio cúbico \par
    $ax^3 + bx^2 + cx + d = (ex +f)(gx^2 + hx + j)$ donde $a =eg$, $b = eh + fg$, $c = ej + fh$ y $d = fj$.}
    \startejemplo
      \ini{Factorice completamente $2x^3 + 6 - 7x - x^2$}

      Primero reescribimos en el orden que nos da la teoría
      \startformula
        \startalign
          \NC = \NC 2x^3 - x^2 - 7x + 6 \NR
          \NC = \NC 2x^3 - 2x^2 + x^2 - x - 6x + 6 \NR
          \NC = \NC 2x^2(x - 1) + x(x - 1) - 6(x - 1) \NR
          \NC = \NC (x - 1) (2x^2 + x - 6) \NR
        \stopalign
      \stopformula
      Vemos que el segundo factor es un trinomio cuadrático no mónico.

      Partimos de
      \startformula
        \startalign
          \NC P = 2(-6) \NC -12  \NR
          \NC S = \NC  1 \NR
        \stopalign
      \stopformula
      Obtenemos los valores -3 y 4 y operamos,
      
      \startformula
        \startalign
          \NC = \NC (x - 1) (2x^2 -3x + 4x -6) \NR
          \NC = \NC (x - 1) [x(2x - 3) + 2(2x - 3)] \NR
          \NC = \NC (x - 1) (2x - 3) (x + 2) \NR
        \stopalign
      \stopformula
    \stopejemplo

  \stopsection

  \startsection[title={Factorización de otros binomios}]

    \startteorema
      Si $x,y \in \reals$, entonces:
      \startitemizer
        \startitem
          si $n \in \naturalnumbers$ par, $(x + y) \mid (x^n - y^n)$
        \stopitem
        \startitem
          si $n \in \naturalnumbers$ par, $(x + y) \nmid (x^n + y^n) \land (x - y) \nmid (x^n - y^n)$ 
        \stopitem
        \startitem
          si $n \in \naturalnumbers$, entonces $(x - y) \mid (x^n - y^n)$
        \stopitem
        \startitem
          si $n \in \naturalnumbers$ impar, $(x + y) \mid (x^n + y^n)$
        \stopitem
      \stopitemizer
    \stopteorema

    \startejemplos
      Factorice primamente.

      \startitemejem
        \startitem
          \ini{x^5 + y^5}

          Por la parte $iv)$ del teorema anterior,
          \startformula
            (x + y) \mid (x + y)
          \stopformula
        \stopitem

        Efectuamos la división larga de $x^5 + y^5 \div (x + y)$

        
        \stopitemejem
          

    \stopejemplos
    
  \stopsection

\stopchapter

\stopcomponent