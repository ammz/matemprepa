\startcomponent c_conjuntos

\project project_matemprepa

\margindata[youtube][method=top]{\from[AE10A]}
\startchapter[title={Conjuntos}]

  \startsection[title={Relación de igualdad}]
    Las cosas que se colocan a cada lado del signo de igualdad tienen
    que representar una y la misma cosa.

    \startaxioma{Propiedades fundamentales de la igualdad}
      Sean $x, y, z$ tres objetos o cosas. Entonces:
      \startitemizer
        \startitem
          \obj{Ley reflexiva}: $x = x$
        \stopitem
        \startitem
          \obj{Ley simétrica}: $x = y --> y = x$
        \stopitem
        \startitem
          \obj{Ley transitiva}: $x = y \wedge y = z --> x = z$
        \stopitem
        \startitem
          \obj{Ley de sustitución}: Si $x = y$, entonces en cualquier
          enunciado en donde aparezca $x$ podemos escribir $y$, y
          viceversa, sin alterar el valor de verdad de dicho
          enunciado.
        \stopitem
      \stopitemizer
    \stopaxioma

    \startobservacion

      Si aceptamos como cierta únicamente la ley de sustitución, se
      pueden demostrar las otras tres leyes a partir de ésta.
    \stopobservacion

    \startteorema[teo:1]
      Sean $x, y, z$ tres objetos o cosas. Si $x=y$, entonces
      \startitemizer
        \startitem
          $x = x$
        \stopitem
        \startitem
          $y = x$
        \stopitem
        \startitem
          si $y=z$, entonces $x=z$
        \stopitem
      \stopitemize
    \stopteorema

    \startdemop
      \startitemizep
        \startitem
          Por hipótesis, $x=y$. Luego, por la ley de sustitución,
          $x=x$.
        \stopitem
        \startitem
          Por hipótesis, $x=y$. Entonces, $y=x$, por la ley de
          sustitución.
        \stopitem
        \startitem
          Por hipótesis, $x=y, y=x$. En consecuencia, debido a la ley
          de sustitución, $x=z$
        \stopitem
      \stopitemizep
    \stopdemop

  \stopsection

  \startsection[title={Definición de conjunto}]
    Partimos del concepto de \obj{colección} que se entiende como un
    término indefinido o primitivo.

    \startdefinicion
      \startitemizer
        \startitem
          \obj{Una colección bien definida} es aquella en donde
          sabemos seguro qué cosas están y qué cosas no están en ella.
        \stopitem

        \startitem
          Cualquier cosa bien definida es \obj{un conjunto}.
        \stopitem

        \startitem
          Los componentes de un conjunto se llaman \obj{elementos o
            miembros de éste}.
        \stopitem
      \stopitemizer
    \stopdefinicion

    \margindata[youtube]{\from[AE10B]}
    \startejemplos
      \startitemize[n]
        \startitem
          \ini{La colección de las letras del idioma español} está
          bien definida. Por lo tanto, esa colección es un
          conjunto. Sus elementos o miembros son: a, e, i, o, u.
        \stopitem
        \startitem
          \ini{La colección de cinco libros escritos en español} no
          está bien definida. luega, esta colección no es un conjunto.
        \stopitem
      \stopitemize
    \stopejemplos

    \startdiscusion{Nombres de conjuntos}
      A los conjuntos los representamos con letra mayúsculas.

      \startejemplos
        \startitemize[packed]
          \startitem
            $V:\;$ representa el conjunto de las letras vocales del
            idioma español.
          \stopitem
          \startitem
            $x:\;$ representante o miembro de un conjunto.
          \stopitem
        \stopitemize
      \stopejemplos

    \stopdiscusion

    Indicamos que una cosa es miembro de un conjunto por medio del
    símbolo $\in$.
    \startformula
      \text{elemento} \in \text{conjunto}
    \stopformula
    Si no pertenece lo indicamos con el signo $\nin$.
    \startformula
      \text{elemento} \nin \text{conjunto}
    \stopformula

    \startdiscusion{Representación de conjuntos}
      \startitemize[n]
        \startitem
          \obj{Descripción verbal}: A es el conjunto de las letras del
          alfabeto español.
        \stopitem
        \startitem
          \obj{Tabulación o enumeración}:
          $\{a, e, i, o, u\} = V\quad;\quad A = \{a,b,c, \dots,z\}$
        \stopitem
        \startitem
          \obj{Forma estándar o constructiva}: $\{x \mid p\}$ siendo
          $p$ un enunciado que da una propiedad de los elementos del
          conjunto.

          $V = \{x \mid x\; \text{es una letra vocal del idioma
            español}\}$
        \stopitem
        \startitem
          \obj{Diagramas de Venn}

          \startcenteraligned
            \externalfigure[diagramasVenn.png][maxwidth=0.5\textwidth]
          \stopcenteraligned

        \stopitem
      \stopitemize

      \margindata[youtube]{\from[AE11A]}
      \startejemplo
        Sea $L = \{a, 2, \{a, 5\}, q\}$ entonces tendremos:

        $a \in L,\; 2 \in L,\; \{a, 5\} \in L,\; q \in L,\; 5 \nin L$
      \stopejemplo
    \stopdiscusion

    \startdefinicion
      \obj{Un suscrito} es un símbolo que se escribe en la parte
      inferior derecha de otro símbolo. Su uso es para ordenar o para
      identificar.
    \stopdefinicion

    \startejemplos
      \startitemize[n]
        \startitem
          Los cuatro casos que representan los posibles valores de
          verdad que podrían ocurrir para dos enunciados son
          $C_1, C_2, C_3 \text{ y } C_4$. En este caso sirve para
          ordenar.
        \stopitem
        \startitem
          Si $A = \{a, 3, -8, -d\}$ tenemos que
          $a_1 = a;\, a_2 = 3;\, a_3 = -8;\, a_4 = -d$ (identificar)
        \stopitem
        \startitem
          Si $A = \{3, -d, a, -8\}$ tenemos que
          $a_1 = 3;\, a_2 = -d;\, a_3 = a;\, a_4 = -8$ (identificar)
        \stopitem
      \stopitemize
    \stopejemplos

  \stopsection

  \margindata[youtube][method=top]{\from[AE11B]}
  \startsection[title={Un conjunto muy especial: el conjunto de los
      números naturales o de contar, $\naturalnumbers$}]

    \startformula
      \naturalnumbers = \{1, 2, 3, \dots\}
    \stopformula

    \startdefinicion
      \startitemizer
        \startitem
          El símbolo usado para representar una cantidad se llama
          \obj{un numeral}.
        \stopitem
        \startitem
          La cantidad que representa un numeral se llama \obj{un
            número}.
        \stopitem
        \startitem
          El proceso de parear objetos, cosas o personas con los
          numerales que representan su cantidad, se conoce como
          \obj{conteo}.
        \stopitem
        \startitem
          El conjunto de signos o \obj{dígitos} y la manera de
          construir los numerales con ellos se conoce como \obj{un
            sistema de numeración}.
        \stopitem
      \stopitemizer
    \stopdefinicion

  \stopsection

  \startsection[title={Características de los conjuntos}]

    \startdefinicion
      Sea $A$ un conjunto. La \obj{cardinalidad de $A$} denotada por
      cualquiera de los símbolos $\#A \equiv c(A) \equiv n(A)$, se
      refiere al número de elementos que contiene $A$.
    \stopdefinicion

    \startdefinicion
      \margindata[youtube]{\from[AE12A]} \obj{El conjunto nulo o
        vacío} es aquel que no contiene elementos. Se representa por
      cualquiera de los signos: $\{\,\}$ o $\emptyset$.
    \stopdefinicion

    \startejemplos
      \startitemize[packed,n]
        \startitem
          El conjunto de las personas que han durado más de mil años.
        \stopitem
        \startitem
          El conjunto de las palabras en español que contienen cinco o
          más letras consonantes juntas.
        \stopitem
      \stopitemize
    \stopejemplos

    \startdefinicion
      \startitemizer
        \startitem
          Dos conjuntos, $A$ y $B$, se llaman \obj{iguales}, denotado
          por el símbolo $A = B$, ssi
          $\forall\, x \in A;\, x \in B \wedge \forall\, x \in B, x \in
          A$. Es decir, ssi $A$ y $B$ tienen los mismos elementos. Si
          $A$ no es igual a $B$, escribimos $A /= B$
        \stopitem
        \startitem
          Dos conjuntos $A$ y $B$ se llaman \obj{equivalentes}, denotado
          por el símbolo $A <-> B$, ssi $\#A = \#B$. Si $A$ y $B$ no son
          equivalentes los representamos por el símbolo
          $A \nleftrightarrow B$.
        \stopitem
      \stopitemizer
    \stopdefinicion

    \startobservacion
      \startitemize[packed,joinedup]
        \startitem
          Al representar un conjunto por tabulación o enumeración no hay
          que repetir los elementos.
        \stopitem
        \startitem
          Dos conjuntos pueden ser iguales aunque tengan sus elementos
          enumerados en diferente orden.
        \stopitem
        \startitem
          Dos conjuntos iguales son, automáticamente, equivalentes.
        \stopitem
      \stopitemize
    \stopobservacion

    \startejemplos
      \startitemejem

        \startitem
          \ini{Sea $V = \{a, e, i, o, u\}$ y
            $W = \{i, u, e, u, a, o, i\}$.}
          \startitemize[packed]
            \startitem
              Para ver si estos dos conjuntos son iguales comprobamos
              que todo elemento del primero tienen que estar en el
              segundo y viceversa. Y, por lo tanto, $V = W$.

              En $W\,$ algunos elementos están repetidos. Podemos
              reescribir este conjunto eliminando los elementos
              repetidos. Tampoco importa el orden en el que estén
              presentados los elementos. Así que ahora tendremos,
              $W = \{i, u, e, a, o\}$.
            \stopitem
            \startitem
              Para ver si estos dos conjuntos son equivalentes
              comprobamos su cardinalidad: $\#V = 5 =\#W$.
            \stopitem
          \stopitemize
        \stopitem

        \margindata[youtube]{\from[AE12B]}
        \startitem
          \ini{Sea $A =\{p, q, \{r, s\}\}$ y $B =\{r, q, s, p\}$.}
          \startitemize[packed]
            \startitem
              Estos dos conjuntos no son iguales. $A \neq B$
            \stopitem
            \startitem
              Como $\#A = 3$ y $\#B = 4$, entonces no son
              equivalentes. $A \nleftrightarrow B$.
            \stopitem
          \stopitemize
        \stopitem

        \startitem
          \ini{$P = \{?, 3, q\}$ y $Q$}
          % \inoutermargin[voffset=1cm]{\externalfigure[Q_Venn.png][maxwidth=2.6cm]}}
          \placefigure [none] {} {\externalfigure[Q_Venn.png][maxwidth=2.6cm]}

          \startitemize[packed]
            \startitem
              Estos dos conjuntos son iguales. $P = Q$. Aunque están
              representados de formas diferentes.
            \stopitem
            \startitem
              Al ser iguales, automáticamente, son equivalentes.
              $A \leftrightarrow B$.
            \stopitem
          \stopitemize
        \stopitem
          
        \startitem
          \ini{Sea $M =\{1, 2, 3\}$ y $N =\{1, 2, c\}$.}
          \startitemize[packed]
            \startitem
              Estos dos conjuntos no son iguales. $M \neq N$
            \stopitem
            \startitem
              Como $\#M = 3 = \#N$, entonces son equivalentes.
              $M \leftrightarrow N$.
            \stopitem
          \stopitemize
        \stopitem

      \stopitemejem
    \stopejemplos

    \startdefinicion
      \startitemizer
        \startitem
          Un conjunto se llama \obj{finito} si podemos terminar de
          contar sus elementos.
        \stopitem
        \startitem
          Un conjunto se llama \obj{infinito} si el proceso de contar
          sus elementos no termina. La cardinalidad de un conjunto se
          llama \obj{infinita} y se representa por el signo $\infty$.
        \stopitem
      \stopitemizer
    \stopdefinicion

    \startejemplos
      \startitemejem
        \startitem
          El conjunto $V = \{a, e, i, o, u\}$ tiene cardinalidad
          $\#V = 5$, luego es un conjunto finito. También lo son los
          conjuntos $A, B, P, Q$ vistos anterioremente.
        \stopitem
        \startitem
          $\naturalnumbers = \{1, 2, 3, \dots \,\}$ es
          infinito. $\#\naturalnumbers = \infty$.
        \stopitem
      \stopitemejem
    \stopejemplos

    \startdefinicion
      Consideremos dos conjuntos $A$ y $B$. Decimos que \obj{$A$ es
        subconjunto de $B$}, representado con el símbolo
      $A \subseteq B$, ssi $\forall\, x \in A, x \in B$. Si $A$ no es
      subconjunto de $B$ lo indicamos con el símbolo $A \nsubseteq B$.
    \stopdefinicion

    \margindata[youtube]{\from[AE13A]}
    \startobservacion
      $X \subseteq A \wedge Y \subseteq A$ se abrevia $X,Y \subseteq A$.
    \stopobservacion

    \startejemplos
      \startitemejem
        \startitem
          \ini{$A = \{1,2,3\}$ y $B = \{1\}$} Tenemos que
          $A \nsubseteq B$ y $B \subseteq A$.
        \stopitem
        \startitem
          \ini{$C = \{x \mid x$ es un número natural desde el uno hasta
            el 15\} y D = \{1, 2, 3, $\dots$, 100\}}. Tenemos que
          $C \subseteq D$ y $D \nsubseteq C$.
        \stopitem
        \startitem
          \ini{$P = \{a, b, c\};\, Q =\{c, a, b\}$} Tenemos que
          $P \subseteq Q$ y $Q \subseteq P$. También observamos que
          ambos son iguales.
        \stopitem
        \startitem
          \ini{$M = \{1, 2\};\, N =\{2, 3\}$} Tenemos que
          $M \nsubseteq N$ y $M \nsubseteq N$.
        \stopitem
      \stopitemejem
    \stopejemplos

    \startobservacion
      $A \nsubseteq B$ ssi $\exists x \in A$ tal que $x \nin B$
    \stopobservacion

    \startlema
      Las afirmaciones $p$ y $p \wedge p$ son equivalentes.
    \stoplema

    \startdemo{}\\
      \startcenteraligned
        \starttable[|cm|c|c|c|]
          \NC \VL $p$ \VL $p$ \VL $p \wedge p$ \NC \AR
          \HL
          \NC c_1 \VL C \VL C \VL C \NC \AR
          \NC c_2 \VL F \VL F \VL F \NC \AR
        \stoptable
      \stopcenteraligned
    \stopdemo

    \startteorema[teo:2]
      \margindata[youtube]{\from[AE13B]} Sea $A$, un conjunto
      cualquiera, y $\emptyset$ un conjunto nulo. Entonces
      \startitemizer
        \startitem
          $A \subseteq A$
        \stopitem
        \startitem
          $\emptyset \subseteq A$
        \stopitem
      \stopitemizer
    \stopteorema

    \startdemop
      \startitemizep
        \startitem
          Por la ley reflexiva de la igualdad, $A = A$. Luego, según la
          definición de conjuntos iguales,
          $\forall\, x \in A, x \in A \wedge \forall\, x \in A, x \in
          A$. Entonces, por el lema que acabamos de demostrar, lo
          anterior se reduce a decir que $\forall\, x \in A, x \in
          A$. En consecuencia, por la definición de subconjunto,
          $A \subseteq A$.
        \stopitem
        \startitem
          (por contradicción) Suponemos que $\emptyset \nsubseteq
          A$. Entonces $\exists\, x \in \emptyset$ tal que $x \nin
          A$. Pero vemos que es una contradicción
          $(\rightarrow\leftarrow)$ a la definición de conjunto nulo.
        \stopitem
      \stopitemizep
    \stopdemop

    \startejemplos{Escriba los subconjuntos de:}
      \startitemejem
        \startitem
          \ini{$Q = \{3, a\}$}

          $\{3, a\},\; \emptyset,\;\{3\},\; \{a\}$
        \stopitem
        \startitem
          \margindata[youtube]{\from[AE14A]}
          \ini{$R =\{p, \{q\}, 3\}$}

          $\{p, \{q\}, 3\},\; \emptyset,\; \{p\},\; \{\{q\}\},\;
          \{3\},\; \{p, \{q\}\},\; \{p, 3\},\; \{\{q\},\; 3\}$
        \stopitem
      \stopitemejem
    \stopejemplos

    \startdefinicion
      Sean $A$ y $B$ dos conjuntos con $A \subseteq B$. Decimos que
      \obj{$A$ es un subconjunto propio de $B$} o que \obj{$A$ está
        contenido propiamente en $B$}, denotado por el símbolo
      $A \subset B$, ssi $A /= B \wedge A /= \emptyset$. Si esto no
      ocurre, lo indicamos con símbolo $A \nsubset B$.
    \stopdefinicion

    \startejemplo
      Los subconjuntos propios de $R$ son $\{p\}$, $\{\{q\}\}$, $\{3\}$,
      $\{p, \{q\}\}$, $\{p, 3\}$, $\{\{q\}$, $3\}$. Se han excluido el
      original, $\{p, \{q\}, 3\}$ y el nulo, $\emptyset$.
    \stopejemplo

    \startteorema[teo:3]
      Sean $A$ y $B$ dos conjuntos. $A \subseteq B \wedge B \subseteq A$
      si y sólo si $A = B$.
    \stopteorema

    \startdemo
      Por ser una bicondicional hay que demostrarlo en las dos
      direcciones.

      $\Longrightarrow\;$ Suponemos que
      $A \subseteq B \wedge B \subseteq A$ (hipótesis). Pero
      $A \subseteq B$ significa, por la definición de subconjunto, que
      $\forall\, x \in A, x \in B$; mientras que $B \subseteq A$
      significa que $\forall\, x \in B, x \in A$. Pero, lo anterior
      implica que, $A = B$, por la definición de igualdad de conjuntos.

      \margindata[youtube]{\from[AE14B]}

      $\Longleftarrow\;$ Suponemos que $A = B$. Esto significa, por la
      definición de conjuntos iguales, que
      $\underbrace{\forall\, x \in A, x \in B}_{(1)} \wedge
      \underbrace{\forall\, x \in B, x \in A}_{(2)}$.

      Pero (1) significa que $A \subseteq B$ y (2) significa que
      $B \subseteq A$, ambos por la definición de subconjunto.
    \stopdemo

    \startteorema[teo:4]
      El conjunto nulo es único.
    \stopteorema

    \startdemo
      Supongamos que $\emptyset$ y $\emptyset'$ son dos conjuntos
      diferentes. Entonces, por el \in{teorema}[teo:2], que nos dice que
      $\emptyset \subseteq A$, tenemos que
      $\emptyset \subseteq \emptyset'$ y por la misma razón
      $\emptyset' \subseteq \emptyset$. Entonces, por el
      \in{teorema}[teo:3], $\emptyset = \emptyset'$.
    \stopdemo

    \startteorema
      Sean $A, B$ y $C$ tres conjuntos con $A \subseteq B$ y
      $B \subseteq C$. Entonces, $A \subseteq C$.
    \stopteorema

    \startdemo
      Como $A \subseteq B$, entonces, por la definición de subconjunto,
      tenemos que $\forall {x \in A}, {x \in B}$. Y como
      $B \subseteq C$, entonces resulta que
      $\forall\, x \in B, x \in C$. Resumiendo,
      $\forall\, x \in A, x \in C$. Pero esto significa que
      $A \subseteq C$.
    \stopdemo

    \startdefinicion
      \margindata[youtube]{\from[AE15A]} Un conjunto en el cual esté
      contenido otro conjunto se llama \obj{un conjunto universo o
        universal de éste}. Damos el nombre $U$ a cualquier conjunto
      universo.
    \stopdefinicion

    \startobservacion
      Un conjunto universal se representa, usualmente, como un
      rectángulo al representarlo por diagramas de Venn.
    \stopobservacion

    \startejemplo
      Si $I = \{1, 2, 3\}$, dos posibles universos de ese conjunto
      serían
      \startitemize[n,packed]
        \startitem
          $U_1 = I = \{1, 2, 3\}$, ya que todo conjunto está contenido
          en él mismo, $A \subseteq A$.
        \stopitem
        \startitem
          $U_2 = \{1, 2, 3, a, b\}$
        \stopitem
      \stopitemize
    \stopejemplo

  \stopsection

  \startsection[title={Operaciones entre conjuntos}]

    Las \obj{operaciones} son ejecutorias o actividades que se pueden
    efectuar con dos objetos o cosas para obtener una tercera cosa. A
    las dos primeras cosas se les llama los \obj{operandos de la
      operación} y la tercera, se le llama el \obj{resultado de la
      operación}.

    \startdefinicion{}
      Sean $A$ y $B$ dos conjuntos. Se definen las siguientes
      operaciones\footnote{Se presenta un repaso detallado de estas
        operaciones al comienzo del video \from[AE16A]}

      \startitemizer[unpacked]
        \startitem
          \obj{La unión de $A$ con $B$}, denotada con el símbolo
          $A \cup B$, es el conjunto
          $\{x \mid x \in A \lor x \in B\}$.
        \stopitem
        \startitem
          \obj{La intersección de $A$ con $B$}, denotada con el
          símbolo $A \cap B$, es el conjunto
          $\{x \mid x \in A \land x \in B\}$.
        \stopitem
        \startitem
          \obj{La diferencia de $A$ con $B$}, representada con
          cualquiera de los símbolos $A \setminus B \equiv A - B$, y
          que se lee \quote{A menos B}, es el conjunto
          $\{x \mid {x \in A} \land {x \nin B}\}$.
        \stopitem
        \startitem
          \obj{La diferencia simétrica de $A$ con $B$}, denotada con
          el símbolo $A \triangle B$, es el conjunto
          $\{x \mid (x \in A \land x \nin B) \lor (x \in B \land x
          \nin A)\} = (A \setminus B) \cup (B \setminus A)$.
        \stopitem
        \startitem
          Si $U$ es un conjunto universo de $A$, el \obj{complemento
            de $A$}, denotado por cualquiera de los símbolos
          $A' \equiv A^c \equiv \bar{A}$, es el conjunto
          $U \setminus A = \{x \mid x \in U \land x \nin A\}$.
        \stopitem
      \stopitemizer
    \stopdefinicion

    \startejemplos
      \margindata[youtube]{\from[AE15B]}

      Considere los conjuntos
      $A = \{1,2,3,4,5\}, B = \{3,4,5,a,b\}, C = \{5,a,b\}$ y
      $D =\{3,4,c,d,9\}$ y suponga que
      $U = \{1,2,3,\dots,10,a,b,c,\dots,h\}$.

      \startitemejem[columns]
        \startitem
          $\ini{B \cup A} = \{3,4,5,a,b,1,2\}$
        \stopitem
        \startitem
          $\ini{C \cup B} = \{5,a,b,3,4\}$
        \stopitem
        \startitem
          $\ini{A \cap B} = \{3,4,5\}$
        \stopitem
        \startitem
          $\ini{C \cap D} = \{\} \equiv\emptyset$
        \stopitem
        \startitem
          $\ini{B \cap C} = \{5,a,b\}$
        \stopitem
        \startitem
          $\ini{A \triangle D} = \{1,2,5\}$
        \stopitem
      \stopitemejem
    \stopejemplos

    \startejemplos
      Halle
      \startitemejem
        \startitem
          $\ini{C - B} = \{ \} \equiv \emptyset$
        \stopitem
        \startitem
          $\ini{D \setminus B} = \{c,d,9\}$
        \stopitem
        \startitem
          $\ini{A \triangle D} = (A \setminus D) \cup (D \setminus A)
          = \{1,2,5\} \cup \{c,d,9\} = \{1,2,5,c,d,9\}$
        \stopitem
        \startitem
          $\ini{B \triangle C} = (B \setminus C) \cup (C \setminus B)
          = \{3,4\} \cup \{\} = \{3,4\}$
        \stopitem
        \startitem
          $\ini{D'} = U - D = \{1,2,5,6,7,8,10,a,b,e,f,g,h\}$
        \stopitem
        \margindata[youtube]{\from[AE16B]}
        \startitem
          $\ini{B'} = U \setminus B = \{1,2,6,7,8,9,10,c,d,e,f,g,h\}$
        \stopitem
      \stopitemejem
    \stopejemplos

%    \comentario{En expresiones con más de una operación, los signos de agrupación nos dirán lo primero que efectuar.}
    % \startejemplos
    %   \startitemejem
    %     \startitem
    %       $\ini{(C \cap D) \cup A} = \{ \} \cup \{1,2,3,4,5\} = \{1,2,3,4,5\}$
    %     \stopitem
    %     \startitem
    %       $\ini{B \cap (C \cup A)} = \{3,4,5,a,b\} \cap \{3,4,c,d,9,1,2,5\} = \{3,4,5\}$
    %     \stopitem
    %     \startitem
    %       $\ini{A - (C \cap D)} = \{1,2,3,4,5\} \setminus \{5,a,b\} = \{1,2,3,4\}$
    %     \stopitem
    %     \startitem
    %       $\ini{(D \triangle B) \setminus C} = \{c,d,9,5,a,b\} \setminus \{5,a,b\} = \{c,d,9\}$
    %     \stopitem
    %     \startitem
    %       $\ini{(A \cup B)'} = \{1,2,3,4,5,a,b\}' = \{6,7,8,9,10,c,d,e,f,g,h\}$
    %     \stopitem
    %     \margindata[youtube]{\from[AE17A]}
    %     \startitem
    %       $\ini{[(A \triangle C) \cap (B \cup D)]^c} = [\{1,2,3,4,a,b\} \cap \{3,4,5,a,b,c,d,9\}]^c$ \par$ = [\{3,4,a,b\}]^c = \{1,2,5,6,7,8,9,10,c,d,e,f,g,h\}$
    %     \stopitem
    %   \stopitemejem
    % \stopejemplos

    \startdefinicion
      Dos conjuntos $A$ y $B$ se llaman \obj{conjuntos disjuntos} ssi
      $A \cap B = \{\} \equiv \emptyset$.
    \stopdefinicion

    \startejemplos
      $C \cap D = \emptyset$. Luego, $C$ y $D$ son disjuntos;
      $C \cap B = \{5,a,b\}$. Luego, $C$ y $B$ no son
      disjuntos. Aunque $C \setminus B = \emptyset$, esto no debe
      confundirnos.
    \stopejemplos
  \stopsection

  % \comentario[align=]{
  %   Recordemos que:
  %   \startitemize[packed,none,joinedup]
  %     \startitem
  %       $A \subseteq B = B \subseteq A <-> A = B$
  %     \stopitem
  %     \startitem
  %       $A \subseteq B <-> \forall x \in A, x \in B$
  %     \stopitem
  %     \startitem
  %       $A \setminus B = A - B = \{ x \mid x \in A \land x \nin B\}$
  %     \stopitem
  %     \startitem
  %       $A \triangle B = (A \setminus B) \cup (B \setminus A)$
  %     \stopitem
  %     \startitem
  %       $A' \equiv A^c \equiv \bar{A} = U \setminus A$
  %     \stopitem
  %   \stopitemize
  %
  %   \blank[big]
  %   Enunciados equivalentes:
  %   \startitemize[packed,none,joinedup]
  %     \startitem
  %       $p \land q\quad$ y $\quad q \land p$
  %     \stopitem
  %     \startitem
  %       $(p \lor q) \lor r\quad$ y $\quad p \lor (q \lor r)$
  %     \stopitem
  %     \startitem
  %       $\neg(p \lor q)\quad$ y $\quad\neg p \land(\neg q)$
  %     \stopitem
  %     \startitem
  %       $\neg(p \land q)\quad$ y $\quad\neg p \lor(\neg q)$
  %     \stopitem
  %   \stopitemize
  % }

  \startsection[title={Demostraciones con operaciones con conjuntos}]
    \margindata[youtube]{\from[AE17B]}

    \startejemplos
      \startitemejem
        \startitem
          \ini{$\ini{A \cap B = B \cap A}$}

          Bastará ver que $A \cap B \subseteq B \cap A$ y que
          $B \cap A \subseteq A \cap B$.

          Para lo primero, tomamos $x \in A \cap B$. Entonces,
          $x \in A \land x \in B$. Pero esto es equivalente a decir
          que $x \in B \land x \in A$, lo que significa que
          $x \in B \cap A$. Luego, $A \cap B \subseteq B \cap A$.

          Para lo segundo, tomamos $x \in B \cap A$. Entonces,
          $x \in B \land x \in A$. Pero esto es equivalente a decir
          que $x \in A \land x \in B$, lo que significa que
          $x \in A \cap B$. Luego, $B \cap A \subseteq A \cap B$.
        \stopitem
        \startitem
          \ini{$(A \cup B) \cup C = A \cup (B \cup C)$}

          Bastará ver que
          $(A \cup B) \cup C \subseteq A \cup (B \cup C)$ y que
          $A \cup (B \cup C) \subseteq (A \cup B) \cup C$.

          Para lo primero, tomamos $x \in (A \cup B) \cup C$. Esto
          significa que $x \in (A \cup B) \lor x \in C$. Pero
          significa que $(x \in A \lor x \in B) \lor x \in C$, y esto
          es equivalente a decir,
          $x \in A \lor (x \in B \lor x \in C)$, lo que equivale a
          decir que $x \in A \lor x \in (B \cup C)$, lo que implica
          que $x \in A \cup (B \cup C)$.

          \margindata[youtube]{\from[AE18A]}

          Para lo segundo, tomamos $x \in A \cup (B \cup C)$. Esto
          significa que $x \in A \lor x \in (B \cup C)$. Pero esto
          significa que $x \in A \lor (x \in B \lor x \in C)$ y ello
          es equivalente a decir,
          $(x \in A \lor x \in B) \lor x \in C$, lo que equivale a
          decir que $x \in (A \cup B) \lor x \in C$, lo que implica
          que $x \in (A \cup B) \cup C$.
        \stopitem
        \startitem
          \ini{$x \nin A \cup B -> x \nin A \land x \nin B$}

          Por hipótesis, $x \nin A \cup B$. De aquí, tiene que ocurrir
          que $\neg(x \in A \lor x \in B)$. Esto equivale a decir
          $\neg(x \in A) \land [\neg(x \in B)]$. Esto es lo mismo que
          decir que $x \nin A \land x \nin B$.
        \stopitem
        \margindata[youtube]{\from[AE18B]}
        \startitem
          \ini{$x \nin A \setminus B -> x \nin A \lor x \in B$}

          Por hipótesis, $x \nin A \setminus B$. Entonces ocurre que
          $\neg(x \in A \land x \nin B)$. Esto equivale a decir,
          $\neg(x \in A) \lor [\neg(x \nin B)]$. En consecuencia,
          $x \nin A \lor x \in B$.
        \stopitem
        \startitem
          \ini{$A \setminus B = A \cap B' = B' \setminus A'$}

          $A \setminus B = \{x \mid x \in A \land x \nin B\} = \{x
          \mid x \in A \land x \in B'\} = A \cap B' = \{x \mid {x \in
            A} \land {x \in B'}\} = \{x \mid x \nin A' \land x \in
          B'\} = \{x \mid x \in B' \land x \nin A'\} = B' \setminus
          A'$.
        \stopitem
      \stopitemejem
    \stopejemplos

  \stopsection

  \startsection[title={Diagramas de Venn}]

    \startejemplos
      Represente con diagramas de Venn

      \blank
      \setupTABLE[frame=off]
      \setupTABLE[c][1][width=1.2em]
      \setupTABLE[c][2,3][width={.48\textwidth}]
      \setupTABLE[c][each][align={middle,lohi}]
      \setupTABLE[r][odd][align=right,left={\ini}]

      \bTABLE[split=yes]
      \bTR[samepage=after]
      \bTD \eTD \bTD  $(1)\quad A$ y $B$, si $A \cap B = \emptyset$ \eTD \bTD $(2)\quad A$ y $B$, si $A \subseteq B$ \eTD
      \eTR
      \bTR
      \bTD \eTD \bTD\ \externalfigure[venn01][maxheight=3cm]\eTD \bTD\ \externalfigure[venn02][maxheight=3cm] \eTD
      \eTR

      \bTR[samepage=after]
      \bTD \eTD \bTD $(3)\quad A$ y $B$ \eTD \bTD  \eTD
      \eTR
      \bTR
      \bTD \eTD \bTD\ \externalfigure[venn03][maxheight=3cm] \eTD \bTD \eTD
      \eTR

      \eTABLE
    \stopejemplos

    \startejemplos
      \margindata[youtube]{\from[AE19A]}
      Represente con diagramas de Venn

      \blank
      \setupTABLE[frame=off]
      \setupTABLE[c][1][width=1.2em]
      \setupTABLE[c][2,3][width={.48\textwidth}]
      \setupTABLE[c][each][align={middle,lohi}]
      \setupTABLE[r][odd][align=right,left={\ini}]

      \bTABLE[split=yes]
      \bTR[samepage=after]
      \bTD \eTD \bTD $(1)\quad A, B, C$ con un universo entre ellos \eTD \bTD $(2)\quad A, B, C,$ tal que $ A \subseteq B$ y $A \cap C = \emptyset$ \eTD \eTR
      \bTR
      \bTD \eTD \bTD\ \externalfigure[venn04][maxheight=2.8cm] \eTD \bTD\ \externalfigure[venn05][maxheight=3cm] \eTD \eTR

      \bTR[samepage=after]
      \bTD \eTD \bTD $(3)\quad A, B, C$ con $A \subseteq B$ y $A \cap C /= \emptyset$ \eTD \bTD \eTD
      \eTR
      \bTR
      \bTD \eTD \bTD\ \externalfigure[venn06][maxheight=3cm] \eTD \bTD \eTD
      \eTR
      \eTABLE
    \stopejemplos

    \startejemplos
      Sombree la región indicada en los siguientes diagrama de Venn,
      dadas las condiciones indicadas.

      \blank
      \setupTABLE[frame=off]
      \setupTABLE[c][1][width=1.2em]
      \setupTABLE[c][2,3][width={.48\textwidth}]
      \setupTABLE[c][each][align={middle,lohi}]
      \setupTABLE[r][odd][align=right,left={\ini}]

      \bTABLE[split=yes]
      \bTR[samepage=after]
      \bTD \eTD \bTD $(1)\quad A \cap B$ \eTD \bTD $(2)\quad A \subseteq B, A \subseteq B$ \eTD
      \eTR
      \bTR
      \bTD \eTD \bTD\ \externalfigure[venn07][maxheight=3cm] \eTD \bTD\ \externalfigure[venn08][maxheight=3cm] \eTD
      \eTR
      \eTABLE

      % Truco para que aparezca el enlace cuando se usa TABLE
      \ \margindata[youtube]{\from[AE19B]}

      \setupTABLE[frame=off]
      \setupTABLE[c][1][width=1.2em]
      \setupTABLE[c][2,3][width={.48\textwidth}]
      \setupTABLE[c][each][align={middle,lohi}]
      \setupTABLE[r][odd][align=right,left={\ini}]

      \bTABLE[split=yes]
      \bTR[samepage=after]
      \bTD \eTD \bTD $(3)\quad A \cup B$ \eTD \bTD $(4)\quad A \setminus B$ \eTD
      \eTR
      \bTR
      \bTD \eTD \bTD\ \externalfigure[venn09][maxheight=3cm] \eTD \bTD\ \externalfigure[venn10][maxheight=3cm] \eTD
      \eTR

      \bTR[samepage=after]
      \bTD \eTD \bTD $(5)\quad B \setminus A$, si $A \cap B = \emptyset$ \eTD \bTD $(6)\quad A - B$, si $A \subseteq B$ \eTD
      \eTR
      \bTR
      \bTD \eTD \bTD\ \externalfigure[venn11][maxheight=3cm] \eTD \bTD\ \externalfigure[venn12][maxheight=3cm] \eTD
      \eTR

      \bTR[samepage=after]
      \bTD \eTD \bTD $(7)\quad A \triangle B$ \eTD \bTD $(8)\quad A'$ \eTD
      \eTR
      \bTR
      \bTD \eTD \bTD\ \externalfigure[venn13][maxheight=3cm] \eTD \bTD\ \externalfigure[venn14][maxheight=3cm] \eTD
      \eTR

      \bTR[samepage=after]
      \bTD \eTD \bTD $(9)\quad (A \cup B)'$ \eTD \bTD $(10)\quad B \setminus (A \cup C)$ \eTD
      \eTR
      \bTR
      \bTD \eTD \bTD\ \externalfigure[venn15][maxheight=3cm] \eTD \bTD\ \externalfigure[venn16][maxheight=3cm] \eTD
      \eTR

      \bTR[samepage=after]
      \bTD \eTD \bTD $(11)\quad A \cap B \cap B$ \eTD \bTD $(12)\quad A \triangle (B \cap C)$ \eTD
      \eTR
      \bTR
      \bTD \eTD \bTD\ \externalfigure[venn17][maxheight=3.5cm] \eTD \bTD\ \externalfigure[venn18][maxheight=3.5cm] \eTD
      \eTR

      \eTABLE
    \stopejemplos

  \stopsection

\stopchapter
\stopcomponent