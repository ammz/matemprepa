\startcomponent c_estructuras

  \project project_matemprepa

  \margindata[youtube]{\from[AE20A]}
  \startchapter[title={Estructuras algebraicas y\\ estructuras analíticas}]
    \startsection[title={Estructuras algebraicas}]
      \startsubsection[title={Operaciones}]

      \startdefinicion
        \startitemizer
          \startitem
            \obj{Una operación en un conjunto} es una regla que relaciona a dos elementos de dicho conjunto con un tercero y único elemento.
          \stopitem
          \startitem
            Una operación en un conjunto se llama \obj{binaria}, ssi el tercer elemento también es miembro del conjunto.
          \stopitem
          \startitem
            Si representamos una operación en un conjunto por medio del signo $\ast$, diremos que el \obj{resultado de la operación $\ast$ para los elementos $a$ y $b$ es el elemento $c$}, haciendo uso del símbolo $a \ast b = c$. En este caso, $a$ y $b$ son llamados los \obj{operandos}. 
          \stopitem
        \stopitemizer
      \stopdefinicion

      \startdefinicion
        Sea $A$ un conjunto. El \obj{conjunto potencia de $A$}, denotado con el símbolo $P(A)$, es el conjunto compuesto por los subconjuntos de $A$ 
      \stopdefinicion

      \startejemplo
        Sea $A = \{1,2,3\}$, los subconjuntos de $A$ son \{1,2,3\}, $\emptyset$, \{1\}, \{2\}, \{3\}, \{1,2\}, \{1,3\}, \{2,3\}. Así, el conjunto potencia de $A$ será:

        \startformula
          P(A) = \left\{ \{1,2,3\}, \emptyset, \{1\}, \{2\}, \{3\}, \{1,2\}, \{1,3\}, \{2,3\} \right\}
        \stopformula
      \stopejemplo

      Las operaciones de unión, intersección y diferencia de conjutos son binarias en $P(A)$.

      \startejemplos
        \startitemejem
          \startitem
            Comprobemos lo anterior para los siguientes casos:
            \startejerformula
              \{1\} \cup \{2\} = \{1,2\} \in P(A)
            \stopejerformula
            \startejerformula
              \{1,2\} \cup \{3\} = \{1,2,3\} \in P(A)
            \stopejerformula
            \startejerformula
              \{1,2\} \cap \{2,3\} = \{2\} \in P(A)
            \stopejerformula
            \startejerformula
              \{2\} \setminus \{3\} = \{2\} \in P(A)
            \stopejerformula
            \startejerformula
              \{1\} \setminus \{1,3\} = \emptyset \in P(A)
            \stopejerformula
          \stopitem

          \margindata[youtube]{\from[AE20B]}
          \startitem
            Consideremos el conjunto $Q = \left\{\{1\}, \{1,2\}, \{1,3\}, \{2,3\}\right\}$. Las operaciones unión de conjuntos, intersección de conjuntos y diferencia de conjuntos no son binarias en $Q$.
            \startejerformula
              \{1,2\} \cup \{1,3\} = \{1,2,3\} \nin Q
            \stopejerformula
            \startejerformula
              \{1,2\} \cap \{2,3\} = \{2\} \nin Q
            \stopejerformula
            \startejerformula
              \{1,3\} \setminus \{1\} = \{3\} \nin Q
            \stopejerformula
            \startejerformula
              \{1\} \cup \{1,2\} = \{1,2\} \in Q
            \stopejerformula
            Aunque alguna operación, como la última del ejemplo, da como resultado un elemento de $Q$ eso no implica que sea binaria, ya que para ello se debería dar en todos los casos.
          \stopitem


          \startitem
            Consideremos el conjunto $B = \{a,b,c,d\}$. Definimos las operaciones \quotation{chuma} y la \quotation{multiplicachón} con los signos $\bigoplus$ y $\bigotimes$, respectivamente con las tablas:

            \blank[medium]
            \setupTABLE[width=1.3em]
            \setupTABLE[c][each][align=middle]
            \setupTABLE[r][first][topframe=off]
            \setupTABLE[r][last][bottomframe=off]
            \setupTABLE[c][first,last][leftframe=off,rightframe=off]

            \dontleavehmode
            \hbox \bgroup \ignorespaces

            \bTABLE
            \bTR \bTD $\bigoplus$ \eTD \bTD a \eTD \bTD b \eTD \bTD c \eTD \bTD d \eTD \eTR
            \bTR \bTD a \eTD \bTD a \eTD \bTD b \eTD \bTD c \eTD \bTD d \eTD \eTR
            \bTR \bTD b \eTD \bTD b \eTD \bTD a \eTD \bTD d \eTD \bTD c \eTD \eTR
            \bTR \bTD c \eTD \bTD c \eTD \bTD d \eTD \bTD a \eTD \bTD b \eTD \eTR
            \bTR \bTD d \eTD \bTD d \eTD \bTD c \eTD \bTD b \eTD \bTD a \eTD \eTR
            \eTABLE

            \unskip \quad \ignorespaces

            \bTABLE
            \bTR \bTD $\bigotimes$ \eTD \bTD a \eTD \bTD b \eTD \bTD c \eTD \bTD d \eTD \eTR
            \bTR \bTD a \eTD \bTD a \eTD \bTD a \eTD \bTD a \eTD \bTD a \eTD \eTR
            \bTR \bTD b \eTD \bTD a \eTD \bTD b \eTD \bTD c \eTD \bTD d \eTD \eTR
            \bTR \bTD c \eTD \bTD a \eTD \bTD a \eTD \bTD a \eTD \bTD a \eTD \eTR
            \bTR \bTD d \eTD \bTD a \eTD \bTD b \eTD \bTD c \eTD \bTD d \eTD \eTR
            \eTABLE
            \unskip \egroup

            \startitemize[horizontal,four,a]
              \startitem
                $a \oplus b = b$ 
              \stopitem
              \startitem
                $b \oplus c = d$
              \stopitem
              \startitem
                $c \otimes d = a$
              \stopitem
              \startitem
                $d \otimes b = b$
              \stopitem
            \stopitemize
          \stopitem

          Se puede verificar que estas operaciones son binarias en ese conjunto $B$.

        \stopitemejem
      \stopejemplos

    \stopsubsection

    \startsubsection[title={Propiedades de las operaciones binarias}]

      \startobservacion
        Cuando una operación es binaria en un conjunto decimos que la \obj{ley de cierre o clausura es válida o rige en ese conjunto para esa operación}.
      \stopobservacion

      
      \startdefinicion
        \margindata[youtube]{\from[AE21A]}
        Una operación binaria $\ast$ en un conjunto $A$ se dice que es \obj{conmutativa} ssi $\forall x,y \in A$, ocurre que $x \ast y = y \ast x$. Es decir, no importa el orden en que escribamos los operandos, el resultado será el mismo. Luego, decimos que \obj{en dicho conjunto $A$ la ley conmutativa para la operación $\ast$ rige o es válida}. 
      \stopdefinicion

      \startejemplos
        \startitemejem
          \startitem
            Las operaciones de unión, intersección\footnote{Ya se
              demostró que ${A \cap B} = {B \cap A}$ en \from[AE17B]}
            % \comentario{Ya se demostró que ${A \cap B} = {B \cap A}$ en \from[AE17B]}
            y diferencia simétrica son conmutativas.
          \stopitem
          \startitem
            Se puede verificar que la operación \quotation{chuma} es
            conmutativa. Para ello, habría que tomar cualesquiera dos
            elementos y operarlos en los dos órdenes.
          \stopitem
          \startitem
            La operación de ``multiplicachón'' no es conmutativa. Así,
            \startformula
              c \otimes d = a \neq c = d \otimes c
            \stopformula
          \stopitem
        \stopitemejem
      \stopejemplos

      \startdefinicion
        Una operación binaria $\ast$ en un conjunto $A$ se dice que es \obj{asociativa} ssi $\forall x,y,z \in A$, ocurre que $(x \ast y) \ast z = x \ast (y \ast z)$. Es decir, no importa la forma en que agrupemos los operandos, el resultado será el mismo. Si esto ocurre, decimos que \obj{la ley asociativa $\ast$ rige o es válida para la operación $\ast$ en el conjunto $A$}. 
      \stopdefinicion

      \startejemplos
        \startitemejem
          \startitem
            Las operaciones de unión\footnote{Ya se demostró en \from[AE17B]}
            % \comentario{Ya se demostró en \from[AE17B]}
            e instersección de conjuntos son asociativas.
          \stopitem
          \startitem
            \margindata[youtube]{\from[AE21B]}
            Se puede verificar que las operaciones de \quotation{chuma} y \quotation{multiplicachón} son asociativas. Veamos algunos casos
            \startformula
              (c \oplus d) \oplus b = b \oplus b = a
            \stopformula
            \startformula
              c \oplus (d \oplus b) = c \oplus c = a
            \stopformula
          \stopitem
        \stopitemejem
      \stopejemplos

      \startdefinicion
        En un conjunto $A$, un elemento $e \in A$ se llama \obj{la (o el elemento) identidad o neutro de la operación binaria $\ast$} ssi $\forall x \in A$ ocurre que $e \ast x = x \ast e = x$. Es decir, el elemento identidad o neutro hace que el resultado sea el otro elemento con el que se está operando. Si esto ocurre, decimos que \obj{rige o es válida la existencia de un elemento identidad o neutro para la operación $\ast$ en el conjunto $A$}.
      \stopdefinicion

      \startejemplos
        \startitemejem
          \startitem
            El conjunto nulo o vacío, $\{\} \equiv \emptyset$, es el elemento identidad o neutro de la operación de unión de conjuntos en cualquier conjunto potencia, $P(A)$, de cualquier conjunto $A$.
            \startformula
              A \cup \emptyset = \emptyset \cup A = A
            \stopformula
          \stopitem
          \startitem
            Se puede verificar que $a$ es el elemento identidad o neutro de la operación \quotation{chuma} en el conjunto B. Verifiquémoslo para algunos casos:
            \startformula
              c \oplus a = c
            \stopformula
            \startformula
              a \oplus c = c
            \stopformula
          \stopitem
          \startitem
            Se puede verificar que no hay elemento identidad o neutro para la \quotation{multiplicachón}.
          \stopitem
          \startitem
            Si tomamos el conjunto potencia, $P(A)$, de un conjunto, y $U = P(A)$, veremos que $U$ es el elemento identidad de la operación de intersección en $P(A)$
            \startformula
              x \cap U = U \cap x = x
            \stopformula
          \stopitem
        \stopitemejem

      \stopejemplos

      \startdefinicion
      \margindata[youtube]{\from[AE22A]}
        En un conjunto $A$, con una operación binaria $\ast$, si para $x \in A$ existe un elemento $x' \in A$ de modo que $x \ast x'  = x' \ast x = e$, donde $e$ es el elemento identidad de $\ast$ en $A$, diremos que \obj{$x'$ es el elemento inverso con respecto a $\ast$ del elemento $e$}. Si esto ocurre, decimos que \obj{rige o es válida la existencia del elemento inverso para la operación $\ast$ en el conjunto $A$}.
      \stopdefinicion

      \startdefinicion
        Sea $A$ un conjunto donde están definidas dos operaciones binarias $\ast$ y $\#$.
        \startitemizer
          \startitem
            Decimos que \obj{$\ast$ es distributiva por la izquierda sobre $\#$} ssi $\forall x, y, z \in A$ ocurre que $x \ast (y \# z) = (x \ast y) \# (x \ast z)$.
          \stopitem
          \startitem
            Decimos que \obj{$\ast$ es distributiva por la derecha sobre $\#$} ssi $\forall x, y, z \in A$ ocurre que $(x \# y) \ast z = (x \ast z) \# (y \ast z)$.
          \stopitem
          \startitem
            Si la operación $\ast$ es distributiva sobre la operación $\#$ por la izquierda y por la derecha, entonces diremos que \obj{$\ast$ es distributiva sobre $\#$ }. 
          \stopitem
        \stopitemizer
      \stopdefinicion

      \margindata[youtube]{\from[AE22B]}
      \startejemplos
        \startitemejem
          \startitem
          $\bigotimes$ es distributiva sobre $\bigoplus$ en $B$. Veamos un caso en particular:
          \startejerformula
            b \otimes (d \oplus c) = b \otimes b = b
          \stopejerformula
          \startejerformula
            (b \otimes d) \oplus (b \otimes c) = d \oplus c = b
          \stopejerformula
          \startejerformula
            (d \oplus c) \otimes b = b \otimes b = b
          \stopejerformula
          \startejerformula
            (d \otimes b) \oplus (d \otimes b) = b \oplus a = b
          \stopejerformula
        \stopitem

        \startitem
          $\bigoplus$ no es distributiva sobre $\bigotimes$ ni por la izquierda ni por la derecha. Así:

          \startejerformula
            c \oplus (a \otimes b) = c \oplus a = c
          \stopejerformula
          \startejerformula
            (c \oplus a) \otimes (c \otimes b) = c \otimes d = a
          \stopejerformula
          \startejerformula
            (a \otimes b) \oplus c = (a \oplus c) = c
          \stopejerformula
          \startejerformula
            (a \oplus c) \otimes (b \oplus c) = c \otimes d = a
          \stopejerformula
        \stopitem

        \startitem
          Se puede ver que la operación de unión de conjuntos es distributiva sobre la intersección de conjuntos. También que la operación de intersección de conjuntos es distributiva por la derecha sobre la unión.
        \stopitem
        \stopitemejem
      \stopejemplos

      \startejemplo
        \margindata[youtube]{\from[AE23A]}
        \ini{Demuestre que la intersección de conjuntos es distributivas sobre la unión de  conjuntos por la izquierda. Esto es, $A \cap (B \cup C) = (A \cap B) \cup (A \cap C)$}.
        
        \startdemoejem
          Bastará ver que $A \cap (B \cup C) \subseteq (A \cap B) \cup (A \cap C)$ y que $(A \cap B) \cup (A \cap C) \subseteq A \cap (B \cup C)$.

          Para lo primero, tomamos $x \in A \cap (B \cup C)$. Entonces, $x \in A, x \in (B \cup C)$. Lo anterior significa que $x \in A \land (x \in B \lor x \in C)$. Si ocurre que $x \in B$, como $x \in A$, entonces $x \in A \cap B$. Entonces $x \in (A \cap B) \cup (A \cap C)$. Si ocurre que $x \in C$, como $x \in A$ entonces $x \in A \cap C$. Entonces $x \in (A \cap B) \cup (A \cap C)$. Por lo tanto, $A \cap (B \cup C) \subseteq (A \cap B) \cup (A \cap C)$.

          \margindata[youtube]{\from[AE23B]}
          Para lo segundo, tomamos $x \in (A \cap B) \cup (A \cap C)$. Luego, $x \in A \cap B \lor x \in A \cap C$. Si ocurre que $x \in A \cap B$, entonces $x \in A \land x \in B$. Luego, $x \in B \cup C$, y como $x \in A$, entonces $x \in A \cap (B \cup C)$. Si ocurre que $x \in A \cap C$, entonces $x \in A \land x \in C$. Luego, $x \in B \cup C$, y como $x \in A$, entonces $x \in A \cap (B \cup C)$. Por lo tanto, $(A \cap B) \cup (A \cap C) \subseteq A \cap (B \cup C)$.
      \stopdemoejem
      \stopejemplo

      \startdefinicion
        Considere la operación binaria y conmutativa $\ast$ en un conjunto $A$. La \obj{operación inversa de $\ast$} es la operación binaria $\sim$, ssi $\forall x, y, z \in A$, con $x \ast y = y \ast x = z$, ocurre que $z \sim x = y \land z \sim y = x$.
      \stopdefinicion

      \startejemplo
        En $B = \{a,b,c,d\}$ definimos la operación $\ominus$, llamada \quotation{rechta}, y vea que es la operación inversa de $\oplus$.

        \blank[medium]
        \setupTABLE[width=1.3em]
        \setupTABLE[c][each][align=middle]
        \setupTABLE[r][first][topframe=off]
        \setupTABLE[r][last][bottomframe=off]
        \setupTABLE[c][first,last][leftframe=off,rightframe=off]

        \bTABLE
        \bTR \bTD $\ominus$ \eTD \bTD a \eTD \bTD b \eTD \bTD c \eTD \bTD d \eTD \eTR
        \bTR \bTD a \eTD \bTD a \eTD \bTD b \eTD \bTD c \eTD \bTD d \eTD \eTR
        \bTR \bTD b \eTD \bTD b \eTD \bTD a \eTD \bTD d \eTD \bTD c \eTD \eTR
        \bTR \bTD c \eTD \bTD c \eTD \bTD d \eTD \bTD a \eTD \bTD b \eTD \eTR
        \bTR \bTD d \eTD \bTD d \eTD \bTD c \eTD \bTD b \eTD \bTD a \eTD \eTR
        \eTABLE

        Para $a \oplus b = b$ tenemos $b \ominus a = b$ y $ b \ominus b = a$.

      \stopejemplo

      \startejemplos
        \margindata[youtube]{\from[AE24A]}
        \startitemejem[columns]
          \startitem
            \ini{$b \oplus c = d$}

            $d \ominus c = b$ \par $d \ominus b = c$
          \stopitem
          \startitem
            \ini{$c \oplus d = b$}

            $b \ominus d = d$ \par $b \ominus d = c$
          \stopitem
        \stopitemejem
      \stopejemplos

      \startobservacion
        Un conjunto en donde una o más operaciones han sido definidas y en donde una o más de las propiedades que éstas pueden tener (clausura, conmutatividad, asociatividad, existencia de identidades, existencia de inversos, distributividad y existencia de operaciones inversas) rigen, se lla una \obj{estructura algebraica}.
      \stopobservacion

      Presentamos una tabla resumen considerando el conjunto $A$ con dos operaciones binarias $\ast$ y $\circ$ definidas en él.
      
      \blank
      \setupTABLE[r][first][align=middle, topframe=off]
      \setupTABLE[c][first][leftframe=off]
      \setupTABLE[c][last][rightframe=off]

      \bTABLE
      \bTR \bTD Propiedades \eTD \bTD Definición \eTD \bTD Leyes generales \eTD \eTR
      \bTR \bTD Clausura o cierre \eTD \bTD $x \ast y \in A$ único \eTD \bTD $x \ast y \ast z \ast w \in A$ único \eTD \eTR
      \bTR \bTD Conmutativa \eTD \bTD $x \ast y = y \ast x$ \eTD \bTD $x \ast y \ast z= y \ast x \ast z = z \ast x \ast x = x \ast z \ast y$ \eTD \eTR
      \bTR \bTD Asociativa \eTD \bTD $(x \ast y) \ast z = x \ast (y \ast z)$ \eTD \bTD $(x \ast y \ast z) \ast (t \ast u)= (x \ast y) \ast z \ast (t \ast u)$ = $x \ast (y \ast z \ast t \ast u)$\eTD \eTR
      \bTR \bTD Distributiva por la izquierda\eTD \bTD $x \ast (y \circ z) = (x \ast y) \circ (x \ast z)$ \eTD \bTD $x \ast (y \circ z \circ w) = (x \ast y) \circ (x \ast z) \circ (x \ast w)$ \eTD \eTR
      \bTR \bTD Distributiva por la derecha\eTD \bTD $(x \circ y) \ast z = (x \ast z) \circ (t \ast z)$ \eTD \bTD $(x \circ y \circ z) \ast w  = (x \ast w) \circ (t \ast w) \circ (z \ast w)$ \eTD \eTR
      \eTABLE
    \stopsubsection
  \stopsection

  \margindata[youtube]{\from[AE24B]}
  \startsection[title={Estructuras analíticas}]
    \startsubsection[title={Relaciones}]
      \startdefinicion
        Una \obj{relación} es una regla que asocia a dos elementos de un conjunto o de dos conjuntos.
      \stopdefinicion

      \startejemplos
        \startitemejem
          \startitem
            \ini{Sea $S = \{a,b,c,d,e\}$. ¿Qué elementos de $S$ están asociados por medio de la relación \quotation{igual a}?}

            $a = a$, $b = b$, $c = c$, $d = d$ y $e = e$. Esta relación asocia a cada elemento del conjunto con ellos mismos.
          \stopitem
          \startitem
            \ini{Sea $A = \{1,2,3\}$ y $P(A) = \left\{\{1,2,3\}, \emptyset, \{1,\}, \{2,\}, \{3\}, \{1,2\}, \{1,3\}, \{2,3\} \right\}$. ¿Qué elementos de $P(A)$ están relacionados por la relación \quotation{es subconjunto de}?}

            \startejerformula
              \{1,2,3\} \subseteq \{1,2,3\}
            \stopejerformula
            \startejerformula
              \emptyset \subseteq \{\{1,2,3\}, \emptyset, \{1,\}, \{2,\}, \{3\}, \{1,2\}, \{1,3\}, \{2,3\}
            \stopejerformula
            \startejerformula
              \{1\} \subseteq \{1,2,3\}, \{1\}, \{1,2\}, \{1,3\}
            \stopejerformula
            \startejerformula
              \{2\} \subseteq \{1,2,3\}, \{2\}, \{1,2\}, \{2,3\}
            \stopejerformula
            \startejerformula
              \{3\} \subseteq \{1,2,3\}, \{3\}, \{1,3\}, \{2,3\}
            \stopejerformula
            \startejerformula
              \{1,2\} \subseteq \{1,2,3\}, \{1,2\}
            \stopejerformula
            \startejerformula
              \{1,3\} \subseteq \{1,2,3\}, \{1,3\}
            \stopejerformula
            \startejerformula
              \{2,3\} \subseteq \{1,2,3\}, \{2,3\}
            \stopejerformula
          \stopitem
        \stopitemejem
      \stopejemplos

      \startejemplo
        \margindata[youtube]{\from[AE25A]}
        Suponga que frente a su casa hay dos casas donde viven dos hermanos, D. José y D. Pedro. Cierto día parece que han decidido celebrar una reunión familiar por lo que en casa de D. José están sus tres hijos, Juan, Jaime y Carlos, mientras que en casa de D. Pedro está su hijo Ramón junto con sus tres hijos (y nietos de D. Pedro), Ana, Samuel y Eric.

        ¿Qué elementos del conjunto compuesto por las personas en esas casas están asociados por la relación quotation{es descendiente de} (D)?

        \blank[big]
        \startcenteraligned
          \startMPcode
            z0 = (0cm,2.35cm) ; z1 = (1.5cm,2.35cm) ; z2 = (3cm,2.35cm) ; z3 = (1.5cm,3.35cm) ;
            z4 = (0cm,0cm) ;

            draw (z0 -- z3) withcolor .425white ;
            draw (z1 -- z3) withcolor .425white ;
            draw (z2 -- z3) withcolor .425white ;

            dotlabel.top("José", z3) withcolor .425white ;
            dotlabel.bot("Juan", z0) withcolor .25white ;
            dotlabel.bot("Jaime", z1) withcolor .425white ;
            dotlabel.bot("Carlos", z2) withcolor .425white ;

            draw (z4) ;
          \stopMPcode
          \qquad
          \startMPcode
            z0 = (0cm,0cm) ; z1 = (1.5cm,0cm) ; z2 = (3cm,0cm) ; z3 = (1.5cm,1.5cm) ; z4 = (1.5cm,3cm) ; z5 = (1.5cm,2cm) ;

            draw (z4 -- z5) withcolor .425white ;
            draw (z0 -- z3) withcolor .425white ;
            draw (z1 -- z3) withcolor .425white ;
            draw (z2 -- z3) withcolor .425white ;

            dotlabel.top("Pedro", z4) withcolor .425white ;
            dotlabel.top("Ramón", z3) withcolor .425white ;
            dotlabel.bot("Ana", z0) withcolor .425white ;
            dotlabel.bot("Samuel", z1) withcolor .425white ;
            dotlabel.bot("Eric", z2) withcolor .425white;
          \stopMPcode
        \stopcenteraligned

        \starttabulate[|ri7.8|c|l|cw(1.1cm)|r|c|l|][split=no]
          \NC Juan   \NC D \NC José \NC \NC Ramón  \NC D \NC Pedro \NC \NR
          \NC Jaime  \NC D \NC José \NC \NC Ana    \NC D \NC Ramón \NC \NR
          \NC Carlos \NC D \NC José \NC \NC Samuel \NC D \NC Ramón \NC \NR
          \NC        \NC   \NC      \NC \NC Eric   \NC D \NC Pedro \NC \NR
          \NC        \NC   \NC      \NC \NC Ana    \NC D \NC Pedro \NC \NR
          \NC        \NC   \NC      \NC \NC Samuel \NC D \NC Pedro \NC \NR
          \NC        \NC   \NC      \NC \NC Eric   \NC D \NC Pedro \NC \NR
        \stoptabulate

        ¿Qué elementos de ese conjunto están asociados por medio de la relación \quotation{es hermano de} (H)?

        \starttabulate[|ri7.8|c|l|cw(1.1cm)|r|c|l|]
          \NC José   \NC H \NC Pedro  \NC \NC Jaime  \NC H \NC Carlos \NC \NR
          \NC Pedro  \NC H \NC José   \NC \NC Carlos \NC H \NC Jaime  \NC \NR
          \NC Juan   \NC H \NC Jaime  \NC \NC Samuel \NC H \NC Ana    \NC \NR
          \NC Jaime  \NC H \NC Juan   \NC \NC Samuel \NC H \NC Eric   \NC \NR
          \NC Juan   \NC H \NC Carlos \NC \NC Eric   \NC H \NC Samuel \NC \NR
          \NC Carlos \NC H \NC Juan   \NC \NC Eric   \NC H \NC Ana    \NC \NR
        \stoptabulate

        ¿Qué elementos de ese conjunto están asociados por medio de la relación \quotation{es hermana de} (S)?

        \startcenteraligned
          \starttabulate[|r|c||l]
            \NC Ana \NC S \NC Samuel \NC\NR
            \NC Ana \NC S \NC Eric \NC\NR
          \stoptabulate
        \stopcenteraligned
        
      \stopejemplo


      \startdefinicion
      \margindata[youtube]{\from[AE25B]}
        Una relación $R$ en un conjunto $A$ se llama \obj{un orden parcial en $A$}, ssi $\forall x, y, z \in A$, las siguientes condiciones se cumplen:
        \startitemizer
          \startitem
            $x \mathbin{R} x$
          \stopitem
          \startitem
            $x \mathbin{R} y \land y \mathbin{R} x <--> x = y$
          \stopitem
          \startitem
            $x \mathbin{R} y \land y \mathbin{R} z --> x \mathbin{R} z$
          \stopitem
        \stopitemizer

        El conjunto $A$ junto con el orden parcial $R$ definido en él si llama un \obj{conjunto parcialmente ordenado} (\quotation{poconjunto}).
      \stopdefinicion

      \startejemplos
        \startitemejem
          \startitem
            \ini{$\subseteq$ es un orden parcial en $P(X)$}

            Las siguientes relaciones ya han sido demostradas.
            \startitemize
              \startitem
                $A \subseteq A$
              \stopitem
              \startitem
                $A \subseteq B \land B \subseteq A <--> A = B$
              \stopitem
              \startitem
                $A \subseteq B \land B \subseteq C --> A \subseteq B$
              \stopitem
            \stopitemize
          \stopitem

          \startitem
            D(escendiente) y H(ermano) no son órdenes parciales, ya que, por ejemplo, Eric noD Eric o Samuel noH Samuel.
          \stopitem

          \startitem
            Consideremos el conjunto $A = \{1,2,3,4,p\}$ y el siguiente diagrama

            \startcenteraligned
              \startMPcode
                z2 = (0cm,1cm) ; z4 = (1cm,0cm) ; z3 = (2cm,1cm) ; z1 = (1cm,2cm) ;
                z5 = (3cm,0cm) ;


                drawarrow (z4..z2) withcolor .425white ;
                drawarrow (z4..z3) withcolor .425white ;
                drawarrow (z2..z1) withcolor .425white ;
                drawarrow (z3..z1) withcolor .425white ;
                drawarrow (z5..z3) withcolor .425white ;


                draw thelabel.top("1", z1) withcolor .425white ;
                draw thelabel.lft("2", z2) withcolor .25white ;
                draw thelabel.top("3", z3) withcolor .425white ;
                draw thelabel.bot("4", z4) withcolor .425white ;
                draw thelabel.bot("p", z5) withcolor .425white ;
              \stopMPcode
            \stopcenteraligned

            Definimos la relación $R$ indicando:
            
            $x \mathbin{R} y$ ssi $x = y\; \lor$ pasamos de $x$ a $y$ siguiendo la dirección de las flechas en ese diagrama.

            \starttabulate[|ri7.8|c|l|cw(1.1cm)|r|c|l|][split=no]
              \NC 1 \NC $R$ \NC 1 \NC \NC 4 \NC $R$ \NC 3 \NC \NR
              \NC 2 \NC $R$ \NC 2 \NC \NC 4 \NC $R$ \NC 1 \NC \NR
              \NC 3 \NC $R$ \NC 3 \NC \NC 2 \NC $R$ \NC 1 \NC \NR
              \NC 4 \NC $R$ \NC 4 \NC \NC 3 \NC $R$ \NC 1 \NC \NR
              \NC p \NC $R$ \NC p \NC \NC p \NC $R$ \NC 3 \NC \NR
              \NC 4 \NC $R$ \NC 2 \NC \NC p \NC $R$ \NC 1 \NC \NR
            \stoptabulate

            Podemos comprobar que se cumplen con las propiedades que determinan una relación de orden parcial.
          \stopitem

          \startitem
            \ini{La relación \quotation{es igual a}, $=$, es un orden parcial}.

            Ya vimos las propiedades de la igualdad, que decían que:
            \startitemize[packed,r]
              \startitem
                $x = x$ (ley reflexiva)
              \stopitem
              \startitem
                $x = y --> y = x$ (ley simétrica)
              \stopitem
              \startitem
                $x = y \land y = z --> x = z$ (ley transitiva)
              \stopitem
            \stopitemize
          \stopitem
        \stopitemejem
      \stopejemplos

      \startdefinicion
        \margindata[youtube]{\from[AE26A]}
        Un conjunto $A$ parcialmente ordenado por la relación $R$, se dice que \obj{está totalmente ordenado por $R$} ssi $\forall x,y \in A$, o $x \mathbin{R} y$ o $y \mathbin{R} x$. Esto es, ssi $\forall x,y \in A, x \mathbin{R} y \veebar  y \mathbin{R} x$. En este caso, llamaremos al orden parcial $R$, \obj{un orden total o lineal}.
      \stopdefinicion

      \startejemplos
        \startitemejem
          \startitem
            El orden parcial \quotation{es igual a} no es un orden total pues, debido a la ley simétria, $x = y \land y = x$
          \stopitem
          \startitem
            El orden parcial \quotation{es un subconjunto de} no es un orden total o lineal pues para $\{1\}, \{2\} \in P(A)$, no es cierto que $\{1\} \subseteq \{2\}$ ni que $\{2\}$ \subseteq $\{1\}$.
          \stopitem
          \startitem
            Considere el orden parcial en $A = \{1,2,3,4,p\}$ dado por

            \startcenteraligned
              \startMPcode
                z2 = (0cm,1cm) ; z4 = (1cm,0cm) ; z3 = (2cm,1cm) ; z1 = (1cm,2cm) ;
                z5 = (3cm,0cm) ;


                drawarrow (z4..z2) withcolor .425white ;
                drawarrow (z4..z3) withcolor .425white ;
                drawarrow (z2..z1) withcolor .425white ;
                drawarrow (z3..z1) withcolor .425white ;
                drawarrow (z5..z3) withcolor .425white ;


                draw thelabel.top("1", z1) withcolor .425white ;
                draw thelabel.lft("2", z2) withcolor .25white ;
                draw thelabel.top("3", z3) withcolor .425white ;
                draw thelabel.bot("4", z4) withcolor .425white ;
                draw thelabel.bot("p", z5) withcolor .425white ;
              \stopMPcode
            \stopcenteraligned

            No se trata de un orden total, pues algunos elementos no están relacionados: $4 \mathbin{R} p$ o $p \mathbin{R} 4$
          \stopitem
          \startitem
            Consideremos el conjunto $\{1,3,p\} \subseteq A$.

            \startitemize
              \startitem
                Está parcialmente ordenado
                \startitemize
                  \startitem
                    $1 \mathbin{R} 1, 3 \mathbin{R} 3, p \mathbin{R} p$ 
                  \stopitem
                  \startitem
                    También se cumple que $x \mathbin{R} y \land y \mathbin{R} x <--> x = y$
                  \stopitem
                  \startitem
                    $p \mathbin{R} 3 \land 3 \mathbin{R} 1 --> p \mathbin{R} 1$
                  \stopitem
                \stopitemize
              \stopitem
              \startitem
                Está totalmente ordenado
                \startitemize
                  \startitem
                    $1 \mathbin{R} 3 \veebar 3 \mathbin{R} 1$
                  \stopitem
                  \startitem
                    $1 \mathbin{R} p \veebar p \mathbin{R} 1$
                  \stopitem
                  \startitem
                    $3 \mathbin{R} p \veebar p \mathbin{R} 3$
                  \stopitem
                \stopitemize
                Vemos que se da para todos los casos posibles.
              \stopitem
            \stopitemize
          \stopitem
          \startitem
            Consideremos $\{4,3,p\} \subseteq A$.

            Está parcialmente ordenado. Sin embargo, no es un orden total ya que ni $4 \mathbin{R} p$ ni $p  \mathbin{R} 4$.
          \stopitem
        \stopitemejem
      \stopejemplos
    \stopsubsection
  \stopsection

  \startsection[title={Resumen}]
    \starttabulate[|p|c|p|][split=no]
      \HL
      \NC Estructuras algebraicas \NC  \NC Estructuras analíticas \NC\NR
      \HL
      \NC Conjuntos con una o más operaciones definidas en él y sus operadores \NC  \NC Conjuntos con una o más relaciones definidas en él y sus características \NC\NR
      \NC
      \startitemize[n]
        \startitem
          Cierre o clausura
        \stopitem
        \startitem
          Conmutatividad
        \stopitem
        \startitem
          Asociatividad
        \stopitem
        \startitem
          Existencia de identidades
        \stopitem
        \startitem
          Existencia de inversos
        \stopitem
        \startitem
          Distributividad
        \stopitem
        \startitem
          Existencia de operaciones inversas
        \stopitem
      \stopitemize
      \NC
      \NC
      \startitemize[n]
        \startitem
          Orden parcial
        \stopitem
        \startitem
          Orden total o lineal
        \stopitem
      \stopitemize
      \NC\NR

    \stoptabulate

  \stopsection
  \stopchapter