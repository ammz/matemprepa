\startcomponent c_logica
\project project_matemprepa

\youtube{\from[AE1A]}
\startdiscusion{Bloques de construcción}
  \startitemize
    \startitem
      \obj{términos indefinidos:} conceptos básicos que no requieren
      definición
    \stopitem
    \startitem \obj{términos definidos:} conceptos que requieren
      definición
    \stopitem
    \startitem \obj{axiomas o postulados:} leyes o propiedades
      aceptadas como ciertas
    \stopitem
    \startitem \obj{teoremas o proposiciones:} leyes o propiedades que
      se tienen que demostrar que son ciertas
    \stopitem
    \startitemize
      \startitem \obj{lemas:} teoremas que se usan como una
        justificación en la demostración de un teorema más imporante
      \stopitem
      \startitem
        \obj{corolarios:} teoremas que son casos especiales o
        consecuencias directas de un teorema previo
      \stopitem
    \stopitemize
  \stopitemize
\stopdiscusion

\startchapter[title={Lógica}]
  La \obj{lógica} se encarga del estudio de los argumentos o
  razonamientos válidos

  \startdefinicion
    Una oración que establece un hecho que, en algún momento, puede
    ser clasificado cierto o falso, se llama \obj{una afirmación o
      enunciado}.
  \stopdefinicion

  \obj{variable:} letra que representa una cosa.

  \startejemplos
    \startitemize[packed]
      \startitem
        Está lloviendo. $\rightarrow\; p$. $p$ es una proposición.
      \stopitem
      \startitem
        Siéntate y mantente en silencio. $\rightarrow\; x$. $x$ no es
        una proposición.
      \stopitem
      \startitem
        Tengo los zapatos nuevos puestos. $\rightarrow\; q$.
      \stopitem
      \startitem
        $p$ y $q\; \rightarrow$ Está lloviendo y tendo los zapatos nuevos
        puestos.
      \stopitem
    \stopitemize
  \stopejemplos

  \margindata[youtube]{\from[AE1B]}
  \startsection[title={Estudio de las afirmaciones compuestas}]

    \startdefinicion
      Una afirmación que contiene dos o más afirmaciones se llama
      \obj{una afirmación o enunciado compuesto}.

      \startitemizer[packed]
        \startitem
          \obj{La conjunción de $p$ con $q$}, leída \quote{p y q}, y
          representada con el símbolo $p \wedge q$.
        \stopitem
        \startitem
          \obj{La disjunción o disyunción de $p$ con $q$}, leída
          \quote{p o q} y denotada con el símbolo $q \vee q$.
        \stopitem
        \startitem
          \obj{La disyunción (o disjunción) exclusiva}, leída \quote{p
            o q, pero no ambas} o \quote{o p o q} y representada por
          el símbolo $p \veebar q$.
        \stopitem
      \stopitemize
    \stopdefinicion

    \startaxioma
      Sean $p$ y $q$ dos enunciados. Entoces
      \startitemizer[packed]
        \startitem $p \wedge q$ será cierta si tanto $p$ y $q$ lo son.
        \stopitem
        \startitem $p \vee q$ será cierta si, al menos, una de las
          afirmaciones $p$ o $q$ lo es.
        \stopitem
        \startitem $p \veebar q$ será cierta si, únicamente, una de
          las dos afirmaciones $p$ o $q$ lo es.
        \stopitem
      \stopitemizer
    \stopaxioma

    \startdefinicion
      Si $p$ es una afirmación, \obj{la negación de p}, representada
      con el símbolo $\neg p$, se lee 'no p'.
    \stopdefinicion

    \startaxioma
      Si $p$ es un enunciado cierto, entonces $\neg p$ es falso y
      viceversa.
    \stopaxioma

    \startdiscusion{Valor de verdad de un enunciado o afirmación}
      Solo existen dos valores posibles: C (cierto) y F (falso)
    \stopdiscusion

    \startdiscusion{Tablas de verdad}
      Una \obj{tabla de verdad} es un esquema donde se recogen los
      valores de verdad de una afirmación.
    \stopdiscusion

    \startitemize[a,columns,joinedup]

      \startitem
        Conjunción \par
        \starttable[|cm|c|c|c|]
          \NC \VL $p$ \VL $q$ \VL $p \wedge q$ \NC \AR
          \HL
          \NC c_1 \VL C \VL C \VL C \NC \AR
          \NC c_2 \VL C \VL F \VL F \NC \AR
          \NC c_3 \VL F \VL C \VL F \NC \AR
          \NC c_4 \VL F \VL F \VL F \NC \AR
        \stoptable
      \stopitem

      \margindata[youtube]{\from[AE2A]}
      \startitem
        Disyunción \par
        \starttable[|cm|c|c|c|]
          \NC \VL $p$ \VL $q$ \VL $p \vee q$ \NC \AR
          \HL
          \NC c_1 \VL C \VL C \VL C \NC \AR
          \NC c_2 \VL C \VL F \VL C \NC \AR
          \NC c_3 \VL F \VL C \VL C \NC \AR
          \NC c_4 \VL F \VL F \VL F \NC \AR
        \stoptable
      \stopitem
    \stopitemize

    \startitemize[continue,columns,joinedup]
      \startitem
        Disyunción exclusiva \par
        \starttable[|cm|c|c|c|]
          \NC \VL $p$ \VL $q$ \VL $p \veebar q$ \NC \AR
          \HL
          \NC c_1 \VL C \VL C \VL F \NC \AR
          \NC c_2 \VL C \VL F \VL C \NC \AR
          \NC c_3 \VL F \VL C \VL C \NC \AR
          \NC c_4 \VL F \VL F \VL F \NC \AR
        \stoptable
      \stopitem

      \startitem
        Negación \par
        \starttable[|cm|c|c|]
          \NC \VL $p$ \VL $\neg p$ \NC \AR
          \HL
          \NC c_1 \VL C \VL F \NC \AR
          \NC c_2 \VL F \VL C \NC \AR
        \stoptable
      \stopitem

    \stopitemize

    \startdiscusion{Conexiones de una afirmación compuesta}
      \starttable[|l|c|cm|]
        \NC Conjunción           \NC y                      \NC \wedge  \NC \AR
        \NC Disjunción           \NC o                      \NC \vee    \NC \AR
        \NC Disjunción exclusiva \NC ... o ..., pero no ambas / o ... o... \NC \veebar \NC \AR
      \stoptable
    \stopdiscusion

    \startejemplos
      Suponga que los siguientes enunciados son ciertos o falsos como se
      indica:
      \startitemize[packed,horizontal]
        \startitem
          $p$: Está lloviendo (C)
        \stopitem
        \startitem
          $q$: Tengo los zapatos nuevos puestos (F)
        \stopitem
        \startitem
          $r$: Me vienen a bucar en carro (C)
        \stopitem
        \startitem
          $s$: Salgo antes de las 2 p.m. (F)
        \stopitem
        \startitem
          $t$: Aprobé el examen con calificación de A (C)
        \stopitem
      \stopitemize

      \startitemejem[packed]
        \startitem
          Está lloviendo y tengo los zapatos nuevos puestos.
          $(p \wedge q)$\par C $\wedge$ F $\longrightarrow$ F
        \stopitem
        \startitem
          Está lloviendo y me vienen a buscar en el carro.
          $(q \wedge r)$\par C $\wedge$ C $\longrightarrow$ C
        \stopitem
        \margindata[youtube]{\from[AE2B]}
        \startitem
          Tengo los zapatos nuevos puestos y salgo antes de las 2
          p.m. $(q \wedge s)$\par F $\wedge$ F $\longrightarrow$ F
        \stopitem
        \startitem
          $(s \vee p)\; \longrightarrow\;$ Salgo antes de las 2p.m o
          está lloviendo \par F $\vee$ C $\longrightarrow$ C
        \stopitem
        \startitem
          $q \vee (\neg s)\; \longrightarrow\;$ Tengo los zapatos
          nuevos puestos o no salgo antes de las 2 p.m. \par F $\vee$
          C $\longrightarrow$ C
        \stopitem
        \startitem
          Tengo los zapatos nuevos puestos o aprobé el examen con
          calificación de A, pero no ambas $(q \veebar t)$ \par F
          $\veebar$ C $\longrightarrow$ C
        \stopitem
        \startitem
          O me vienen a buscar en el carro o aprobé el examen con
          calificación de A $(r \veebar t)$ \par C $\veebar$ C
          $\longrightarrow$ F
        \stopitem
        \margindata[youtube]{\from[AE3A]}
        \startitem
          $q \veebar s\;\longrightarrow\;$ Tengo los zapatos nuevos
          puestos o salgo antes de las 2 p.m. \par O tengo los zapatos
          nuevos puestos o salgo antes de las 2 p.m. \par F $\veebar$
          F $\longrightarrow$ F
        \stopitem
      \stopitemejem
    \stopejemplos

    % \comentario{{\bf Signos de agrupación} \par paréntesis $(\,)$ \par paréntesis angulares o corchetes $[\,]$\par llaves $\{\,\}$ \par barras $\overline{\quad}$ }
    \startdiscusion{Tablas de verdad para enunciados más elaborados}
      \startejemplos
        \startitemejem

          \startitem
            $\ini{(p \wedge q) \wedge (\neg p)}$

            \startcenteraligned
              \starttable[|cm|c|c|c|c|c|]
                \NC \VL $p$ \VL $q$ \VL$p \vee q$ \VL $\neg p$ \VL $(p \wedge q) \wedge (\neg p)$\NC \AR
                \HL
                \NC c_1 \VL C \VL C \VL C \VL F \VL C \NC \AR
                \NC c_2 \VL C \VL F \VL C \VL F \VL C \NC \AR
                \NC c_3 \VL F \VL C \VL C \VL C \VL C \NC \AR
                \NC c_4 \VL F \VL F \VL F \VL C \VL C \NC \AR
              \stoptable
            \stopcenteraligned
          \stopitem

        \stopitemejem
      \stopejemplos

      \startdefinicion
        \obj{Una tautología} es un enunciado cierto en todos sus
        posibles casos.
      \stopdefinicion

      \margindata[youtube]{\from[AE3B]}
      \startejemplos
        \startitemejem[continue]

          \startitem
            $\ini{(p \veebar (\neg q)) \wedge p}$

            \startcenteraligned
              \starttable[|cm|c|c|c|c|c|]
                \NC \VL $p$ \VL $q$ \VL $\neg q$ \VL $p \veebar (\neg q)$ \VL $(p \veebar (\neg q)) \wedge p$\NC \AR
                \HL
                \NC c_1 \VL C \VL C \VL F \VL C \VL C \NC \AR
                \NC c_2 \VL C \VL F \VL C \VL F \VL F \NC \AR
                \NC c_3 \VL F \VL C \VL F \VL F \VL F \NC \AR
                \NC c_4 \VL F \VL F \VL C \VL C \VL F \NC \AR
              \stoptable
            \stopcenteraligned
          \stopitem

          \startitem
            $\ini{(p \vee (\neg q)) \wedge (\neg p)}$

            \startcenteraligned
              \starttable[|cm|c|c|c|c|c|c|]
                \NC \VL $p$ \VL $q$ \VL $\neg q$ \VL $p \vee (\neg q)$ \VL $\neg p$ \VL $(p \vee (\neg q)) \wedge (\neg p)$\NC \AR
                \HL
                \NC c_1 \VL C \VL C \VL F \VL C \VL F \VL F \NC \AR
                \NC c_2 \VL C \VL F \VL C \VL C \VL F \VL F \NC \AR
                \NC c_3 \VL F \VL C \VL F \VL F \VL C \VL F \NC \AR
                \NC c_4 \VL F \VL F \VL C \VL C \VL C \VL C \NC \AR
              \stoptable
            \stopcenteraligned
          \stopitem

          \startitem
            $\ini{\neg (p \vee q) \vee \big[\neg (q \vee p)\big]}$

            \startcenteraligned
              \starttable[|cm|c|c|c|c|c|c|c|]
                \NC \VL $p$ \VL $q$ \VL $p \vee q$ \VL $\neg (p \vee q)$ \VL $q \vee p$ \VL $\neg (q \vee p)$ \VL $\neg (p \vee q) \vee \big[\neg (q \vee p)\big]$\NC \AR
                \HL
                \NC c_1 \VL C \VL C \VL C \VL F \VL C  \VL F \VL F \NC \AR
                \NC c_2 \VL C \VL F \VL C \VL F \VL C  \VL F \VL F \NC \AR
                \NC c_3 \VL F \VL C \VL C \VL F \VL C  \VL F \VL F \NC \AR
                \NC c_4 \VL F \VL F \VL F \VL C \VL F  \VL C \VL C \NC \AR
              \stoptable

        \stopcenteraligned
          \stopitem
        \stopitemejem
      \stopejemplos

      \startdefinicion
        Dos enunciados son \obj{equivalentes} si tienen los mismos
        valores de verdad en todos los casos correspondientes.
      \stopdefinicion

      Podemos ver que los valores de verdad de los ejemplos (3) y (4)
      son equivalentes.

      \margindata[youtube]{\from[AE4A]}
      \startejemplos
        \startitemejem
          \startitem
            \ini{Decida si los enunciados $\neg (p \wedge q)\;$ y
              $\;\neg p \vee (\neg q)$ son equivalentes.}

            \startcenteraligned
              \starttable[|cm|c|c|c|c|c|c|c|]
                \NC \VL $p$ \VL $q$ \VL $p \wedge q$ \VL $\neg (p \wedge q)$ \VL $\neg p$ \VL $\neg q$ \VL $\neg p \vee (\neg q)$\NC \AR
                \HL
                \NC c_1 \VL C \VL C \VL C \VL F \VL F \VL F \VL F \NC \AR
                \NC c_2 \VL C \VL F \VL F \VL C \VL F \VL C \VL C \NC \AR
                \NC c_3 \VL F \VL C \VL C \VL C \VL C \VL F \VL C \NC \AR
                \NC c_4 \VL F \VL F \VL F \VL C \VL C \VL C \VL C \NC \AR
              \stoptable
            \stopcenteraligned

            Se comprueba que ambos enunciados tienen los mismos valores
            de verdad y por consiguiente son equivalentes.
          \stopitem

          \startitem
            \ini{Decida si la negación de la disyunción de dos
              enunciados es equivalente a la conjunción de sus
              negaciones}

            Los enunciados a comparar son $\neg (p \vee q)\;$ y
            $\neg p \wedge (\neg q)$.

            \startcenteraligned
              \starttable[|cm|c|c|c|c|c|c|c|]
                \NC \VL $p$ \VL $q$ \VL $p \vee q$ \VL $\neg (p \vee q)$ \VL $\neg p$ \VL $\neg q$ \VL $\neg p \wedge (\neg q)$ \NC \AR
                \HL
                \NC c_1 \VL C \VL C \VL C \VL F \VL F \VL F \VL F \NC \AR
                \NC c_2 \VL C \VL F \VL C \VL F \VL F \VL C \VL F \NC \AR
                \NC c_3 \VL F \VL C \VL C \VL F \VL C \VL F \VL F \NC \AR
                \NC c_4 \VL F \VL F \VL F \VL C \VL C \VL C \VL C \NC \AR
              \stoptable
            \stopcenteraligned

            También se comprueba que ambos enunciados son equivalentes.
          \stopitem
        \stopitemejem
      \stopejemplos

    \stopdiscusion

    \startdefinicion
      \margindata[youtube]{\from[AE4B]} Sean $p$ y $q$ dos
      enunciados. El enunciado compuesto \quotation{si p, entonces q},
      representado con el símbolo $p \longrightarrow q$, se llama
      \obj{la condicional de p con q}.
    \stopdefinicion

    {\bf conexión de la condicional:} \hskip1em si ..., entonces ... $\qquad\longrightarrow$

    \startaxioma
      La siguiente tabla de verdad da los valores de una condicional.

      \startcenteraligned
        \starttable[|cm|c|c|c|]
          \NC \VL $p$ \VL $q$ \VL $p \longrightarrow q$ \NC \AR
          \HL
          \NC c_1 \VL C \VL C \VL C \NC \AR
          \NC c_2 \VL C \VL F \VL F \NC \AR
          \NC c_3 \VL F \VL C \VL C \NC \AR
          \NC c_4 \VL F \VL F \VL C \NC \AR
        \stoptable
      \stopcenteraligned
    \stopaxioma

    \startdefinicion
      La condicional $q \longrightarrow p$ se llama \obj{el recíproco
        o converso} de la condicional $p \longrightarrow q$.
    \stopdefinicion

    \startcomentario
      Otras formas de expresar el recíproco (esta observación se
      discute en el video \from[AE5B]):
      \startitemize[packed,a][before=,after=]
        \startitem
          $q \longrightarrow p$
        \stopitem
        \startitem
          Sólo $p \longrightarrow q$
        \stopitem
        \startitem
          $q$ sólo si $p$
        \stopitem
      \stopitemize
    \stopcomentario

    \startejemplo
      \startitemize[packed,intro][before=,after=,]
        \startitem
          Condicional: Si está lloviendo, entonces te voy a buscar en
          el carro.
        \stopitem
        \startitem
          Recíproco: Si te voy a buscar en el carro, entonces está
          lloviendo.
        \stopitem
      \stopitemize
    \stopejemplo

    \startdiscusion{¿Es el recíproco o converso equivalente a la
      condicional original?}

      \startcenteraligned
        \starttable[|cm|c|c|c|c|c|]
          \NC \VL $p$ \VL $q$ \VL $q \longrightarrow p$ \VL $p \longrightarrow q$ \NC\AR
          \HL
          \NC c_1 \VL C \VL C \VL C \VL C \NC \AR
          \NC c_2 \VL C \VL F \VL C \VL F \NC \AR
          \NC c_3 \VL F \VL C \VL F \VL C \NC \AR
          \NC c_4 \VL F \VL F \VL C \VL C \NC \AR
        \stoptable
      \stopcenteraligned

      Vemos que no son equivalentes.
    \stopdiscusion

    \startejemplo
      \startitemize[packed][before=,after=,]
        \startitem
          Condicional: Si está lloviendo, entonces te voy a buscar en
          el carro.
        \stopitem
        \startitem
          Recíprocos:
          \startitemize[packed,a]
            \startitem
              Si te voy a buscar en el carro, entonces está lloviendo.
            \stopitem
            \startitem
              Sólo si está lloviendo, entonces te voy a buscar en el
              carro.
            \stopitem
            \startitem
              Te voy a buscar en el carro sólo si está lloviendo
            \stopitem
          \stopitemize
        \stopitem
      \stopitemize
    \stopejemplo

    \startdefinicion
      La condicional $\neg p \longrightarrow (\neg q)$ se llama \obj{el inverso}
      de la condicional $p \longrightarrow q$.
    \stopdefinicion

    \startejemplo
      \startitemize[packed][before=,after=,]
        \startitem
          Condicional: Si está lloviendo, entonces te voy a buscar en
          el carro.
        \stopitem
        \startitem
          Inverso: Si no está lloviendo, entonces no te voy a buscar
          en el carro.
        \stopitem
      \stopitemize
    \stopejemplo

    \margindata[youtube]{\from[AE5A]}
    \startdiscusion{¿Es el inverso equivalente a la condicional
      original?}
      \startcenteraligned
        \starttable[|cm|c|c|c|c|c|c|c|]
          \NC \VL $p$ \VL $q$ \VL $\neg p$ \VL $\neg q$ \VL $\neg p \longrightarrow (\neg q)$ \VL $p \longrightarrow q$ \NC\AR
          \HL
          \NC c_1 \VL C \VL C \VL F \VL F \VL C \VL C \NC \AR
          \NC c_2 \VL C \VL F \VL F \VL C \VL C \VL F \NC \AR
          \NC c_3 \VL F \VL C \VL C \VL F \VL F \VL C \NC \AR
          \NC c_4 \VL F \VL F \VL C \VL C \VL C \VL C \NC \AR
        \stoptable
      \stopcenteraligned

      $\therefore\;$ el inverso no es equivalente a la condicional
      original
    \stopdiscusion

    \startobservacion
      El inverso y el recíproco de la condicional son equivalentes.
    \stopobservacion

    \startdefinicion
      La condicional $\neg q \longrightarrow (\neg p)$ se llama \obj{el
        contrapositivo} de la condicional $p \longrightarrow q$.
    \stopdefinicion

    \startejemplo
      \startitemize[packed][before=,after=,]
        \startitem
          Condicional: Si está lloviendo, entonces te voy a buscar en
          el carro.
        \stopitem
        \startitem
          Contrapositivo: Si no te vengo a buscar en el carro,
          entonces no está lloviendo.
        \stopitem
      \stopitemize
    \stopejemplo

    \startdiscusion{¿Es el contrapositivo de una condicional
      equivalente a ésta?}
      \startcenteraligned
        \starttable[|cm|c|c|c|c|c|c|c|]
          \NC \VL $p$ \VL $q$ \VL $\neg q$ \VL $\neg p$ \VL $\neg q \longrightarrow (\neg p)$ \VL $p \longrightarrow q$ \NC\AR
          \HL
          \NC c_1 \VL C \VL C \VL F \VL F \VL C \VL C \NC \AR
          \NC c_2 \VL C \VL F \VL C \VL F \VL F \VL F \NC \AR
          \NC c_3 \VL F \VL C \VL F \VL C \VL C \VL C \NC \AR
          \NC c_4 \VL F \VL F \VL C \VL C \VL C \VL C \NC \AR
        \stoptable
      \stopcenteraligned

      $\therefore\;$ el contrapositivo es equivalente a la condicional
      original
    \stopdiscusion

    \margindata[youtube]{\from[AE5B]}
    \startejemplo
      \ini{Verifique que el enunciado
        $\big[p \wedge (p \longrightarrow q)\big] \longrightarrow q$ es una tautología.}

      \startcenteraligned
        \starttable[|cm|c|c|c|c|c|]
          \NC \VL $p$ \VL $q$ \VL $p \longrightarrow q$ \VL $p \wedge (p \longrightarrow q)$ \VL $\big[p \wedge (p \longrightarrow q)\big] \longrightarrow q$ \NC\AR
          \HL
          \NC c_1 \VL C \VL C \VL C \VL C \VL C \NC \AR
          \NC c_2 \VL C \VL F \VL F \VL F \VL C \NC \AR
          \NC c_3 \VL F \VL C \VL C \VL F \VL C \NC \AR
          \NC c_4 \VL F \VL F \VL C \VL F \VL C \NC \AR
        \stoptable
      \stopcenteraligned

      $\therefore\;$ es una tautología.
    \stopejemplo

    \startdefinicion
      Sean $p$ y $q$ dos enunciados. La afirmación \quote{p si y sólo
        si q}, que se representa con el símbolo
      $p \longleftrightarrow q$, se llama \obj{la bicondicional} de p
      y q y significa que
      $(p \longrightarrow q) \wedge (q \longrightarrow p)$.
    \stopdefinicion

    \startaxioma
      \margindata[youtube]{\from[AE6A]} La siguiente tabla da los
      valores de verdad de la bicondicional $p \longleftrightarrow q$.

      \startcenteraligned
        \starttabulate[|cm|c|c|c|c|c|]
          \NC \NC   \NC   \NC   \NC   \VL $(p \longrightarrow q) \wedge (q \longrightarrow p)$  \NC \AR
          \NC \VL $p$ \VL $q$ \VL $p \longrightarrow q$ \VL $q \longrightarrow p$ \VL $p \longleftrightarrow q$ \NC\AR
          \HL
          \NC c_1 \VL C \VL C \VL C \VL C \VL C \NC \AR
          \NC c_2 \VL C \VL F \VL F \VL C \VL F \NC \AR
          \NC c_3 \VL F \VL C \VL C \VL F \VL F \NC \AR
          \NC c_4 \VL F \VL F \VL C \VL C \VL C \NC \AR
        \stoptabulate
      \stopcenteraligned
    \stopaxioma

    \blank[3*big]
    \startplacetable[title=Resumen de enunciados compuestos, location=here]
      \starttabulate[|p(3.5cm)|rm|p(5.5cm)|]
        \HL
        \NC {\tfx CONJUNCIÓN}
        \NC p \wedge q:
        \NC cierta si ambas lo son.
        \NC\FR
        \TB[halfline]
        % \HL
        \NC {\tfx DISJUNCIÓN}
        \NC p \vee q:
        \NC cierta cuando al menos una lo es.
        \NC\MR
        \TB[halfline]
        % \HL
        \NC {\tfx DISJUNCIÓN EXCLUSIVA}
        \NC p \veebar q:
        \NC cierta cuando únicamnete una lo es.
        \NC\MR
        \TB[halfline]
        % \HL
        \NC {\tfx CONDICIONAL}
        \NC p \longrightarrow q:
        \NC cierta en todos los casos excepto si la primera es cierta
        y la segunda es falsa.
        \NC\MR
        \TB[halfline]
        % \HL
        \NC {\tfx BICONDICIONAL}
        \NC p \longleftrightarrow q:
        \NC cierta cuando ambas sean ciertas o cuando ambas sean
        falsas.
        \NC\LR \HL
      \stoptabulate
    \stopplacetable

    \startdiscusion{Revisión de la condicional}
      La frase \quote{si $p$ entonces $q$} tiene una implicación: será
      suficiente que ocurra $p$ para que en consecuencia ocurra
      $q$. Por lo tanto, en una condicional a la primera afirmación se
      la conoce como una condición suficiente y a la segunda se le
      dice que es una condición necesaria.

      \startformula
        \startalign[n=3]
          \NC p \NC \longrightarrow \NC q \NR
          \NC \text{\small condición suficiente} \NC \NC \text{\small condición necesaria} \NR
        \stopalign
      \stopformula

      Podemos decir que es suficiente que ocurra $p$ para que en
      consecuencia ocurra $q$. Y también, que si ocurre $p$
      necesariamente tiene que ocurrir $q$.

      \startejemplos{Otras formas de expresar la condicional}
        \startitemize[packed]
          \startitem
            Si está lloviendo, entonces te voy a buscar en el carro.
          \stopitem
          \startitem
            Estar lloviendo es una condición suficiente para irte a
            buscar en el carro.
          \stopitem
          \startitem
            Irte a buscar en el carro es una condición necesaria para
            que esté lloviendo.
          \stopitem
        \stopitemize
      \stopejemplos

    \stopdiscusion

    \startdiscusion{Revisión de la bicondicional}
      En una bicondicional ambas afirmaciones son condición suficiente
      y necesaria.

      \startejemplos
        \startitemize[packed]
          \startitem
            Mi mamá me ama si y sólo si yo soy bueno.
          \stopitem
          \startitem
            Mi mamá me ama es una condición necesaria y suficiente
            para yo ser bueno.
          \stopitem
          \startitem
            Yo ser bueno es una condición necesaria y suficiente para
            que mi mamá me ame.
          \stopitem
        \stopitemize
      \stopejemplos

    \stopdiscusion

    \startdefinicion
      \margindata[youtube]{\from[AE6B]} En la condicional $p \longrightarrow q$,
      al enunciado $p$ se le llama \obj{la hipótesis de la
        condicional} mientras que al $q$ se le llama \obj{la
        conclusión de la condicional}.
    \stopdefinicion

    \startejemplos
      {\ini Identifique las hipótesis y la conclusión en las
        siguientes condicionales.}
      \startitemize[n,intro]
        \startitem
          Si $\underbrace{\text{está lloviendo}}_{h}$, entonces
          $\underbrace{\text{te voy a buscar en el carro}}_{c}$.
        \stopitem
        \startitem
          $\underbrace{\text{Estar hambriento}}_{c}$ es una condición
          necesaria para
          $\underbrace{\text{comerme dos raciones de comidad.}}_{h}$
        \stopitem
        \startitem
          Sólo si $\underbrace{\text{ceno temprano}}_{c}$ entonces
          $\underbrace{\text{iré al cine.}}_{h}$
        \stopitem
        \startitem
          $\underbrace{\text{Tener el tanque del coche lleno}}_{h}$ es
          una condición suficiente para
          $\underbrace{\text{dar un paseo largo.}}_{c}$
        \stopitem
        \startitem
          $\underbrace{\text{Llevaré mi paraguas}}_{h}$, sólo si
          $\underbrace{\text{está nublado.}}_{c}$
        \stopitem
      \stopitemize
    \stopejemplos

  \stopsection


  \startsection[title={Métodos de prueba}]

    \startdescripcion{Método directo}
      $p \longrightarrow q$
    \stopdescripcion

    \startdescripcion{Método indirecto, de contradicción o de
      reducción al absurdo}
      Para demostrar que $p$ es cierta, suponemos que es cierto
      $\neg p$ y argumentamos hasta contradecir algo que es cierto.
    \stopdescripcion

    \startdescripcion{Prueba doble}
      $p \longleftrightarrow q$. Hay que hacer las dos pruebas
      $(p \longrightarrow q) \wedge (q \longrightarrow p)$.
    \stopdescripcion

    \startdescripcion{Prueba del contrapositivo}
      $p \longrightarrow q$.  Recordamos que
      $\neg q \longrightarrow (\neg p)$ es una proposición
      equivalente.
    \stopdescripcion

    \margindata[youtube]{\from[AE7A]}
    \startdescripcion{Prueba de existencia}
      Se puede comprobar la existencia por dos métodos:
      \startitemize[packed]
        \startitem
          {\bf por construcción}: formar el objeto a partir de sus
          características.
        \stopitem
        \startitem
          {\bf por presentación}: presenta el objeto y comprueba que
          cumple con las condiciones.
        \stopitem
      \stopitemize
    \stopdescripcion

    \startdescripcion{Prueba de existencia única}
      Suponemos que existe otro objeto distinto al que el teorema
      supone que es único y llegamos al mismo objeto que se alega que
      es único, luego la conclusión es que es el mismo objeto que
      queríamos demostrar que era único.
    \stopdescripcion

    \startdescripcion{Desprueba por un contraejemplo}
      Todos los teoremas comienzan como una \obj{conjetura}\footnote{
        {\bf Conjetura:} teorema que todavía no se ha demostrado que
        es cierto.} hasta que se demuestre que son ciertos o aparece
      un contraejemplo que prueba que la conjetura es falsa.
    \stopdescripcion

    \startejemplo
      Demuestre que: \ini{Si yo no como dulces de chocolate, entonces
        yo siento calor.}

      Datos ciertos:
      \startitemize[packed, a, intro]
        \startitem
          Si voy al cine, entonces yo cmo dulces de chocolate.
        \stopitem
        \startitem
          Me tomo un refresco o me tomo un jugo.
        \stopitem
        \startitem
          Si yo me tomo un refresco, entonces siento calor.
        \stopitem
        \startitem
          Yo no como dulces de chocolate.
        \stopitem
        \startitem
          Si yo viajo por carro, entonces yo siento fresco.
        \stopitem
        \startitem
          Si yo tomo jugo, entonces yo voy al cine.
        \stopitem
      \stopitemize

      \startdemoejem
        Aplicamos el \ini{método directo}.

        Por hipótesis, yo no como dulces de chocolate. Entonces por el
        contrapositivo de (a), usted no va al cine. Luego, por el
        contrapositivo de (f) usted no toma jugo. Entonces, por (b)
        usted se toma un refresco. Luego, por (c) usted siente calor.
      \stopdemoejem
    \stopejemplo

    \margindata[youtube]{\from[AE7B]}
    \startejemplo
      Demuestre que: \ini{Se daña el carro.}

      Datos ciertos:
      \startitemize[packed,a,intro]
        \startitem
          Está haciendo calor.
        \stopitem
        \startitem
          Yo completé mi tarea.
        \stopitem
        \startitem
          Si anuncian tormenta, entonces suspenden las clases.
        \stopitem
        \startitem
          Si suspenden las clases, entonces yo me voy para mi pueblo.
        \stopitem
        \startitem
          Si yo me voy para mi pueblo, entonces yo no completé mi
          tarea.
        \stopitem
        \startitem
          Si no se daña el carro, entonces yo no me voy para la playa.
        \stopitem
        \startitem
          Si yo me voy para la playa, entonces estamos en el mes de
          septiembre.
        \stopitem
        \startitem
          Si estamos en el mes de septiembre, entonces anuncian
          tormeta.
        \stopitem
      \stopitemize

      \startdemoejem
        Utilizamos el \ini{método indirecto}.

        Suponga que no se daña el carro. Entonces por (f) yo me voy
        para la playa. Luego, por (g) estamos en el mes de
        septiembre. En consecuecia, por (h), anuncian
        tormenta. Entonces, por (c) suspenden las clases. Luego, por
        (d) yo me voy para mi pueblo. En consecuencia, por (e), yo no
        completé mi tarea. Esto es una contradicción al dato (b).
      \stopdemoejem
    \stopejemplo

    \margindata[youtube]{\from[AE8A]}
    \startejemplo
      Demuestre que: \ini{Si está lloviendo, entonces me pongo los
        zapatos nuevos.}

      Datos ciertos:
      \startitemize[packed,a,intro]
        \startitem
          Si no me pongo los zapatos nuevos, entonces voy a la
          biblioteca.
        \stopitem
        \startitem
          Si apruebo el examen de matemáticas con calificación de A,
          entonces no está lloviendo.
        \stopitem
        \startitem
          Si no estudio para el examen de inglés, entonces apruebo el
          examen de matemáticas con calificación de A.
        \stopitem
        \startitem
          Si voy a la biblioteca, entonces no estudio para el examen
          de inglés.
        \stopitem
      \stopitemize

      \startdemoejem
        Utilizamos el \ini{método del contrapositivo.}

        Suponga que no me pongo los zapatos nuevos. Entonces, por (a)
        yo voy a la biblioteca. Luego, por (d) yo no estudio para
        examen de inglés. En consecuencia, por (c), aprueba el examen
        de matemáticas con calificación de A. Finalmente, por (b), no
        está lloviendo.
      \stopdemoejem

      Se ha utilizado el \obj{formato de demostración por
        párrafo}. Sin embargo, también existe el \obj{formato por
        columna}.

      \startcenteraligned
        \starttable[|l|l|]
          \NC \REF[c]{Paso} \NC \REF[c]{Razón} \NC\AR
          \HL
          \NC (1) no me pongo los zapatos nuevos          \NC (1) Suposición \NC\AR
          \NC (2) voy a la biblioteca                     \NC (2) (a)        \NC\AR
          \NC (3) no estudio para el examen de inglés     \NC (3) (d)        \NC\AR
          \NC (4) apruebo el examen de matemáticas con A  \NC (4) (c)        \NC\AR
          \NC (5) no está lloviendo                       \NC (5) (b)        \NC\AR
        \stoptable
      \stopcenteraligned

    \stopejemplo

    \margindata[youtube]{\from[AE8B]}
    \startejemplo
      Demuestre que: \ini{Hace calor si y sólo si como papas.}

      Datos ciertos:
      \startitemize[packed,a,intro]
        \startitem
          Si hace calor, entonces me pongo los zapatos nuevos.
        \stopitem
        \startitem
          Me pongo los zapatos nuevos si y sólo si voy al cine.
        \stopitem
        \startitem
          Voy al cine si y sólo si compro papas.
        \stopitem
        \startitem
          Si me pongo los zapatos nuevos, entonces no almuerzo a las
          doce del mediodía.
        \stopitem
        \startitem
          Si no almuerzo a las doce del mediodía, entonces hace calor.
        \stopitem
      \stopitemize

      \startdemoejem
        Utilizamos el \ini{método de la doble prueba.}

        $\Longrightarrow$ Suponenmos que hace calor. Entonces, debido a (a), me
        pongo los zapatos nuevos. Luego, de acuerdo con (b), voy al
        cine. En consecuencia, por (c) como papas.

        $\Longleftarrow$ Suponemos que como papas. Entonces, por (c) voy al
        cine. Luego, por (b), me pongo los zapatos nuevos. Luego, por
        (d), no almuerzo a las doce del mediodía. Luego, por (e), hace
        calor.
      \stopdemoejem

    \stopejemplo

    \startejemplo
      Demuestre que: \ini{Hace calor si y sólo si como papas.}

      Datos ciertos:
      \startitemize[packed,a,intro]
        \startitem
          Si hace calor, entonces me pongo los zapatos nuevos.
        \stopitem
        \startitem
          Me pongo los zapatos nuevos si y sólo si voy al cine.
        \stopitem
        \startitem
          Voy al cine si y sólo si compro papas.
        \stopitem
        \startitem
          Si no hace calor, entonces almuerzo a las docde del
          mediodía.
        \stopitem
        \startitem
          Si almuerzo a las doce de mediodía, entonces como postre.
        \stopitem
        \startitem
          Si como postre, entonces no como papas.
        \stopitem
      \stopitemize

      \startdemoejem{}

        $\Longrightarrow$ Suponemos que hace calor. Entonces, debido a
        (a), me pongo los zapatos nuevos. Luego, de acuerdo con (b), voy
        al cine. En consecuencia, por (c) como papas.

        $\Longleftarrow$ Ahora necesitamos usar el \ini{método del
          contrapositivo.} Suponemos que no hace calor. Entonces, por
        (d), almuerzo antes de las doce del mediodía. Luego, por (e),
        como postre. Entonces, por (f), no como papas.
      \stopdemoejem
    \stopejemplo

    \margindata[youtube]{\from[AE9A]}
    \startejemplo
      Demuestre: \ini{Existe un animal cuya cabeza posee las
        siguientes partes:}

      \startcenteraligned
        \externalfigure[partes.png][maxwidth=0.3\textwidth]
      \stopcenteraligned

      \startdemoejem{Este es un caso de \ini{prueba de existencia.}}

        \startcenteraligned
          \starttable[|lp(4cm)|lp(4cm)|]
            \NC \REF[c]{\ini{Por construcción}}
            \VL \REF[c]{\ini{Por presentación}}
            \NC\AR
            \HL
            \NC Vamos colocando las partes hasta obtener el animal.
            \VL Partimos de una representación y comprobamos que
            coinciden las partes.
            \NC\AR
            \NC
            \startcenteraligned
              \externalfigure[construccion.png][maxwidth=0.2\textwidth]
            \stopcenteraligned
            \VL
            \startcenteraligned
              \externalfigure[presentacion.png][maxwidth=0.3\textwidth]
            \stopcenteraligned
            \NC\AR
          \stoptable
        \stopcenteraligned

      \stopdemoejem
    \stopejemplo

  \stopsection


  \margindata[youtube]{\from[AE9B]}
  \startsection[title={Cuantificadores}]

    Los \obj{cuantificadores} son palabras o frases que indican
    cantidad en algún enunciado o afirmación.

    \startdiscusion{Cuantificadores matemáticos}

      \startitemize[packed,n]
        \startitem
          existe: $\exists$
        \stopitem
        \startitem
          para todo: $\forall$
        \stopitem
        \startitem
          existe un sólo: $\exists!$
        \stopitem
      \stopitemize
    \stopdiscusion

    \startdiscusion{Presentación en matemáticas de los enunciados que
      usan identificadores}
      Sean $x$ un objeto y $p$ una afirmación,
      \startitemize[packed,n]
        \startitem
          Podemos decir: \ini{$\;\exists\, x $ de modo que $p$}, o
          bien, \ini{$\exists\, x \;\cdot\ni\cdot\; p\;$} o
          \ini{$\;\exists\, x$ t.q. $p$}.
        \stopitem
        \startitem
          \ini{$\forall\, x, p$}
        \stopitem
        \startitem
          \ini{$\exists!\, x \;\cdot\ni\cdot\; p\;$} o
          \ini{$\;\exists!\, x$ t.q. $\,p$}
        \stopitem
      \stopitemize
    \stopdiscusion

    \startdiscusion{Negación de los cuantificadores}
      \startitemize[packed,n]
        \startitem
          $\exists\, x \;\cdot\ni\cdot\; p\qquad;\qquad
          \ini{\nexists\,x \;\cdot\ni\cdot\; p}$ o preferiblemente
          \ini{$\forall\, x, \neg p$}.
        \stopitem
        \startitem
          $\forall\, x, p\qquad;\qquad \ini{\exists\, x
          \;\cdot\ni\cdot\; \neg p}$
        \stopitem
      \stopitemize
    \stopdiscusion

  \stopsection
\stopchapter

\stopcomponent