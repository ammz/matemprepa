
\startcomponent c_enteros

\project project_matemprepa

  \startchapter[title={El conjunto de los números enteros}]

    \margindata[youtube][method=top]{\from[AE34B]}
    \startsection[title={El inverso aditivo}]
      \startaxioma
        $\forall n \in \naturalnumbers, \exists -n$ tal que $n + (-n) = -n + n = 0$. Llamamos a $-n$ \obj{el inverso aditivo (o con respecto a la suma) o el negativo de n}.
      \stopaxioma

      Además de los cardinales $\mathbb{W} = \{0, 1, 2, 3, \dots\}$, ahora tenemos el conjunto de los negativos de $\naturalnumbers$, $\{-1, -2, -3, \dots\}$ y si unimos a ambos obtenemos

      \startformula
        \mathbb{W} \cup \naturalnumbers =\{\dots, -3, -2, -1, 0, 1, 2, 3, \dots \} = \integers
      \stopformula

      que conocemos como \obj{el conjunto de los números enteros}.

      Recordemos que $\naturalnumbers \subseteq \mathbb{W} \subseteq \integers$.

      \margindata[youtube]{\from[AE35A]}
      Tenemos que $\integers = \{\dots, -3,-2,-1, \overbrace{0, 1, 2, 3, \dots}^{\mathbb{W}}\}$. Para los números cardinales, $\mathbb{W}$, rigen ciertas propiedades para (+) y (\times).

      Ya conocemos que $-n + n = n + (-n) = 0$ y 0 es el único elemento identidad para la suma en \mathbb{W}.

      Vemos, pues, que $\forall n \in \integers,\; n + 0 = 0 + n = n$; o sea existe elemento neutro para la suma en $\integers$.

      \startaxioma
        0 es el único elemento identidad para la suma en $\integers$.
      \stopaxioma

      Si $n \in \naturalnumbers,\; -(-n) = n$ ya que $-n + \left[-(-n)\right] = 0$, y sabemos que $-n + n = 0$, entonces debe ocurrir que $-(-n) = n$.

      \startejemplos
        \startitemejem[horizontal,three,intro, joinedup]
          \startitem
            $\ini{-(-1)} = 1$
          \stopitem
          \startitem
            $\ini{-(-16)} = 16$
          \stopitem
          \startitem
            $\ini{-(-1075)} = 1075$
          \stopitem
        \stopitemejem
      \stopejemplos

      \startaxioma
      $\forall x \in \integers,\, \exists -x \in \integers$ tal que $x + (-x) = -x + x = 0)$.
      \stopaxioma

      \startdiscusion{¿Cuál es el inverso aditivo del 0?}

        \startteorema
          - 0 = 0.
        \stopteorema
      
        \startdemo
          Tiene que ocurrir que $0 + (-0) = 0$. Esto se puede reescribir, por la ley simétrica de la igualdad, como
          \startformula
            0 = 0 + (-0)
          \stopformula
          Y, puesto que, 0 es el elemento identidad de la suma en $\integers$,
          \startformula
            0 + (-0) = -0.
          \stopformula
          \startformula
            \therefore 0 = -0, \text{ que se reescribe como } -0 = 0.
          \stopformula
        \stopdemo

      \stopdiscusion

    \stopsection

    \startsection[title={El conjunto de los enteros positivos}]
      \startdefinicion
        \margindata[youtube]{\from[AE35B]}
        \startitemizer
          \startitem
            Llamamos a $\integers^{+} \subseteq \integers$, \obj{el conjunto de los números enteros positivos}, ssi $\integers^{+}$ cumple con las siguientes condiciones
            \startitemize[a]
              \startitem
                (\obj{ley de cierre de la suma})
                \startformula
                  a, b \in \integers^{+} --> a + b \in \integers^{+}.
                \stopformula
              \stopitem
              \startitem
                (\obj{ley de cierre de la multipliacación})
                \startformula
                  a, b \in \integers^{+} --> ab \in \integers^{+}.
                \stopformula
              \stopitem
              \startitem
                (\obj{ley de tricotomía}) \par Para cada $a \in \integers$, una y sólo una de las siguientes afirmaciones es cierta:
                \startformula
                  a = 0 \quad\veebar\quad a \in \integers^{+} \quad\veebar\quad -a \in \integers^{+}.
                \stopformula
              \stopitem
            \stopitemize
          \stopitem
          \startitem
            Si $a \nin \integers^{+}$ y $ a \neq 0$, decimos que \obj{$a$ es un número negativo}.
          \stopitem
          \startitem
            Decimos que para $a,b \in \integers$, \obj{$a$ es mayor (o más grande) que $b$}, representado con el símbolo $ a > b$, ssi $a - b \in \integers^{+}$.
          \stopitem
          \startitem
            Decimos que para $a,b \in \integers$, \obj{$a$ es menor (o más pequeño) que $b$}, representado con el símbolo $ a < b$, ssi $b > a$.
          \stopitem
        \stopitemizer
      \stopdefinicion

      \startobservacion
        Tomemos $a,b \in \integers$.
        \startitemize[n,unpacked]
          \startitem
            $a > 0$ ssi $a - 0 \in \integers^{+}$, pero $a - 0 = a$, luego $a \in \integers^{+}$.
            \startformula
              \therefore a > 0 \text{ ssi } a \in \integers^{+}.
            \stopformula
          \stopitem
          \startitem
            $0 = 0$, entonces, por la ley de tricotomía, $0 \nin \integers^{+}$. Por tanto, el 0 no es un número entero positivo.
          \stopitem
          \startitem
            El símbolo $a \geq b$ significa que $a > b \veebar a = b$ y \par el símbolo $a \leq b$ significa que $a < b \veebar a = b$.
          \stopitem
          \startitem
            Cualquier enunciado o afirmación que contenga uno o más de los signos $>, <, \geq \text{ o } \leq$ se llama \obj{una desigualdad}.
          \stopitem
        \stopitemize
      \stopobservacion

    \stopsection

    \startsection[title={Propiedades de los números enteros}]
      \startaxioma
        \comentario{Se trata de las mismas leyes que hemos visto para los números cardinales y que se transfieren a los enteros sin discusión.}
        Sean $a,b,c \in \integers$. Entonces:
        \startitemizer

          \startitem
            \obj{Leyes de clausura}
            \startitemize[a]
              \startitem
                (\obj{suma}) \quad $a + b \in \integers$, único
              \stopitem
              \startitem
                (\obj{multiplicación}) \quad $a \cdot b \in \integers$, único
              \stopitem
            \stopitemize
          \stopitem

          \startitem
            \obj{Leyes conmutativas}
            \startitemize[a]
              \startitem
                (\obj{suma}) \quad $a + b = b + a$
              \stopitem
              \startitem
                (\obj{multiplicación}) \quad $a \cdot b = b \cdot a$
              \stopitem
            \stopitemize
          \stopitem

          \startitem
            \obj{Leyes asociativas}
            \startitemize[a]
              \startitem
                (\obj{suma}) \quad $(a + b) + c = a (b + c)$
              \stopitem
              \startitem
                (\obj{multiplicación}) \quad $(a \cdot b) \cdot c = a \cdot (b \cdot c)$             
              \stopitem
            \stopitemize
          \stopitem

          \startitem
            \obj{Existencia de identidades}
            \startitemize[a]
              \startitem
                (\obj{suma}) \quad $0 + a = a + 0 = a$
              \stopitem
              \startitem
                (\obj{multiplicación}) \quad $1 \cdot a = a \cdot 1 = a$
              \stopitem
            \stopitemize
          \stopitem

          \startitem
            \obj{Existencia de inversos de la suma} \quad $ a + (-a) = -a + a = 0$
          \stopitem

          \startitem
            \obj{Leyes distributivas}
          \stopitem

          \startitemize[a]
            \startitem
              (\obj{izquierda}) \quad $a \cdot (b + c) = ab + ac$
            \stopitem
            \startitem
              (\obj{derecha}) \quad $(a + b) \cdot c = ac + bc$
            \stopitem
          \stopitemize

        \stopitemizer
      \stopaxioma

      \startaxiomaextra{\tf continuación}
        \margindata[youtube][method=height]{\from[AE36A]}
        \startitemizer[start=7]
          \startitem
            (\obj{Ley de suma de la igualad}) \quad $a = b --> a + c = {b + c} \land {c + a = c + b}$
          \stopitem
          \startitem
            (\obj{Ley de resta de la igualdad}) \quad $a = b --> a - c = b - c$
          \stopitem
          \startitem
            (\obj{Ley de multiplicación de la igualdad}) \quad $a = b --> {ac = bc} \land {ca = cb}$
          \stopitem
          \startitem
            (\obj{Ley de división de la igualdad}) \quad $a = b \land c \neq 0 --> \dfrac{a}{c} = \dfrac{b}{c}$
          \stopitem
          \startitem
            (\obj{Propiedad multiplicativa del cero}) \quad $a \cdot 0 = 0 \cdot a = 0$
          \stopitem
          \startitem
            (\obj{División de iguales}) \quad $a \neq 0 --> \dfrac{a}{a} = 1$
          \stopitem
          \startitem
            (\obj{División entre 1}) \quad $\dfrac{a}{1} = a$
          \stopitem
          \startitem
            (\obj{Resta de cero}) \quad $a - 0 = a$
          \stopitem
          \startitem
            (\obj{Resta de iguales}) \quad $a - a = 0$
          \stopitem
        \stopitemizer
    \stopaxiomaextra

      \startteorema
        Sea $x \in \integers$. Entonces $-x$ es único.
      \stopteorema

      \startdemo
        Suponemos que existe $e \in \integers$ que tiene la misma propieadad que $-x$. Esto es,
        \startformula
          x + e = e + x = 0
        \stopformula
        Queremos ver que $e = -x$.
        \comentario{
          \startitemize
            \startitem
              Ley de suma de la igualdad\par
              $a = b --> a + c = {b + c} \land {c + a = c + b}$
            \stopitem
            \startitem
              Ley asociativa (+)\par
              $(a + b) + c = a (b + c)$
            \stopitem
            \startitem
              Inversos de la suma\par
              $a + (-a) = 0$
            \stopitem
            \startitem
              Identidad de la suma\par
              $a + 0 = a$
            \stopitem
          \stopitemize
        }
        \startformula
          \startalign
            \NC e + x                     \NC = 0        \NR
            \NC e + x + (-x)              \NC = 0 + (-x) \NR
            \NC e + \left[x + (-x)\right] \NC = -x       \NR
            \NC e + 0                     \NC = -x       \NR
            \NC e                         \NC = -x\NR
          \stopalign
        \stopformula
      \stopdemo

      \startteorema
        Sean $a,b \in \integers$. Entonces
        \comentario{Este teorema permite establecer de otra forma la unicidad del 0 y del inverso aditivo.}
        \startitemizer
          \startitem
            $a + b = b + a = a --> b = 0$
          \stopitem
          \startitem
            $a + b = b + a = 0 --> a = -b \land b = -a$
          \stopitem
        \stopitemizer
      \stopteorema

      \startdemop
        $i)$
        \comentario{
          \startitemize
            \startitem
              Ley de suma de la igualdad\par
              $a = b --> a + c = {b + c} \land {c + a = c + b}$
            \stopitem
            \startitem
              Ley asociativa (+)\par
              $(a + b) + c = a (b + c)$
            \stopitem
            \startitem
              Inversos de la suma\par
              $a + (-a) = 0$
            \stopitem
            \startitem
              Identidad de la suma\par
              $a + 0 = a$
            \stopitem
          \stopitemize
        }
        \startformulas
          \startformula
            \startalign
              \NC a + b        \NC = a       \NR
              \NC -a + a + b   \NC = - a + a \NR
              \NC (-a + a) + b \NC = 0       \NR
              \NC 0 + b        \NC = 0       \NR
              \NC b            \NC = 0       \NR
            \stopalign
          \stopformula
          \startformula
            \startalign
              \NC b + a        \NC = a       \NR
              \NC b + a - a    \NC = a - a \NR
              \NC b + (a - a)  \NC = 0       \NR
              \NC b + 0        \NC = 0       \NR
              \NC b            \NC = 0       \NR
            \stopalign
          \stopformula
        \stopformulas

        $ii)$
        \startformulas
          \startformula
            \startalign
              \NC a + b          \NC = 0        \NR
              \NC -a + a + b     \NC = - a + 0  \NR
              \NC (-a + a) + b   \NC = -a       \NR
              \NC 0 + b          \NC = -a       \NR
              \NC b              \NC = -a       \NR
            \stopalign
          \stopformula
          \startformula
            \startalign
              \NC b + a          \NC = 0        \NR
              \NC b + a (-a)     \NC = 0 + (-a) \NR
              \NC b + [a + (-a)] \NC = -a       \NR
              \NC b + 0          \NC = -a       \NR
              \NC b              \NC = -a       \NR
            \stopalign
          \stopformula
        \stopformulas

      \stopdemop

      \startteorema
        \margindata[youtube]{\from[AE36B]}
        Sean $a,b \in \integers$. Entonces $-(a + b) = - a + (- b)$
      \stopteorema

      \startdemo
        Este teorema establece que el negativo de $a + b$ es la suma de los negativos de $a$ y $b$. Luego, bastará ver que
        \comentario{Existencia de inversos de la suma \par $ a + (-a) = -a + a = 0$}
        \startformula
          (a + b) + [-(a + b)] = (a + b) + [-a + (-b)] = 0
        \stopformula
        \comentario{
          \startitemize
            \startitem
              Ley asociativa general
            \stopitem
            \startitem
              Ley conmutativa general
            \stopitem
            \startitem
              Elemento identidad (+)\par
              $a + 0 = a$
            \stopitem
          \stopitemize
        }
        Pero,
        \startformula
          \startalign
            \NC (a + b) + [-a + (-b)] \NC = a + b + (-a) + (-b)     \NR
            \NC                       \NC = a + (-a) + b + (-b)     \NR
            \NC                       \NC = [a + (-a)] + [b + (-b)] \NR
            \NC                       \NC = 0 + 0 = 0               \NR
          \stopalign
        \stopformula
      \stopdemo

      \startejemplos
        \startitemize[n]
          \startitem
            $\ini{(-3) + (-4)} = -(3 + 4) = -7$
          \stopitem
          \startitem
            \starttabulate[|r|][before=]
              \NC $\ini{-20}$ \NR
              \NC $\ini{-16}$ \NR
              \NC $\overbar{-36}$ \NR
            \stoptabulate
          \stopitem
          \startitem
            $\ini{-12 + (-16) + (-245)} = -28 + (-245) = -(28 + 245) = -273$
          \stopitem
          \startitem
            \starttabulate[|r|][before=]
              \NC $\ini{-25}$ \NR
              \NC $\ini{-136}$ \NR
              \NC $\ini{-205}$ \NR
              \NC $\ini{-79}$ \NR
              \NC $\overbar{-445}$ \NR
            \stoptabulate
          \stopitem
        \stopitemize
      \stopejemplos

      \startteorema
        Sean $a,b \in \integers$. Entonces
        \startitemizer
          \startitem
            $(-1) a = -a$
          \stopitem
          \startitem
            $(-a)b = a(-b) = -ab$
          \stopitem
          \startitem
            $(-a)(-b)= ab$
          \stopitem
        \stopitemizer
      \stopteorema

      \startdemop
        $i)\quad$ Partimos de

        \comentario{%
          \startitemize
            \startitem
              Existencia de invesos (+)\par
              $x + (-x) = 0$
            \stopitem
            \startitem
              Ley de la multiplicación de la igualdad\par
              $x = y --> zx = zy $
            \stopitem
            \startitem
              Ley distributiva por la izquierda\par
              $x(y+z) = xy + xz$
            \stopitem
            \startitem
              $x + y = 0 --> y = -x$
            \stopitem
            \startitem
              Ley conmutativa (\times)
            \stopitem
          \stopitemize
        }
        \startformula
          \startalign
            \NC 1 + (-1)\NC = 0 \NR
            \NC a [1 + (-1)]\NC = a \cdot 0 \NR
            \NC a \cdot 1 + a (-1)\NC = 0 \NR
            \NC a + a(-1)\NC = 0 \NR
            \NC a(-1) \NC = -a \NR
            \NC (-1)a \NC = -a \NR
          \stopalign
        \stopformula

        $ii)\quad$ Que como ejercicio. Se ha de forma parecida al caso $iii)$.

        \comentario{
          \startitemize
            \startitem
              $(-1) a = -a$
            \stopitem
            \startitem
              Ley asociativa general (\times)
            \stopitem
            \startitem
              Ley conmutativa general (\times)
            \stopitem
            \startitem
              $-(-n) = n$
            \stopitem
          \stopitemize
        }
        \margindata[youtube]{\from[AE37A]}
        $iii)\quad$ Partimos de
        \startformula
          \startalign
            \NC (-a) (-b) \NC = [(-1)a][(-1)b]\NR
            \NC \NC = (-1) a (-1) b \NR
            \NC \NC = (-1)(-1)ab\NR
            \NC \NC = [(-1)(-1)]ab\NR
            \NC \NC = [-(-1)]ab\NR
            \NC \NC = 1 \cdot ab = ab\NR
          \stopalign
        \stopformula
      \stopdemop

      \startejemplos
        \startitemize[n]
          \startitem
            $\ini{(-2) 3} = -2 \cdot 3 = 6 $
          \stopitem
          \startitem
            $\ini{16 (-5)} = -16 \cdot 5 = -90$
          \stopitem
          \startitem
            \starttabulate[|rTm|][before=]
              \NC \ini{-109}  \NR
              \NC \ini{-28}   \NR
              \HL
              \NC 872         \NR
              \NC {213\ \ }   \NR
              \HL
              \NC {3052}     \NR
            \stoptabulate
          \stopitem
          \startitem
            \starttabulate[|rTm|][before=]
              \NC \ini{-71}  \NR
              \NC \ini{-22}   \NR
              \HL
              \NC 142         \NR
              \NC {142\ \ }   \NR
              \HL
              \NC {1562}     \NR
            \stoptabulate
          \stopitem
          \startitem
            $\ini{(-12)(-5)\,8\,(-4)} = [(-12)(-5)]\,[8\,(-4)] = 60\,[-32] = -1920$
          \stopitem
          \startitem
            $\ini{26(-2)(-13)(-1)(-30)}$ \par $ = [26\,(-2)]\,[-13\,(-1)](-30) = (-52)\,13\,(-30) = (-52)\,[13\, (-30)] = -52\,[-390]= 20280$
          \stopitem
        \stopitemize
      \stopejemplos
    \stopsection

    \startsection[title={Operación de resta de los enteros}]
      
      En los números cardinales habíamos definido la resta así, $a - b = x$ ssi $a = b + x = x + b$. Esta misma definición la aplicamos a la resta de los números enteros.

      Así,
      \startformula
        \startalign
          \NC (-3) + (-5) \NC = 8  \NR
          \NC -8 - (-3)   \NC = -5 \NR
          \NC -8 - (-5)   \NC = -3 \NR
        \stopalign
      \stopformula

      \startteorema{equivalencia de la resta}
        Sean $a,b \in \integers$, entonces $a - b = a + (-b)$.
      \stopteorema

      \startdemo
        \margindata[youtube]{\from[AE37B]}
        Llamemos
        \startformula
          a - b  = x
        \stopformula
        Entonces,
        \comentario{Definición de resta\par$a - b = x --> a = x + b$}
        \startformula
          a  = x + b 
        \stopformula
        Luego,
        \comentario{
          \startitemize
            \startitem
              Ley de suma de la igualdad\par
              $a = b --> a + c = {b + c} \land {c + a = c + b}$
            \stopitem
            \startitem
              Ley asociativa (+)\par
              $(a + b) + c = a (b + c)$
            \stopitem
            \startitem
              Inversos de la suma\par
              $a + (-a) = 0$
            \stopitem
            \startitem
              Identidad de la suma\par
              $a + 0 = a$
            \stopitem
            \startitem
              Ley de sustitución\par
             $a - b = x$
            \stopitem
            \startitem
              Ley simétrica de la igualdad\par
              $x = y --> y = x$
            \stopitem
          \stopitemize
        }
        \startformula
          \startalign
            \NC a + (-b) \NC = x + b + (-b)   \NR
            \NC a + (-b) \NC = x + [b + (-b)] \NR
            \NC a + (-b) \NC = x + 0          \NR
            \NC a + (-b) \NC = x              \NR
            \NC a + (-b) \NC = a - b          \NR
            \NC a - b\NC = a + (-b)           \NR
          \stopalign
        \stopformula
      \stopdemo
    \stopsection

    \startejemplos
      \startitemejem
        \startitem
          $\ini{5 - 2} = 5 + (-2)$
        \stopitem
        \startitem
          $\ini{-6 -8} = -6 + (-8)$
        \stopitem
        \startitem
          $\ini{13 - (-9)} = 13 + [-(-9)]$
        \stopitem
        \startitem
          $\ini{-7-(-318)} = -7 +[-(-318)]$
        \stopitem
        \startitem
          Sabemos que $5 - 2 = 4$. Luego $5 + (-2) = 3$
        \stopitem
        \startitem
          Sabemos que $-6 + (-8) = -14$. Por lo tanto, $-6 -8 = -14$
        \stopitem
      \stopitemejem
    \stopejemplos

    Recordemos que los números enteros positivos $\integers^{+}$ cumplen las siguientes propiedades:

    \startitemize
      \startitem
        ley de clausura (+)
      \stopitem
      \startitem
        ley clausura (\times)
      \stopitem
      \startitem
        ley de tricotomia: Para $a \in \integers$ ocurre \par
        $a = 0 \veebar a \in \integers^{+} \veebar -a \in \integers^{+}$
      \stopitem
      \startitem
        $a \neq 0 \land a \nin \integers^{+}, a$ es un entero negativo
      \stopitem
      \startitem
        $a > b$ ssi $a -b \in \integers^{+}$
      \stopitem
      \startitem
        $b < a$ ssi $b > a$
      \stopitem
      \startitem
        $a > 0 <--> a $ es entero positivo
      \stopitem
      \startitem
        0 no es entero positivo
      \stopitem
    \stopitemize

    \startobservacion
      \startitemize
        \startitem
          Vemos que si $a < 0$, entonces $0 > a$. Luego, $0 - a \in \integers^{+}$. O sea, $0 + (-a) \in \integers^{+}$. Es decir, $-a \in \integers^{+}$. Por lo tanto, $a$ es un entero negativo.

          $ a < 0 <--> a$ es entero negativo.
        \stopitem

        \margindata[youtube]{\from[AE38A]}
        \startitem
          Como $0 = 0$, entonces, por la ley de tricotomía, $0 \nin \integers^{+}$ y el negativo de cero tampoco es un entero positivo, $-0 = 0 \nin \integers^{+}$.

          0 no es un entero negativo.
        \stopitem
      \stopitemize
    \stopobservacion

    \startteorema{leyes distributivas de la multiplicación de números enteros sobre la resta de números enteros}
      Sean $x,y,z \in \integers$
      \startitemizer
        \startitem
          \obj{(de la izquierda)} $x(y-z) = xy - xz$
        \stopitem
        \startitem
          \obj{(de la derecha)} $(x-y)z = xz - yz$
        \stopitem
      \stopitemizer
    \stopteorema
    \comentario{%
      \startitemize
        \startitem
          $a - b = a + (-b)$
        \stopitem
        \startitem
          Distributiva por la izquierda / derecha
        \stopitem
        \startitem
          $a (-b) = -ab$
        \stopitem
      \stopitemize
    }
    \startdemop
      \startformulas
        \startformula
          \startalign
            \NC i)\quad x(y-z) \NC = x(y+(-z))\NR
            \NC \NC = xy + x(-z)\NR
            \NC \NC = xy - xz\NR
          \stopalign
        \stopformula

        \startformula
          \startalign
            \NC ii)\quad (x-y)z \NC = (x+(-y))z\NR
            \NC \NC = xz - (y)z)\NR
            \NC \NC = xz - yz\NR
          \stopalign
        \stopformula
      \stopformulas
    \stopdemop

    \startteorema{propiedades fundamentales de las desigualdades}
      Sean $a,b,c \in \integers$. Entonces
      \startitemizer
        \startitem
          \obj{(ley de tricotomía)} Una y sólo una de las siguientes afirmaciones es cierta:
          \startformula
            a = b  \;\veebar\; a > b \;\veebar\; a < b.
          \stopformula
        \stopitem
        \startitem
          \obj{(ley transitiva)} \quad $a > b \;\land\; b > c --> a > c$.
        \stopitem
        \startitem
          \obj{(ley de suma)} \quad $a > b --> a + c > b + c \;\land\; c + a > c + b$.
        \stopitem
        \startitem
          \obj{(leyes de multiplicación)}\quad $a > b \;\land$
          \startitemize[a]
            \startitem
              \obj{(positiva)} \quad $c > 0 --> ac > b c \land ca > cb$.
            \stopitem
            \startitem
              \obj{(negativa)} \quad $c < 0 --> ac < bc \land ca < cb$.
            \stopitem
          \stopitemize
        \stopitem
      \stopitemizer
    \stopteorema

    \startdemop
      \youtube{\from[AE38B]}
      \comentario{%
        Ley de equivalencia de la resta\par
        $x -y = x +(-y) \in \integers$
      }
      $i)\quad$ Consideremos el número $a - b$, por la ley de equivalencia de la resta tenemos que

      \startformula
        a - b = a + (-b) \in \integers
      \stopformula

      \comentario{%
        Ley de tricotomía de $\integers$\par
        $\forall a \in \integers -> a = 0 \veebar a \in \integers^{+} \veebar -a \in \integers^{+}$
      }
      La ley de tricotomía de $\integers$ nos dice que sólo una de las siguientes posibilidades puede ocurrir:
      \startitemizer[a,packed][style=\sl,right=$)$]
        \startitem
          $a - b = 0$
        \stopitem
        \startitem
          $a - b \in \integers^{+}$
        \stopitem
        \startitem
          $-(a - b) \in \integers^{+}$
        \stopitem
      \stopitemizer

      \blank[2*big]
      \startitemize
        \startitem
          Si ocurre $a)$
          \startformula
            \startalign
              \NC a - b + b      \NC = 0 + b \NR
              \NC a + (-b) + b   \NC = b     \NR
              \NC a + [(-b) + b] \NC = b     \NR
              \NC a + 0          \NC = b     \NR
            \stopalign
          \stopformula
          \startformula
            \quad\quad \therefore \quad \graymath{a = b}
          \stopformula
        \stopitem

        \startitem
          \comentario{$a > b <--> a -b \in \integers^{+}$}
          Si ocurre $b)$
          \startformula
            a - b \in \integers^{+} \text{ ssi }\; \graymath{a > b}
          \stopformula
        \stopitem
      
        \comentario{%
          \startitemize[1]
            \startitem
              $-(x + y) = -x + (-y)$\par
            \stopitem
            \startitem
              $x - y = x + (-y) \in \integers$\par
            \stopitem
            \startitem
              $a > b <--> a - b \in \integers^{+}$\par
            \stopitem
            \startitem
              $a < b <--> b > a$
            \stopitem
          \stopitemize
        }
        \startitem
          Si ocurre $c)$
          \startformula
            \startalign
              \NC -(a - b) \NC = -(a + (-b))  \NR
              \NC          \NC = -a + (-(-b)) \NR
              \NC          \NC = -a + b       \NR
              \NC          \NC = b + (-a)     \NR
              \NC          \NC = b - a \in \integers^{+} \text{ ssi } b > a \text{ ssi }\; \graymath{a < b} \NR
            \stopalign
          \stopformula
        \stopitem

      \stopitemize

      $ii)\quad$ Por hipótesis, $a > b \land b > c$. Luego,
      \comentario{$a > b <-->  a - b \in \integers^{+}$}
      \startformula
        \startalign[n=1]
          \NC a > b <--> a - b \in \integers^{+} \NR
          \NC b > c <--> b - c \in \integers^{+} \NR
        \stopalign
      \stopformula
      \comentario{%
        \startitemize
          \startitem
            Ley de clausura (+)\par
          \stopitem
          \startitem
            Ley asociativa general (+)\par
          \stopitem
          \startitem
            $a > b <--> a - b \in \integers^{+}$
          \stopitem
        \stopitemize
      }
      Por tanto,
      \startformula
        \startalign[n=1]
          \NC  (a - b) + (b - c) \in \integers^{+} \NR
          \NC  a + (-b) + b - c \in \integers^{+} \NR
          \NC  a + [(-b) + b] -c \in \integers^{+} \NR
          \NC  a + 0 - c \in \integers^{+} \NR
          \NC  (a + 0) - c \in \integers^{+} \NR
          \NC  a - c \in \integers^{+} \NR
          \NC  a > c \NR
        \stopalign
      \stopformula

      \youtube{\from[AE39A]}

      $iii)\quad$ Por hipótesis, $a > b$. Luego,
      \comentario{%
        \startitemize
          \startitem
            $x > y <--> x - y \in \integers^{+}$
          \stopitem
          \startitem
            $x - y = x + (-y)$
          \stopitem
          \startitem
            $-(x + y) = (-x) + (-y)$
          \stopitem
        \stopitemize
      }
      \placeformula
      \startformula
        a - b \in \integers^{+}
      \stopformula
      Pero,
      \startformula
        \startalign
          \NC a - b \NC = (a + 0) - b \NR
          \NC \NC = a + 0 - b\NR
          \NC \NC = a + [c + (-c)] - b \NR
          \NC \NC = a + c + (-c) -b \NR
          \NC \NC = a + c + (-c) + (-b) \NR
          \NC \NC = a + c + (-b) + (-c) \NR
          \NC \NC = a + c + [(-b) + (-c)] \NR
          \NC \NC = a + c + [-(b + c)] \NR
          \NC \NC = (a + c) + [-(b + c)] \NR
          \NC \NC = (a + c) - (b + c) \NR
        \stopalign
      \stopformula
      O sea, $a - b = (a + c) - (b + c)$. Sustituyendo en $(1)$, tenemos que $(a + c) - (b + c) \in \integers^{+}$. Por lo tanto, $a + c > b + c$, que se puede reescribir como $c + a > c + b$.

      $iv)\quad b)\quad$ Por hipótesis,
      \placeformula
      \startformula
        a > b
      \stopformula
      y $c < 0$, lo que equivale a decir que $0 > c$. Entonces, $0 + (-c) > c + (-c)$, que se simplifica a $-c > 0$. Entonces, por la ley de multiplicación positiva, y $(2)$,
      \startformula
        \startalign
          \NC a (-c) \NC > b (-c) \NR
          \NC -ac \NC > - bc \NR
        \stopalign
      \stopformula
      \youtube{\from[AE39B]}

      Por lo tanto, $-ac - (-bc) \in \integers^{+}$ y esto se convierte en:
      \comentario{%
        \startitemize
          \startitem
            $x -y = x + (-y)$
          \stopitem
          \startitem
            $-(-n) = n$
          \stopitem
          \startitem
            $x > y <--> x - y \in \integers^{+}$
          \stopitem
        \stopitemize
      }
      \startformula
        \startalign[n=1]
          \NC -ac + [-(-bc)] \in \integers^{+} \NR
          \NC -ac + bc \in \integers^{+} \NR
          \NC bc + (-ac) \in \integers^{+}\NR
          \NC bc - ac \in \integers^{+} \NR
          \NC bc > ac \NR
          \NC ac < bc. \NR
        \stopalign
      \stopformula

      $iv)\quad\ a)\quad$ Se de muestra de forma similar al anterior.
    \stopdemop

    \startteorema
      Sea $a \in \integers^{+}$ con $a > 0$. Entonces $-a < 0$.
    \stopteorema

    \startdemo
      Por hipótesis, $a > 0$. Entonces, por la ley de suma de las desigualdades
      \startformula
        a + (-a) > 0 + (-a)
      \stopformula
      Es decir, $0 > -a$, que equivale a $-a < 0$.
    \stopdemo

    \startteorema
      $\forall n \in \naturalnumbers,\; n >0$.
    \stopteorema

    \startdemo
      Bastará ver que $1 > 0$. Lo haremos por contradicción.

      Suponemos que $\graymath{1 < 0}$. Esto es equivalente a decir que
      \comentario{$a > b, c < 0 --> ca < cb$}
      \startformula
        \startalign[n=1]
          \NC 0 > 1 \NR
          \NC 1 \cdot 0 < 1 \cdot 1 \NR
          \NC \graymath{0 < 1} \NR
        \stopalign
      \stopformula
      \comentario{$a = b \,\veebar\, a > b \,\veebar\, a < b$}
      Esto contradice la ley de tricotomía.

      Ahora, que tenemos que $1 > 0$ y podemos hacer lo siguiente
      \comentario{%
        \startitemize
          \startitem
            $a > b --> a+c > b+c$
          \stopitem
          \startitem
            $a > b \land b > c --> a > c$
          \stopitem
        \stopitemize
      } 
      \startformula
        \startalign[n=1]
          \NC 1 > 0 \NR
          \NC 1 + 1 > 0 + 1 \NR
          \NC 2 > 1 \NR
          \NC \therefore 2 > 0 \NR
          \startintertext
            Si ahora hacemos
          \stopintertext
          \NC 2 + 1 > 0 + 1 \NR
          \NC 3 > 1 \NR
          \NC \therefore 3 > 0 \NR
        \stopalign
      \stopformula
      y así se podría seguir para todos los números naturales.
    \stopdemo

    \youtube{\from[AE40A]}
    Hemos visto en el teorema anterior que todos los números naturales van a ser mayores que cero. Por lo tanto, $\naturalnumbers = \{1,2,3,\dots\}$ está compuesto por enteros positivos.

    Recordamos que el conjunto $\{0\}$ contiene un entero que no es entero positivo ni entero negativo.

    Además, vimos un teorema que establecía que los números mayores que cero, (enteros positivos) tenían su inverso aditivo menor que 0 (enteros negativos):  $a > 0 --> -a < 0$.

    Por lo tanto, el conjunto compuesto por los números negativos de los naturales $\{-1, -2, -3, \dots \}$ va a estar compuesto por enteros negativos.

    Pero, $\integers = \underbrace{\{-1, -2, -3, \dots \}}_{\text{enteros negativos}} \cup\; \{ 0 \} \;\cup \underbrace{\{ 1, 2, 3, \dots \}}_{\text{enteros negativos}} ={\{ \dots, -3, -2, -1, 0, 1, 2, 3, \dots \}}$.

    Por tanto, $\naturalnumbers = \integers^{+}$ y $\{-1, -2, -3, \dots \} = \integers^{-}$, es el conjunto de enteros negativos.

    \startobservacion
      Todas las propiedades relacionadas a desigualdades en que aparezca el signo $>$, son ciertas aunque aparezcan los signos $<, \geq$ o $\leq$ en su lugar. Excepto en las leyes de multiplicación de $c > 0$ en la positiva y $c < 0$ en la negativa. Así,
      \startformula
        a < b --> a + c < b + c \land c + a < c + b
      \stopformula
      \startformula
        a < 0 --> -a > 0
      \stopformula
      \startformula
        a < b \land c > 0 --> ac < bc \land ca < cb
      \stopformula
      \startformula
        a < b \land c < 0 --> ac > bc \land ca > cb
      \stopformula
    \stopobservacion
    
    \startdefinicion
      Sea $a \in \integers$. El \obj{valor absoluto de $a$}, representado por el símbolo $\abs{a}$, se define como:
      \startformula
        \abs{a} = 
        \startcases
          \MC a, \MC \text{si } a \geq 0 \NR
          \MC -a, \MC \text{si } a < 0 \NR
        \stopcases
      \stopformula
    \stopdefinicion

    \startejemplos
      \startitemejem
        \startitem
          $\abs{2} = 2$, ya que $2 > 0$.
        \stopitem
        \startitem
          $\abs{-3} = -(-3) = 3$, ya que $-3 < 0$.
        \stopitem
        \startitem
          $-\abs{6} = -6$.
        \stopitem
        \startitem
          $-\abs{-12} = -[-(12)] = -12$
        \stopitem
        \startitem
          $\abs{0} = 0$
        \stopitem
      \stopitemejem
    \stopejemplos

    \youtube{\from[AE40B]}
    \startteorema
      Sean $a,b \in \integers^{+} = \naturalnumbers$. Esto es, $a,b > 0$. Entonces,
      \startitemizer
        \startitem
          $a + b > 0$ y $\abs{a + b} = \abs{a} + \abs{b}$.
        \stopitem
        \startitem
          $-a - b < 0$ y $\abs{-a -b} = \abs{a} + \abs{b}$.
        \stopitem
        \startitem
          Si $a > b$, entonces $a + (-b) > 0$ y $\abs{a + (-b)} = \abs{a} - \abs{b}$.
        \stopitem
        \startitem
          Si $a < b$, entonces $a + (-b) < 0$ y $\abs{a + (-b)} = \abs{b} - \abs{a}$
        \stopitem
      \stopitemizer
    \stopteorema

    \startdemop
      $i)\quad$ Como $a,b \in \integers^{+}$, entonces $a + b \in \integers^{+}$ (ley clausura (+) en $\integers^{+}$). Luego, $\graymath{a + b > 0}$. Por lo tanto, $\abs{a + b} = a + b$. Como $a,b > 0$, $\abs{a} = a$  y $\abs{b} = b$. Luego, por sustitución, $\graymath{\abs{a + b} = \abs{a} + \abs{b}}$.

      \comentario{%
        \startitemize[joinedup]
          \startitem
            Equivalencia de la resta\par$x-y = x+ (-y)$
          \stopitem
          \startitem
            $-x + (-y) = -(x+y)$
          \stopitem
          \startitem
            $ a > 0 --> -a < 0'$
          \stopitem
          \startitem
            $-(-n) = n$
          \stopitem
        \stopitemize
        }
      $ii)\quad$ $-a  - b = -a + (-b) = -(a + b)$. Vimos en el apartado $i)$ que $a + b > 0$. Luego, $-(a + b) = 0$. Por lo tanto $\graymath{-a - b < 0}$, al hacer la sustitución.

      Entonces $\abs{-a -b} = - (-a -b) = [-a + (-b)] = -(-a) + [(-b)] = a + b$.
      \startformula
        \therefore \graymath{\abs{-a - b} = \abs{a} + \abs{b}}
      \stopformula

      \comentario{$x > y --> x + z > y + z$}
      $iii)\quad$ Como $a > b$, entonces 
      \startformula
        a + (-b) > b + (-b) 
      \stopformula
      \startformula
        \graymath{a + (-b) > 0}
      \stopformula

      Vemos ahora que
      \startformula
        \abs{a + (-b)} = a + (-b) = a - b = \abs{a} - \abs{b}
      \stopformula
      \startformula
        \therefore \graymath{\abs{a + (-b)} = \abs{a} - \abs{b}}
      \stopformula

      \youtube{\from[AE41A]}
      $iv)\quad$ Por hipótesis, $a < b$. Entonces, por la ley de suma de las desigualdades,
      \startformula
        a + (-b) < b + (-b)
      \stopformula
      \startformula
        \graymath{a + (-b) < 0}
      \stopformula
      Entonces,
      \startformula
        \abs{a + (-b)} = -[a + (-b)] = -a + [-(-b)] = -a +b = b + (-a) = b - a
      \stopformula
      Como $\abs{a} = a $ y $\abs{b} = b$, pues, por hipótesis, $a,b > 0$, al sustituir y aplicando la ley transitiva de la igualdad, obtenemos que
      \startformula
        \graymath{\abs{a + (-b)} = \abs{b} - \abs{a}}
      \stopformula

    \stopdemop


    \comentario{%
      \startitemizer
        \startitem
          $a + b > 0 \land \abs{a + b} = \abs{a} + \abs{b}$
        \stopitem
        \startitem
          $-a - b < 0 \land \abs{- a - b} = \abs{a} + \abs{b}$
        \stopitem
        \startitem
          $a > b --> a + (-b) > 0 \land \abs{a + (-b)} = \abs{a - b}$
        \stopitem
        \startitem
          $a < b --> a + (-b) < 0 \land \abs{a + (-b)} = \abs{b - a}$
        \stopitem
      \stopitemizer
      \blank
      Equivalencia de la resta\par$a - b = a + (-b)$
      \blank
      $a,b > 0\; (a,b \in \naturalnumbers)$
    \startitemizer
      \startitem
        a \cdot b > 0
      \stopitem
      \startitem
        (-a) (-b) = a \cdot b > 0
      \stopitem
      \startitem
        a(-b) = (-a) b = -ab < 0
      \stopitem
    \stopitemizer
    }
    \startdiscusion{Resumiendo}
      \startitemize[n, unpacked]
        \startitem
          Si sumamos numeros positivos, el resultaod es postivo, si sumamos números negativos, el resultado es negativo, y lo que hay que hacer es sumar los valores absoluto de esos números.
        \stopitem
        \startitem
          Si sumamos números enteros con diferente signo, el resultado lleva el signo del de mayor valor absoluto y se resta el de mayor valor absoluto menos el de menos valor absoluto.
        \stopitem
        \youtube{\from[AE41B]}
        \startitem
          Para la resta se aplican las mismas reglas de la suma.
        \stopitem
        \startitem
          Al multiplicar dos números enteros con el mismo signo, el producto será postivo.
        \stopitem
        \startitem
          Al multiplicar dos números enteros con diferente signo, el producto será negativo.
        \stopitem
      \stopitemize
    \stopdiscusion

    \startejemplos
      \startitemejem
        \startitem
          \ini{-16 + 4} = -(16-4) = 12
        \stopitem
        \startitem
          \ini{28 + (-17)} = 28 -17 = 11
        \stopitem
        \startitem
          \ini{305 + (-458)} =-(458-305) = -153
        \stopitem
        \startitem
          \ini{-72 + 239} = 239 -72 = 167
        \stopitem
        \startitem
          \ini{46 + 27 + (-13) + (-31)} = (46 + 27) + [(-13)+ (31)] = 73 + [-(13 + 31)] = 73 + [-44] = 73 - 44 = 29
        \stopitem
        \startitem
          En forma vertical
          \starttabulate[|r|][before=]
            \NC $\ini{254}$  \NR
            \NC $\ini{-329}$ \NR
            \NC $\ini{-645}$ \NR
            \NC $\ini{432}$  \NR
            \NC $\overbar{-288}$ \NR
          \stoptabulate
        \stopitem
        \startitem
          En forma vertical
          \starttabulate[|r|][before=]
            \NC $\ini{-35}$  \NR
            \NC $\ini{748}$ \NR
            \NC $\ini{229}$ \NR
            \NC $\ini{-96}$  \NR
            \NC $\ini{-235}$  \NR
            \NC $\overbar{611}$ \NR
          \stoptabulate
        \stopitem
        \startitem
          Efectúe las siguientes restas:
          \startitemize[a][stopper=)]
            \startitem
              \ini{13 - (-25)} = 13 + 25 = 38
            \stopitem
            \startitem
              \ini{132 - 141} = 132 + (-141) = -9
            \stopitem
            \startitem
              \starttabulate[|r|][before=]
                \NC $\ini{-725}$  \NR
                \NC $\ini{-(-648)}$  \NR
                \NC $\overbar{\quad\quad -77}$ \NR
              \stoptabulate
            \stopitem
          \stopitemize
        \stopitem
      \stopitemejem
    \stopejemplos

    \youtube{\from[AE42A]}
    % \definemathmatrix[Bmatrix] [left={\left\{\,},right={\,\right\}}]
    \definemathmatrix[Bmatrix] [right={\,\Bigg\}}]
    \comentario{Reglas de signos
      \setupformulas[align=right]
      \startformula
        (+) + (+) = +
      \stopformula
      \startformula
        (-) + (-) = -
      \stopformula
      \startformula
        \startBmatrix
          \NC (+) + (-) \NR
          \NC (-) + (+) \NR
        \stopBmatrix
        = \text{signo de mayor valor absoluto}
      \stopformula
      \startformula
        (+)(+) = (-)(-) = +
      \stopformula
      \startformula
        (+)(-) = (-)(+) = -
      \stopformula
    }

    \startejemplos
      Efectúe las siguientes restas
      \starttabulate[|c|r|c|r|c|r|c|r|][before=]
        \NC 1) \NC  1045 \NC 2) \NC  -528 \NC 3) \NC    49 \NC 4) \NC  229 \NR
        \NC    \NC -1032 \NC    \NC -1024 \NC    \NC - -75 \NC    \NC -178 \NR
        \NC    \NC $\overbar{\;\quad -13}$
        \NC    \NC $\overbar{\;\quad -1552}$
        \NC    \NC $\overbar{\;\quad 124}$
        \NC    \NC $\overbar{\;\quad 51}$ \NR
      \stoptabulate

      \starttabulate[|c|r|c|r|c|r|][before=]
        \NC 5) \NC   -741 \NC 6) \NC    -528 \NC 7) \NC  223 \NR
        \NC    \NC - -372 \NC    \NC - -1021 \NC    \NC -572 \NR
        \NC    \NC $\overbar{\;\quad -369}$
        \NC    \NC $\overbar{\;\quad 276}$
        \NC    \NC $\overbar{\;\quad -349}$ \NR
      \stoptabulate
    \stopejemplos

    \youtube{\from[AE42B]}
    \startsection[title={División de números enteros}]
      Aplicamos la misma definición de la división en $\mathbb{W}$ a la división en $\integers$. Esto es: $a,b \in \integers$, entonces $a \div b = x$ ssi $a = xb = bx$, siempre que $b \neq 0$.
      \startejemplos
        \startitemejem
          \startitem
            Como $5(-3) = -15$, entonces $-15 - 5 = -3 \,\land\, -15 \div (-3) = 5$
          \stopitem
          \startitem
            Como $(-12)(-10) = 120$, entonces $120 \div (-12) = -10 \,\land\, 120 \div (-10) = -12$
          \stopitem
        \stopitemejem
      \stopejemplos

      \startteorema
        Sean $a,b \in \integers\, b \neq 0$, de modo que las siguientes divisiones están definicidas (tienen resultado en $\integers$). Luego,
        \startitemize[n, stopper=)]
          \startitem
          Si $a > 0 \land b < 0 --> a \div b < 0$ 
        \stopitem
        \startitem
          Si $a < 0 \land b > 0 --> a \div b < 0$
        \stopitem
        \startitem
          Si $a,b < 0 --> a \div b > 0$
        \stopitem
        \stopitemize
      \stopteorema

      \startdemop
        $i)\quad$ Por hipótesis $a \div b = x \in \integers$. Luego, por la definiión de división:
        \placeformula
        \startformula[+]
          a = b x
        \stopformula
        Por hipótesis $b<0$. En consecuencia $b$ será el negativo de algún número natural, esto es $b = -c$, esto es $b = -c,\, c \in \naturalnumbers$.

        Para demostrar que $x < 0$, lo veremos por contradicción. Supongamos que $x > 0$, esto es, $x \in \naturalnumbers$. Entonces (1) se convierte, al sustituirlo por $b$, en
        \startformula
          a = (-c) x
        \stopformula
        \comentario{%
          \setupformula[align=right]
          \startformula
            (-a)b = -ab;\; (-a)(-b) = ab
          \stopformula
          \startformula
            a > 0 --> -a < 0
          \stopformula
        }
        Pero, por un teorema previo,
        \startformula
          (-c)x = -cx = a
        \stopformula
        donde $cx \in \naturalnumbers$ por la ley de clausura de la multiplicación en $\naturalnumbers$. Luego $cx > 0$ y $-cx = a < 0$. Llegamos a una contradiccón $(--><--)$, pues $a > 0$.

        $ii)\quad$ queda como ejercicio.

        $iii)\quad$ Por hipótesis, $a \div b \in \integers$. Luego, $a \div b = x \in \integers$, y, por definición de división,
        \placeformula
        \startformula[+]
          a = bx.
        \stopformula
        Como $b < 0$ (por hipótesis), $b = -c, \,c \in \naturalnumbers$. Para demostrar que $x > 0$, usaremos también cotradicción.

        Suponemos que $x < 0$. Entonces $x = -d \in \naturalnumbers$. Luego, al sustituir en (1), obtenemos:
        \startformula
          a = (-c)(-d) = cd,
        \stopformula
        y como $c,d \in \naturalnumbers$, entonces $cd \in \naturalnumbers$ (debido a la ley de cierre de la multiplicación en $\naturalnumbers$).

        Luego,
        \startformula
          cd = a \in \naturalnumbers
        \stopformula
        En consecuencia $a > 0$ $(--><--)$ Se llega a una contradicción, pues $a < 0$.
      \stopdemop

      \youtube{\from[AE43A]}
      \startdiscusion{Resumiendo}
        \startitemize[n]
          \startitem
          Si dividimos dos enteros positivos o dos enteros negativos, el resultado es positivo.
        \stopitem
        \startitem
          Si dividimos dos números enteros con diferente signo, el resultado es negativo.
        \stopitem
        \stopitemize

        \startejemplos
          \startitemejem
            \startitem
              $\ini{\dfrac{-28}{4}} = -7$
            \stopitem
            \startitem
              $\ini{-56/(-8)} = 7$
            \stopitem
            \startitem
              $\ini{36 \div (-6)} = -6$
            \stopitem
            \startitem
              $\ini{-3\overline{\smash{}{)-24}}} = 8$%$
            \stopitem
          \stopitemejem
        \stopejemplos

        \comentario{Los signos de valor absoluto se consideran como si fueran signos de agrupación en la jerarquía de operaciones.}
        \startejemplos
          Evalúe
          \startplaceformula
            \startejerformula
              \startalign
                \NC   \NC \ini{-6 + 4 (-5) - (-14) -10 + 2 \cdot 9} \NR[+]
                \NC = \NC -6 + (-20) - (-14) -10 + 18               \NR
                \NC = \NC -26 + 14 + (-10) + 18                     \NR
                \NC = \NC -4                                        \NR
              \stopalign
            \stopejerformula
          \stopplaceformula

          \startplaceformula
            \startejerformula
              \startalign
                \NC   \NC \ini{6 + (-2)^3 - \dfrac{3 \times 4}{-2} + \abs{5(-4)}} \NR[+]
                \NC = \NC 6 + (-2)^3 - \dfrac{3 \times 4}{-2} + \abs{-20}         \NR
                \NC = \NC 6 + (-2)^3 - \dfrac{3 \times 4}{-2} + 20 \NR
                \NC = \NC 6 + (-8) - \dfrac{3 \times 4}{-2} + 20 \NR
                \NC = \NC 6 + (-8) - (-6) + 20 \NR
                \NC = \NC (-2) + 6 + 20  \NR
                \NC = \NC 24 \NR
              \stopalign
            \stopejerformula
          \stopplaceformula

          \startplaceformula
            \startejerformula
              \startalign
                \NC   \NC \ini{-12 + 5 \times(-25) -4^2 + 42/-7} \NR[+]
                \NC = \NC -12 + 5 \times(-25) -16 + 42/-7 \NR
                \NC = \NC -12 + (-125) -16 + (-6) \NR
                \NC = \NC  -137 + (-16) + (-6)\NR   
                \NC = \NC  159\NR
              \stopalign
            \stopejerformula
          \stopplaceformula

          \youtube[method=top]{\from[AE43B]}
          \startplaceformula
            \startejerformula
              \startalign
                \NC \NC \ini{2(-4 - 7) + \dfrac{5(-3)(-2)}{6} - \abs{2^2 -5 + (-1)^3}} \NR[+]
                \NC = \NC 2(-11) + \dfrac{5(-3)(-2)}{6} - \abs{4 -5 + (-1)} \NR
                \NC = \NC 2(-11) + \dfrac{5(-3)(-2)}{6} - \abs{4 + (-5) + (-1)} \NR
                \NC = \NC 2(-11) + \dfrac{5(-3)(-2)}{6} - \abs{-2} \NR
                \NC = \NC 2(-11) + \dfrac{5(-3)(-2)}{6} - 2 \NR
                \NC = \NC -22 + 5 + (-2) \NR
                \NC = \NC -19 \NR
              \stopalign
            \stopejerformula
          \stopplaceformula

          \startplaceformula
            \startejerformula
              \startalign
                \NC \NC \ini{-25 -\left[(-2)^2 - 3(5^2 -(-14)) -3\times(-5)\right] + 3^3} \NR[+]
                \NC = \NC -25 -\left[(-2)^2 - 3(25 -(-14)) -3\times(-5)\right] + 3^3 \NR
                \NC = \NC -25 -\left[(-2)^2 - 3(25 + 14) -3\times(-5)\right] + 3^3 \NR
                \NC = \NC -25 -\left[(-2)^2 - 3(39) -3\times(-5)\right] + 3^3 \NR
                \NC = \NC -25 - [4 - 3(39) -3\times(-5)] + 3^3 \NR
                \NC = \NC -25 - [4 - 117 - (-15)] + 3^3 \NR
                \NC = \NC -25 - [4 + (-117) + 15] + 3^3 \NR
                \NC = \NC -25 - [-98] + 3^3 \NR
                \NC = \NC -25 - [-98] + 27 \NR
                \NC = \NC -25 + 98 + 27 \NR
                \NC = \NC 100 \NR
              \stopalign
            \stopejerformula
          \stopplaceformula

          \ \youtube{\from[AE44A]}
          \startplaceformula
            \startejerformula
              \startalign
                \NC \NC \ini{-6 \cdot 4/3 + \left\{ -4^2 + 2\left[5 - 3^2 \left(9 - 2^2\right) - 6^2\right] +2(4)\right\} -(-5)^2} \NR[+]
                \NC = \NC -6 \cdot 4/3 + \left\{ -4^2 + 2\left[5 - 3^2 (9 - 4) - 6^2\right] +2(4)\right\} -(-5)^2 \NR
                \NC = \NC -6 \cdot 4/3 + \left\{ -4^2 + 2\left[5 - 3^2 (5) - 6^2\right] +2(4)\right\} -(-5)^2 \NR
                \NC = \NC -6 \cdot 4/3 + \left\{ -4^2 + 2 [5 - 9 (5) - 36] +2(4)\right\} -(-5)^2 \NR
                \NC = \NC -6 \cdot 4/3 + \left\{ -4^2 + 2[5 - 45 - 36] +2(4)\right\} -(-5)^2 \NR
                \NC = \NC -6 \cdot 4/3 + \left\{ -4^2 + 2[5 + (-45) + (-36)] +2(4)\right\} -(-5)^2 \NR
                \NC = \NC -6 \cdot 4/3 + \left\{ -4^2 + 2 [76] +2(4)\right\} -(-5)^2 \NR
                \NC = \NC -6 \cdot 4/3 + \{ -16 + 2 [-76] +2(4)\} -(-5)^2 \NR
                \NC = \NC -6 \cdot 4/3 + \{ -16 + [-152] + 8\} -(-5)^2 \NR
                \NC = \NC -6 \cdot 4/3 + \{ -160 \} -(-5)^2 \NR
                \NC = \NC -6 \cdot 4/3 + \{ -160 \} -(25) \NR
                \NC = \NC -8 + \{ -160 \} -(25) \NR
                \NC = \NC -8 + \{ -160 \} + (-25) \NR
                \NC = \NC -193 \NR
              \stopalign
            \stopejerformula
          \stopplaceformula
        \stopejemplos

        \youtube{\from[AE44B]}
        \comentario{%
            \startformula[align=right]
              x > y --> x - y \in \integers^{+}
            \stopformula
            \startformula[align=right]
              x < y --> y > x
            \stopformula
            \startformula[align=right]
              a > 0 --> -a < 0
            \stopformula
        }
        \startejemplos
          \startitemejem
            \startitem
              \ini{Demuestre que $3 < 5$}.

              $3 < 5$ ssi $5 > 3$ ssi $5 - 3 \in \integers^{+}$ ssi $2 \in \naturalnumbers$, que sabemos que es cierto.
            \stopitem

            \startitem
              \ini{Demuestre que $-8 > -15$}.

              $-8 > -15 <--> -8 -(-15) \in \integers^{+} <--> -8 + 15 \in \integers^{+} <--> 15 + (-8) \in \integers^{+} <--> 15 - 8 \in \integers^{+} <--> 7 \in \naturalnumbers$, que sabemos que es cierto.
            \stopitem
            \startitem
              \ini{Demuestre que si $x < 0$ y si $y > 0$, entonces $x < y$}.

              Como $y > 0$, entonces podemos decir que
              \placeformula
              \startformula
                0 < y
              \stopformula
              Por hipótesis,
              \placeformula
              \startformula
                x < 0
              \stopformula
              Luego, por (1) y (2) y la ley transitiva de las desigualdades,
              \startformula
                x < 0 < y
              \stopformula
              \startformula
                x < y.
              \stopformula
            \stopitem

            \startitem
              \ini{Demuestre que $\forall\, x \in \integers,\, \abs{x} \geq x$}.

              Si $x \geq 0$,
              \resetcounter[formula]
              \placeformula
              \startformula
                \abs{x} = x
              \stopformula
              Si $x < 0$,
              \placeformula
              \startformula
                \abs{x} = -x
              \stopformula
              Además, también ocurre que
              \placeformula
              \startformula
                0 > x
              \stopformula
              y
              \placeformula
              \startformula
                - x > 0
              \stopformula
              Luego, por (4), (3) y la ley transitiva de las desigualdades, tenemos que
              \placeformula
              \startformula
                -x > x
              \stopformula
              Entonces por (5), (2) y la ley de sustitución
              \placeformula
              \startformula
                \abs{x} > x
              \stopformula
              Por lo tanto, resumiendo, por (1) y (6), concluimos que
              \startformula
                \abs{x} \geq x.
              \stopformula
            \stopitem
          \stopitemejem
        \stopejemplos

        \youtube{\from[AE45A]}
        \startejemplo
          \ini{Pruebe que $\abs{a} = \abs{-a}$}.

          Si $a \leq 0$,
          \resetcounter[formula]
          \placeformula
          \startformula
            \abs{a} = a
          \stopformula
          También, $-a \leq 0$. Luego
          \startformula
            \abs{-a} = - (-a) = a,
          \stopformula
          que, por la ley simétrica de la igualdad equivale a
          \placeformula
          \startformula
            a = \abs{-a}
          \stopformula
          Entonces, por (1), (2) y la ley transitiva de la igualdad,
          \startformula
            \abs{a} = \abs{-a}.
          \stopformula

          Si $a < 0$, entonces
          \placeformula
          \startformula
            \abs{a} = -a,
          \stopformula
          y también
          \startformula
            -a > 0
          \stopformula
          En consecuencia,
          \startformula
            \abs{-a} = -a,
          \stopformula
          que podemos escribir, por la ley simétrica de la igualdad, como
          \placeformula
          \startformula
            - a = \abs{-a}
          \stopformula
          Luego, por (3), (4) y la ley transitiva de la igualdad concluimos que
          \startformula
            \abs{a} = \abs{-a}.
          \stopformula
          
        \stopejemplo

      \stopdiscusion

      \startsection[title={Orden en $\integers$}]
        \startdefinicion
          Sea $n \in \integers$.
          \startitemizer
            \startitem
              \obj{El número entero sucesor o consecutivo de $n$} es el número entero $n + 1$. Decimos que $n$ y $n + 1$ son \obj{números enteros sucesivos o consecutivos}.
            \stopitem
            \startitem
              El entero $n - 1$ se llama \obj{el antecesor o predecesor del número entero $n$}. Vea que $n - 1$ y $n$ son también consecutivos o sucesivos ya que $(n -1) + 1 = n + [(-1) + 1] = n + 0 = n$.
            \stopitem
          \stopitemizer
        \stopdefinicion
      \stopsection

      \startejemplos
        \startitemejem
          \startitem
           $2 \text{ y } (2 + 1 =)\, 3$ son enteros consecutivos o sucesivos.
          \stopitem
          \startitem
            $2 = 3 - 1$, entonces 2 es el antecesor de 3.
          \stopitem
          \startitem
            $-15 \text{ y } (-15 + 1 =)\, -14 $ son sucesivos o consecutivos.
          \stopitem
          \startitem
            $-15 - 1 = -16$, pues el -16 es el predecesor de -15.
          \stopitem
        \stopitemejem
      \stopejemplos

      \youtube{\from[AE45B]}
      \startteorema
        Para todo $n \in \integers,\, n < n + 1$
      \stopteorema

      \startdemo
        Como $1 > 0$, entonces por la ley de suma de las desigualdades,
        \startformula
          n + 1 > n + 0,
        \stopformula
        esto es $n + 1 > n$, que equivale a decir que $n < n + 1$.
      \stopdemo

      \startcorolario
        Para todo $n \in \integers,\, n - 1 < n$. (El antecesor de un número es menor que éste).
      \stopcorolario

      \startdefinicion
        Un conjunto $S$ de números enteros se llama \obj{bien ordenado} ssi $\forall T \subseteq S, T \neq \emptyset,\,\exists s_0 \in T$ tal que $s_0 \leq s,\, \forall s \in T$.
      \stopdefinicion

      Un subconjutno de números enteros se llamará bien ordena ssi en todo subconjutno de ese subconjutno podemos encontrar su elemento menor.

      \comentario{Para demotrar que un conjutno de número enteros no está bien ordenado, bastará encontrar un subconjunto no vacio de éste, en donde no podamos determinar su elemento más pequeño.}
      \startejemplos
        \startitemejem
          \startitem
            \ini{$\integers$ no está bien ordenado.}

            Considere $\{\dots,-101,-100,-99,-98\}$. Vemos que en este subconjunto de $\integers$ no podemos encontrar su elemento más pequeño ya que consiste de -98 y los antecesores de sus antecesores, y el corolario anterior asegura que estos son menores unos de otros.
          \stopitem
          \startitem
            \ini{Considere a $\naturalnumbers$, y los siguientes subconjuntos de $\naturalnumbers$:}
            \startitemize
              \startitem
                $\{1,2,3\} \subseteq \naturalnumbers$. El elemento más pequeño es el 1.
              \stopitem
              \startitem
                $\{100,101,102,\dots\} \subseteq \naturalnumbers$. El elemento más pequeño es el 100.
              \stopitem
              \startitem
                $\{16,14,11,13,38\} \subseteq \naturalnumbers$. El elemento más pequeño es el 11.
              \stopitem
              \startitem
                $\{28,31,32,27,26,21,41,40,56,39,72,73,74,75,\dots\} \subseteq \naturalnumbers$. El elemento más pequeño es el 21.
              \stopitem
            \stopitemize
            Parece que el conjunto de los números enteros está bien ordenado, según los casos que hemos analizado.
          \stopitem
        \stopitemejem
      \stopejemplos

      \youtube{\from[AE46A]}
      \startaxioma{el principio del buen orden, pbo}
        $\naturalnumbers$ está bien ordenado
      \stopaxioma

      \startdiscusion{Notación}
        Sean $a,b,c \in \integers$ con $a < b \land b < c$. Abreviamos esto con cualquiera de los símbolos
        \startformula
          a < b < c \,\lor\, c > b > a.
        \stopformula
      \stopdiscusion

      \startteorema
        El principio del buen orden implica que 1 es el elemento más pequeño en $\naturalnumbers$.
      \stopteorema

      \startdemo
        (por contradicción). Recordemos que todo número natural es positivo. Consideremos $M = \{x \mid x \in \naturalnumbers,\, 0 < x < 1\}$. Si $M = \emptyset$, entonces 1 será el elemento más pequeño en $\naturalnumbers$.

        Si $M \neq \emptyset$, por el principio del buen orden, como $M \subseteq \naturalnumbers$, $M$ tendrá un elemento más pequeño que llamaremos $m$. Entonces como $m \in M,\, 0 < m < 1$. Pero eso significa que $0 < m$, o sea que $m > 0$, y que
        \resetcounter[formula]\placeformula
        \startformula
          m < 1.
        \stopformula
        Entonces, por la ley de multiplicación positiva de las desigualdades,
        \startformula
          m \cdot m > 0 \cdot m \;\land\; m \cdot m < 1 \cdot m.
        \stopformula
        O sea, que $m^2 > 0$ que puede escribirse como
        \placeformula
        \startformula
          0 < m^2,
        \stopformula
        que
        \placeformula
        \startformula
          m^2 < m
        \stopformula
        Entonces, resumiendo
        \startformula
          0 < m^2 < m.
        \stopformula
        Luego, por (1) y la ley transitiva de las desigualdades, tenemos que
        \startformula
          0 < m^2 < 1
        \stopformula
        Como la multiplicación es binaria en $\naturalnumbers$. Luego, $m \cdot m = m^2 \in \naturalnumbers,\, M$. Pero vimos que $m^2 < m$. Pero $m$ era el elemento más pequeño en $M$. Esto es una contradicción.
        \startformula
          \therefore M = \emptyset \text{ y esto implica que 1 es el natural más pequeño.}
        \stopformula
      \stopdemo

      \startteorema
      \youtube{\from[AE46B]}
        Si $n \in \naturalnumbers$, entonces no existe $k \in \naturalnumbers$ con $n < k < n+1$. Esto es, no existe un número natural entre un número natural y su consecutivo.
      \stopteorema

      \startdemo
        (por contradicción). Supongamos que existe dicho $k \in \naturalnumbers$ cn $n < k < n+1$. Entonces, esto significa que
        \startformula
          n < k \,\land\, k < n+1.
        \stopformula
        Luego, por la ley de suma de las desigualdades,
        \startformula
          n + (-n) < k + (-n) \,\land\, k + (-n) < n + 1 + (-n) = [n + (-n)] + 1 = 0 + 1
        \stopformula
        O sea,
        \startformula
          0 < k + (-n) \,\land\, k + (-n) < 1
        \stopformula
        \startformula
          \therefore \, 0 < k + (-n) < 1.
        \stopformula
        Como la suma es binaria en $\integers$ y $k + (-n) > 0, k + (-n) \in \naturalnumbers$, que es menor que 1. Llegamos a una contradicción ya que el teorema anterior establecía que el 1 es el número más pequeño de los naturales.
      \stopdemo

      \startobservacion
        \startitemize[n]
          \startitem
            Entre 1 y 2 no hay un número natural; entre 2 y 3,igual; entre 105 y 106, igual; etc.
          \stopitem
          \startitem
            Siguiendo argumentos análogos, podemos demostrar que en $\{-1,-2,-3,\dots\}$ el -1 es el elemento más grande. También que entre $n-1$ y $n$ no hay ningún entero. Así, entre $-1$ y $-2$ no hay un entero; entre $-2$ y $-3$, tampoco, etc.
          \stopitem
        \stopitemize
      \stopobservacion

    \stopsection

    \startsection[title={La recta numérica}]
      $\dots --> -4 --> -3 --> -2 --> -1 --> 0 --> 1 --> 2 --> 3 --> 4 --> \dots$      

      Entre cada número entero y su consecutivo hay la misma distancia, una unidad de distancia.

      \youtube{\from[AE47A]}
      \startdiscusion{Repaso}
        \startitemize
          \startitem
            1 es el natural menor.
          \stopitem
          \startitem
            $-1$ es el entero negativo mayor.
          \stopitem
          \startitem
            $n$ y $n+1$ son números enteros consecutivos, y $n-1$ es el antecesor del entero $n$.
          \stopitem
          \startitem
            $\nexists\, k \in \integers$ con $n < k < n+1$
          \stopitem
        \stopitemize
      \stopdiscusion

      El orden de los números enteros lo represetamos con la recta numérica.

      GRAFICO

      \startdefinicion
        Llamamos al largo de la parte de la recta numérica entre el punto que representa a un número entero y el punto que corresponde a su consecutivo, \obj{el largo unitario}. Éste se representa con la variable $u$ y se le asigna el valor de 1.
      \stopdefinicion

      Como $a - b = x --> a = b + x$
      \startformula
        a = b + x \cdot 1,
      \stopformula
      diremos que para movernos del putno que corresponde al entero $b$ al que corresponde al entero $a$, en una recta númerica, nos movemos $x$ largos unitarios partiendo de $b$.

      Este movimiento se hará hacia la derecha (o hacia arriba) si $x > 0$; se hará hacia la izquierda (o hacia abajo) si $x < 0$.

      \startejemplos
        GRAFICOS
        \startitemejem
          \startitem
            $\ini{5 - 2 = 3} --> 5 = 2 + 3 \cdot 1$ 
          \stopitem
          \startitem
            $\ini{2 - 4 = -2} --> 2 = 4 + (-2) \cdot 1$
          \stopitem
        \stopitemejem
      \stopejemplos

      \youtube{\from[AE47B]}
      Recordemos que  $\abs{a} = \abs{-a}$. Luego, si consideremo al número $x - y \in \integers$, donde $x, y \in \integers$, vemos que
      \startformula
        \startalign
          \NC \abs{x - y} = \NC \abs{-(x-y)} \NR
          \NC               \NC \abs{-(x + (-y))} \NR
          \NC               \NC \abs{-x +[-(-y)]} \NR
          \NC               \NC \abs{-x + y} \NR
          \NC               \NC \abs{y + (-x)} \NR
          \NC               \NC \abs{y - x} \NR
        \stopalign
      \stopformula
      \startformula
        \therefore\, \abs{x-y} = \abs{y -x}
      \stopformula

      \startdefinicion
        Sean $a,b \in \integers$. \obj{La distancia entre $a$ y $b$}, representada con la variable $d$, se define como $d = \abs{a - b} = abs{b -a}$.
      \stopdefinicion

      \startejemplos
        Determine la distancia entre:
        \startitemejem
          \startitem
            $\ini{229 \text{ y } 315}$
            \startformula[align=right]
              d = \abs{229-315} = \abs{-86} = -(-86) = 86
            \stopformula
            \startformula[align=right]
              d = \abs{315-229} = \abs{86} = 86
            \stopformula
          \stopitem
          \startitem
            $\ini{-8 \text{ y } -31}$
            \startformula[align=right]
              d = \abs{-8 -(-31)} = \abs{-8 + 31} = \abs{23} = 23
            \stopformula
          \stopitem
          \startitem
            $\ini{5 \text{ y } -2}$
            \startformula[align=right]
              d = \abs{-2 -5)} = \abs{-2 + (-5)} = \abs{-7} = -(-7) = 7
            \stopformula
          \stopitem
        \stopitemejem
      \stopejemplos

      \startobservacion
        \startitemize[n]
          \startitem
            La distancia entre dos números enteros corresponde al número de largos unitarios que hay entre los putnos que corresponden a esos enteros en una recta numérica. 
          \stopitem
          \startitem
            En la notación $a < b < c$, por lo discutido al construir la recta numérica, los puntos que corresponden en ésta a los números $a,b$ y $c$ irán de izquierda a derecha.
          \stopitem
        \stopitemize
      \stopobservacion
    \stopsection
  \stopchapter
\stopcomponent