
\startcomponent c_operaciones

\project project_matemprepa

\startchapter[title={Operaciones fundamentales con expresiones algebraicas}]

  \startdefinicion
    Llamamos una \obj{expresión algebraica} a cualquier combinación de símbolos consistentes de números relaes fijos, que llamaremos \obj{constantes}, y letras, que representarán a números reales, llamdas \obj{variables}, relacionados entre sí por medio de las operaciones de suma, resta, multiplicación, división y exponenciación.
  \stopdefinicion
 
  \startejemplos
    \startitemejem
      \startitem
        $\ini{2x^2 - 3x - 1}\quad$ constantes: 2, 3, 1; variables: $x$; operaciones: multiplicación, exponenciación y restas
      \stopitem
      \startitem
        $\ini{\left(a^2b -1\right)\left(\dfrac{a^2}{3c - b^3}\right)}\quad$ constantes: 1, 3, 2; variables: $a, b, c$; operaciones: restas, multiplicaciones, división y exponenciaciones.
      \stopitem
      \startitem
        $\ini{\dfrac{x^3 -x +11}{2x^3 +5x^2 -3x^4}}\quad$, constantes: 3, 2, 5, 4, 11; variables: $x$; operaciones: exponenciaciones, restas, sumas, división y multipliaciones.
      \stopitem
    \stopitemejem
  \stopejemplos

  \startdefinicion
    \startitemizer
      \startitem
        Cada sumando en una expresión algebraica se llama un \obj{término}
      \stopitem
      \startitem
        Cada número (constante o variable) que esté multiplicado por otro en una expresión algebraica se llama un \obj{factor}.
      \stopitem
    \stopitemizer
  \stopdefinicion

  \startejemplos
    \startitemejem
      \startitem
        \ini{Términos en la expresión algebraica $2x^2 -3x -1$:} $2x^2$, $-3x$, $-1$.
      \stopitem
      \startitem
        \ini{Factores en el primer término de la expresión algebraica anterior:} 2, $x^2$.
      \stopitem
      \startitem
        \ini{Términos en el primer factor de $\left(a^2b -1\right)\left(\dfrac{a^2}{3c - b^3}\right)$:} $a^2b$, $-1$.
      \stopitem
      \startitem
        \ini{Factores en el primer término del denominador del segundo factor en la expresión algebraica anterior:} 3, $c$.
      \stopitem
    \stopitemejem
  \stopejemplos

  \startdefinicion
    \startitemizer
      \startitem
        Una expresión algebraica que es, fundamentalmente, una suma (de términos) se llama un \obj{multinomio}.
      \stopitem
      \startitem
        Una expresión algebraica que es, fundamentalmente, una multiplicación se llama un \obj{producto}.
      \stopitem
      \startitem
        La expresión algebraica que es, fundamentalmente, una división se llama una \obj{fracción algebraica}.
      \stopitem
    \stopitemizer
  \stopdefinicion

  \startejemplos
    \startitemejem
      \startitem
        $2x^2-3x-1$ es un multinomio.
      \stopitem
      \startitem
        $\left(a^2b -1\right)\left(\dfrac{a^2}{3c - b^3}\right)$ es un producto.
      \stopitem
      \startitem
        $\dfrac{x^3 -x +11}{2x^3 +5x^2 -3x^4}$ es una fracción algebraica.
      \stopitem
    \stopitemejem
  \stopejemplos

  Podemos clasificar los multinomios de acuerdo al número de términos que contiene:

  \startcenteraligned
    \starttabulate[|||]
      \HL
      \NC \bf  nonomio:    \NC dos términos    \NC\NR
      \NC \bf  binomio:    \NC dos términos    \NC\NR
      \NC \bf  trinomio:   \NC tres términos   \NC\NR
      \NC \bf  tetranomio: \NC cuatro términos \NC\NR
      \NC \bf  pentanomio: \NC cinco términos  \NC\NR
      \NC \bf  hexanomio:  \NC seis términos   \NC\NR
      \HL
 \stoptabulate
  \stopcenteraligned

  \startejemplos
    \ini{Clasifique:}
    \startitemejem
      \startitem
        $\ini{-2x^3y^4}\quad$ monomio
      \stopitem
      \startitem
        $\ini{5x -3y +9}\quad$ trinomio
      \stopitem
      \startitem
        $\ini{(x-y)(x+2)-6x^2}\quad$ binomio
      \stopitem
      \startitem
        $\ini{x^2 -3y(x+1)-\dfrac{3}{y^3}-5 -3x^3y^2 + 9y}\quad$ hexanomio
      \stopitem
    \stopitemejem
  \stopejemplos

  \startdefinicion
    En un término particular
    \startitemizer
      \startitem
        el producto de sus factores constantes es un \obj{coeficiente (numérico)}.
      \stopitem
      \startitem
        el producto de sus factores variables se llama su \obj{coeficiente literal}.
      \stopitem
      \startitem
        dada el producto de varios de sus factores, al producto del resto de sus factores se le llama el \obj{coeficiente del producto dado}.
      \stopitem
      \startitem
        si el término no contiene factores variables en su denominador, su \obj{grado} es el total de la suma de los exponentes de sus variables.
      \stopitem
    \stopitemizer
  \stopdefinicion

  \startejemplos
    \startitemejem
      \startitem
        $\ini{3x^2y^3(-2z)}$
        \startitemize[packed]
          \startitem
            coeficiente numérico (c.n): $3(-2) = -6$
          \stopitem
          \startitem
            coeficiente literal (c.l): $x^2\cdot y \cdot z$
          \stopitem
          \startitem
            coeficiente de $3x^2$: $-2y^3z$
          \stopitem
          \startitem
            coefiente de de $x^2z$: $3(-2)y^3 = -6y^3$
          \stopitem
          \startitem
            coeficiente de $-2xyz$: $3xy^2$
          \stopitem
          \startitem
            grado: $2+3+1 = 6$
          \stopitem
        \stopitemize
      \stopitem

      \startitem
        $\ini{\dfrac{4x^3y^4z^2}{5}}$
        \startitemize[packed]
          \startitem
            c.n: $\dfrac{4}{5}$
          \stopitem
          \youtube{\from[AE70A]}
          \startitem
            c.l: $x^3y^4z^2$
          \stopitem
          \startitem
            coeficiente de $\dfrac{1}{5}y^4$: $4x^3z^2$
          \stopitem
          \startitem
            coeficiente de $2x^2y^4z$: $\dfrac{2xz}{5}$
          \stopitem
          \startitem
            coeficiente de $\dfrac{4}{5}xy^3$: $x^2yz^2$
          \stopitem
          \startitem
            grado: $3+4+2 = 9$
          \stopitem
        \stopitemize
      \stopitem

      \startitem
        $\ini{\dfrac{-3a^3b^5c^6}{4d^7}}$
        \startitemize[packed]
          \startitem
            c.n: $-\dfrac{3}{4}$
          \stopitem
          \startitem
            c.l: $\dfrac{a^3b^5c^6}{d^7}$
          \stopitem
          \startitem
            coeficiente de $\dfrac{3}{4}a^3b^4$: $-\dfrac{bc^6}{d^7}$
          \stopitem
          \startitem
            coeficiente de $\dfrac{a^2b^2c}{-4d^2}$: $\dfrac{3ab^3c^5}{d^5}$
          \stopitem
          \startitem
            coeficiente de $\dfrac{-3b^5c^2}{d^5}$: $\dfrac{a^3c^4}{4d^2}$
          \stopitem
          \startitem
            su grado no está definido pues contiene variables en denominador.
          \stopitem
        \stopitemize
      \stopitem
    \stopitemejem
  \stopejemplos

  \startdefinicion
    Dos términos se llaman \obj{semejantes} ssi tienen los mismos coeficientes literales.
  \stopdefinicion

  \startejemplos
    \startitemejem
      \startitem
        \ini{$3x$ y $-2y$:} no son semejantes
      \stopitem
      \startitem
        \ini{$5^2$ y $-6x^2$:} son semejantes
      \stopitem
      \startitem
        \ini{$x^3y^2$ y $4x^2y^3$:} no son semejantes
      \stopitem
      \startitem
        \ini{$\dfrac{4}{5}a^4b$ y $-3ba^4$:} son semejantes
      \stopitem
    \stopitemejem
  \stopejemplos
  
  \youtube{\from[AE70B]}
  \startejemplo
   \ini{Simplifique las siguientes expresiones:}
   \startitemejem
     \startitem
       $\ini{3x -2y^3 + 7x^2 -1 + 2y^3 - 6x^2} = 3x + x^2 - 1$ 
     \stopitem
     \startitem
       $\ini{2x^3y + 1 - \left[2y + 6\left(3x^2-2+x^3-y\right)\right]+7x^3y} = 2x^3y + 1 -\left[2y +6 \left(4x^3-2-y\right)\right]+7x^3y = 2x^3y + 1 - \left[ 2y + 24x^3 -12 -6y\right] + 7x^3y = 2x^3y + 1 - \left[-4y + 24x^3 -12\right] + 7x^3y = 2x^3y + 1 + 4y - 24x^3 + 12 + 7 x^3y = 9x^3y + 13 + 4y - 24x^3$
     \stopitem
   \stopitemejem
  \stopejemplo

  \youtube{\from[AE71A]}
  \startejemplo
    \ini{Simplifique:}
    \startejerformula
      \startalign
        \NC   \NC \ini{\left(2x-y^3+x\right) - \left{2x - \left[y^3+4-6x+\left(2-5x-1\right)\right]\right} + 4y^3} \NR
        \NC = \NC \left(3x-y^3\right) - \left{2x - \left[y^3+4-6x+\left(1-5x\right)\right]\right} + 4y^3\NR
        \NC = \NC 3x-y^3 - \left{2x - \left[y^3+4-6x+1-5x\right]\right} + 4y^3\NR
        \NC = \NC 3x-y^3 - \left{2x - \left[y^3+5-11x\right]\right} + 4y^3\NR
        \NC = \NC 3x-y^3 - \left{2x - y^3 -5 +11x\right} + 4y^3\NR
        \NC = \NC 3x-y^3 - \left{13x - y^3 -5\right} + 4y^3\NR
        \NC = \NC 3x-y^3 - 13x + y^3 +5 + 4y^3\NR
        \NC = \NC -10x +4y^3 +5\NR
      \stopalign
    \stopejerformula
  \stopejemplo

  \startdefinicion
    Decimos que \obj{evaluamos numéricamente una expresión algebraica} ssi al conocer valores específicos de sus variables los sustituimos por éstos y entonces simplificamos.
  \stopdefinicion

  \startejemplos
    \startitemejem
      \startitem
        \ini{$a + 4b - c^2 + d^2$ si $a=3,\; b=-2,\; c=5,\; d=2$}

        $= 3 + 4(-2) -5^2 + 2^3 = 3 + 4(-2) -25 + 8 = 3 + (-8) -25 + 8 = [3+(-8)] -25 + 8 = [-5+(-25)]+8 = -30 + 8 = -22$
      \stopitem
      \startitem
        \youtube{\from[AE71B]}
        \ini{$x - y \left\{z^2 - \left[ 5 -xy + \left(7-x^2 \right)\right]\right\}$ si $x=3,\; y=2,\; z=4$}

        \startejerformula
          \startalign
            \NC   \NC 3 - 2\left\{4^2 -\left[5-3\cdot 2 + \left(7 - 3^2\right)\right]\right\} \NR
            \NC = \NC 3 - 2\left\{4^2 -\left[5-3\cdot 2 + \left(7 - 9\right)\right]\right\} \NR
            \NC = \NC 3 - 2\left\{4^2 -\left[5-3\cdot 2 + \left(- 2\right)\right]\right\} \NR
            \NC = \NC 3 - 2\left\{4^2 -\left[5- 6 + \left(- 2\right)\right]\right\} \NR
            \NC = \NC 3 - 2\left\{4^2 -\left[5 + (-6) + \left(- 2\right)\right]\right\} \NR
            \NC = \NC 3 - 2\left\{4^2 -\left[-3\right]\right\} \NR
            \NC = \NC 3 - 2 \{16 -\left(-3\right)\} \NR
            \NC = \NC 3 - 2 \{19 \} \NR
            \NC = \NC 3 - 38 \NR
            \NC = \NC 35 \NR
          \stopalign
        \stopejerformula
      \stopitem
    \stopitemejem
  \stopejemplos

  \startsection[title={Leyes de exponentes}]

    \startteorema{leyes de exponentes}
      Sean $x,y \in \reals$ y $m,n \in \naturalnumbers$. Entonces,
      \startitemizer
        \startitem
          $x^mx^n = x^{m+n}$
        \stopitem

        \startitem
          si $x \neq 0$, entonces

          \startformula
            \dfrac{x^m}{x^n} =
            \startmathcases
              \NC x^{m-n},            \NC si $\;m \geq n$ \NR
              \NC \dfrac{1}{x^{n-m}}, \NC si $\;m < n$    \NR
            \stopmathcases
          \stopformula
        \stopitem

        \startitem
          $\left(x^m\right)^n = x^{m\cdot n}$
        \stopitem

        \startitem
          $\left(xy\right)^n = x^ny^n$
        \stopitem

        \startitem
          si $y \neq 0, \;\; \left(\dfrac{x}{y}\right)^n = \dfrac{x^n}{y^n}$
        \stopitem
      \stopitemizer
    \stopteorema
    
    \startdemop

      {\sl i)} $\;x^mx^n = \underbrace{(x \cdot x \cdot x \cdot \dots \cdot x)}_{\text{m-factores}}(\underbrace{x \cdot x \cdot x \cdot \dots \cdot x}_{\text{n-factores}}) = \underbrace{x \cdot x \cdot x \cdot \dots \cdot x \cdot x \cdot x \cdot x \cdot \dots \cdot x}_{\text{(m+n) factores}} = x^{m+n}$

      {\sl ii)} si $m \geq n$
      \startformula
        \dfrac{x^m}{x^n} = \dfrac{\overbrace{x \cdot x \cdot x \cdot \dots \cdot x}^{\text{m-factores}}}{\underbrace{x \cdot x \cdot x \cdot \dots \cdot x}_{\text{n-factores}}} = \dfrac{\overbrace{(x \cdot x \cdot x \cdot \dots \cdot x)}^{\text{n-factores}}\overbrace{(x \cdot x \cdot x \cdot \dots \cdot x)}^{\text{(m-n)-factores}}}{\underbrace{x \cdot x \cdot x \cdot \dots \cdot x}_{\text{n-factores}}} = \underbrace{x \cdot x \cdot x \cdot \dots \cdot x}_{\text{(m-n) factores}} = x^{m-n}
      \stopformula
      \youtube{\from[AE72A]}
      \starteffect[hidden]{\sl ii)} \stopeffect si $m < n$
        \startformula
          \dfrac{x^m}{x^n} = \dfrac{\overbrace{x \cdot x \cdot x \cdot \dots \cdot x}^{\text{m-factores}}}{\underbrace{x \cdot x \cdot x \cdot \dots \cdot x}_{\text{n-factores}}} = \dfrac{\overbrace{x \cdot x \cdot x \cdot \dots \cdot x}^{\text{m-factores}}}{\underbrace{(x \cdot x \cdot x \cdot \dots \cdot x)}_{\text{m-factores}}\underbrace{(x \cdot x \cdot x \cdot \dots \cdot x)}_{\text{(n-m)-factores}}} = \dfrac{1}{\underbrace{(x \cdot x \cdot x \cdot \dots \cdot x)}_{\text{(n-m)-factores}}} = \dfrac{1}{x^{n-m}}
        \stopformula

      {\sl iii)}\; Se deja como ejercicio

      {\sl iv)} $\;(xy)^n = \underbrace{xy \cdot xy \cdot xy \cdot \dots \cdot xy}_{\text{n factores}} = \underbrace{(x \cdot x \cdot x \cdot \dots \cdot x)}_{\text{n factores}} \cdot \underbrace{(y \cdot y \cdot y \cdot \dots \cdot y)}_{\text{n factores}} = x^ny^n$

      {\sl v)} $\;\left(\dfrac{x}{y}\right)^n = \underbrace{\dfrac{x}{y} \cdot \dfrac{x}{y} \cdot \dots \cdot \dfrac{x}{y}}_{\text{n factores}} = \dfrac{\overbrace{x \cdot x \cdot \dots \cdot x}^{\text{n factores}}}{\underbrace{y \cdot y \cdot \dots \cdot y}_{\text{n factores}}} = \dfrac{x^n}{y^n}$
    \stopdemop

    \startdiscusion{Exponente 0 \;$x^0$}

      Consideremos el siguiente razomaniemto

      \startformula
        x^0 = x^{1+(-1)} = x^{1-1} = \dfrac{x}{x} = 1
      \stopformula
    
      \startdefinicion
        Sea $x \in \reals$, $x \neq 0$. Entonces,
        \startformula
          x^0 = 1.
        \stopformula
      \stopdefinicion

    \stopdiscusion

    \youtube{\from[AE72B]}
    \startdiscusion{$x^n\;$ siendo $n$ un entero negativo}
      Consideremos $x^{-n}$. Lo vamos a multiplicar por $x^n$:
      \comentario{$x^mx^n = x^{m+n}$}

      \startformula
        \comentario{inverso multiplicativo\\$ab = 1 --> a = b^{-1} \wedge b = a^{-1}$}
        x^{-n}x^n = x^{-n+n} = x^0 = 1, \text{ si } x \neq 0
      \stopformula
      \startformula
        x^{-n} = \left(x^n\right)^1 = \left(\dfrac{x^n}{-1}\right)^{-1} = \dfrac{1}{x^n}, \text{ si } x \neq 0
      \stopformula
    \stopdiscusion

    \startdefinicion
      Sea $x \in \reals, \; x \neq 0$, y sea $n \in \naturalnumbers$. Entonces,
      \startformula
        x^{-n} = \dfrac{1}{x^n}.
      \stopformula
    \stopdefinicion

    \startteorema
      Sean $x,y \in \reals,\, x,y \neq 0$, y sea $n \in \naturalnumbers$. Entonces,
      \startitemizer
        \startitem
          $\dfrac{1}{x^{-n}} = x^n$
        \stopitem
        \startitem
          $\left(\dfrac{x}{y}\right)^{-n} = \left(\dfrac{y}{x}\right)^n$
        \stopitem
      \stopitemizer
    \stopteorema

    \startdemop
      {\sl i)} Queda como ejercicio

      {\sl ii)} $\left(\dfrac{x}{y}\right)^{-n} = \dfrac{1}{\left(\dfrac{x}{y}\right)^n} = \dfrac{1}{\dfrac{x^n}{y^n}} = \dfrac{(1)y^n}{\left(\dfrac{x^n}{y^n}\right) y^n} = \dfrac{y^n}{x^n} = \left(\dfrac{y}{x}\right)^n$
    \stopdemop

    \startdiscusion{Recopilatorio de las leyes de exponentes}
      \startitemizer[columns,joinedup]
        \startitem
          $x^m \cdot y^n = x^{m+n}$
        \stopitem

        \startitem
          $\dfrac{x^m}{x^n} =
            \startmathcases
              \NC x^{m-n},            \NC si $\;m \geq n$ \NR
              \NC \dfrac{1}{x^{n-m}}, \NC si $\;m < n$    \NR
            \stopmathcases$
        \stopitem

        \startitem
          $\left(x^m\right)^n = x^{mn}$
        \stopitem

        \startitem
          $(xy)^n = x^n y^n$
        \stopitem

        \startitem
          $\left(\dfrac{x}{y}\right)^n = \dfrac{x^n}{y^n}$
        \stopitem

        \startitem
          $x \neq 0, \, x^0=1$
        \stopitem

        \startitem
          $x^{-n} = \dfrac{1}{x^n}; \; \dfrac{1}{x^{-n}} = x^n,\, x \neq 0$
        \stopitem

        \startitem
          $\left(\dfrac{x}{y}\right)^{-n} = \left(\dfrac{y}{x}\right)^n,\, x,y \neq 0$
        \stopitem
      \stopitemizer
    \stopdiscusion

    \startejemplos
      \startplaceformula
        \startejerformula
          \startalign
            \NC   \NC \ini{5 \cdot 3^2xy^2z^3\cdot 3^3ax^4y^5} \NR[+]
            \NC = \NC 5 \cdot \left(3^2 \cdot 3^3\right)\left(x x^4\right)\left(y^2y^3\right)z^3a \NR
            \NC = \NC 5 \cdot 3^{2+3}x^{1+4}y^{2+5}z^3a \NR
            \NC = \NC 5 \cdot 3^5x^5y^7z^3a \NR
          \stopalign
        \stopejerformula

        \startplaceformula
          \startejerformula
            \startalign
              \NC \NC \ini{\left(-2x^5y^3\right)^2\left(-3xy^5z\right)^3} \NR[+]
              \NC = \NC (-2)^2\left(x^5\right)^2\left(y^3\right)^2(-3)^3x^3\left(^5\right)^3z^3 \NR
              \NC = \NC 4x^{5\cdot 2}y^{3\cdot 2}(-27)x^3y^{5 \cdot 3}z^3 \NR
              \NC = \NC -108x^{10}y^6x^3y^{15}z^3 \NR
              \NC = \NC -108x^{13}y^{21}z^3 \NR
            \stopalign
          \stopejerformula
        \stopplaceformula

        \youtube{\from[AE73A]}
        \startplaceformula
          \startejerformula
            \startalign
              \NC   \NC \ini{-\left(\dfrac{3^4xy^5}{3^5x^6y^2}\right)^3} \NR[+]
              \NC = \NC -\left(\dfrac{y^{5-2}}{3^{5-4}x^{6-1}}\right)^3 
              = -\left(\dfrac{y^3}{3x^5}\right)^3 
              = -\dfrac{\left(y^3\right)^3}{\left(3x^5\right)^3} 
              = -\dfrac{y^9}{3^3\left(x^5\right)^3}
              = -\dfrac{y^9}{27x^{15}} \NR
             \stopalign
          \stopejerformula
        \stopplaceformula

        \startplaceformula
          \startejerformula
            \startalign
              \NC \NC \ini{\left(\dfrac{-2a^3b^{-2}c}{a^5b^{-9}c^{-3}}\right)^{-5}} \NR[+]
              \NC = \NC \left(\dfrac{-2b^9cc^3}{a^5b^2}\right)^{-5}
              = \left(\dfrac{-2b^7c^4}{a^5}\right)^{-5}
              = \left(\dfrac{a^5}{-2b^7c^4}\right)^5
              = \dfrac{\left(a^5\right)^5}{\left(-2b^7c^4\right)^5}
              = \dfrac{\left(a^5\right)^5}{\left(-2b^7c^4\right)^5}
              \NR
              \NC = \NC \dfrac{a^{25}}{(-2)^5\left(b^7\right)^5\left(c^4\right)^5}
              = \dfrac{a^{25}}{-32b^{35}c^{20}}\NR
            \stopalign
          \stopejerformula
        \stopplaceformula
      \stopejemplos
      \youtube{\from[AE73B]}

      \startejemplos
        \startitemejem[start=5]
          \startitem
            \ini{¿Cómo se comparan los valores de $\left(3x\right)^0$ y $3x^0$, si $x \neq 0$?}

            \startformula
              (3x)^0 = 1; \quad 3 \cdot x^0 = 3 \cdot 1 = 3
            \stopformula
          \stopitem

          \startitem
            \ini{Si $x-2y-7z \neq 0$ y si $a^2 + 3b \neq 0$, evalúe la expresión \\ $(x-2y-7z)^0 - 5\left(a^2 + 3b\right)^0 + 3$}

            \startformula
              = 1 - 5 \cdot 1 + 3 = 1 - 5 + 3 = -1
            \stopformula
          \stopitem
        \stopitemejem
      \stopejemplos

      Se puede demostrar que todas estas leyes de exponentes que han sido dadas para exponentes naturales funcionan aunque éstos sean enteros (no necesariamente positivos)

      \startejemplos
        \startitemejem[start=7]
          \startitem
            \ini{Demuestre que si $x \in \reals \setminus \{\emptyset\}$, y $m,n \in \integers$, entonces $x^mx^n = x^{m+n}.$}

            Veamos solo el caso en que $m,n \in \integers^{-}$. Los otros casos se dejan como ejercicio.

            \startformula
              x^m x^n = \dfrac{1}{x^{-m}}\dfrac{1}{x^{-n}}
            \stopformula
            donde ahora $-m,-n \in \integers^{+} \equiv \naturalnumbers$
            \startformula
              = \dfrac{1 \cdot 1}{x^{-m}x^{-n}} = \dfrac{1}{x^{-m+(-n)}} = \dfrac{1}{x^{-(m+n)}} = x^{m+n}
            \stopformula
          \stopitem
        \stopitemejem
      \stopejemplos
      
  \stopsection

  \youtube{\from[AE74A]}
  \startsection[title={Polinomios}]
    \startdefinicion
      Un \obj{polinomio en la variable $x$ de grado $n$}, donde $x \in \reals$ y $n \in \mathbb{W}$, es un multinomio en la forma
      \startformula
        a_n x^n + a_{n-1}x^{n-1} + a_{n-2}x^{n-2} + \cdots + a_1 x + a_0,
      \stopformula
      donde $a_n, a_{n-1}, a_{n-2},\dots, a_0 \in \reals$ fijos, llamados los \obj{coeficientes del polinomio}. Si el polinomio es el número cero, su grado no se define.
    \stopdefinicion

    \startejemplos
      Determine si las siguientes igualdades son ciertas o falsas,
      \startitemejem
        \startitem
          $\ini{x^2 + x^3 = x^5} --> $ Falsa
        \stopitem
        \startitem
          $\ini{3x^5 + 7x^5 = 10x^{10}} --> $ Falsa
        \stopitem
        \startitem
          $\ini{x^3 x^2 = x^6} --> $  Falsa
        \stopitem
        \startitem
          $\ini{3x^2 \cdot 2x^3 = 5x^5} --> $  Falsa
        \stopitem
        \startitem
          $\ini{\left(x^3\right)^5 = x^8} --> $  Falsa
        \stopitem
      \stopitemejem
    \stopejemplos

    \startsubsection[title={Características de los polinomios}]
      \startformula
        a_n x^n + a_{n-1}x^{n-1} + a_{n-2}x^{n-2} + \cdots + a_1 x + a_0,
      \stopformula


      \startitemize[n]
        \startitem
          es una suma de términos
        \stopitem
        \startitem
          la variable tiene que estar elevada a potencias enteras y no negativas
        \stopitem
        \startitem
          la variable no puede aparecer en ningún denominador, ya que $\dfrac{1}{x^n} = x^{-n}$ y, por tanto, el exponente sería negativo
        \stopitem
        \startitem
          el grado es la potencia más alta a la cual aparece elevada la variable
        \stopitem
        \startitem
          puede tener más de una variable, las cuales tienen que cumplir con las mismas condiciones con las que cumple una sola
        \stopitem
        \startitem
          si hay más de una variable, su grado es el grado más alto de entre sus términos, el grado de un polinomio $p$ se representa por símbolo $\partial^{0} p$
        \stopitem
      \stopitemize
    \stopsubsection

  \stopsection
  
\stopchapter
  
\stopcomponent
