
\startcomponent c_operaciones

\project project_matemprepa

\youtube{\from[AE69A]}
\startchapter[title={Operaciones fundamentales con expresiones algebraicas}]

  \startsection[title={Expresiones algebraicas}]
    \startdefinicion
      Llamamos una \obj{expresión algebraica} a cualquier combinación de símbolos consistentes de números relaes fijos, que llamaremos \obj{constantes}, y letras, que representarán a números reales, llamdas \obj{variables}, relacionados entre sí por medio de las operaciones de suma, resta, multiplicación, división y exponenciación.
    \stopdefinicion
    \startejemplos
      \startitemejem
        \startitem
          $\ini{2x^2 - 3x - 1}\quad$ constantes: 2, 3, 1; variables: $x$; operaciones: multiplicación, exponenciación y restas
        \stopitem
        \startitem
          $\ini{\left(a^2b -1\right)\left(\dfrac{a^2}{3c - b^3}\right)}\quad$ constantes: 1, 3, 2; variables: $a, b, c$; operaciones: restas, multiplicaciones, división y exponenciaciones.
        \stopitem
        \startitem
          $\ini{\dfrac{x^3 -x +11}{2x^3 +5x^2 -3x^4}}\quad$, constantes: 3, 2, 5, 4, 11; variables: $x$; operaciones: exponenciaciones, restas, sumas, división y multipliaciones.
        \stopitem
      \stopitemejem
    \stopejemplos
    \startdefinicion
      \startitemizer
        \startitem
          Cada sumando en una expresión algebraica se llama un \obj{término}
        \stopitem
        \startitem
          Cada número (constante o variable) que esté multiplicado por otro en una expresión algebraica se llama un \obj{factor}.
        \stopitem
      \stopitemizer
    \stopdefinicion
    \startejemplos
      \startitemejem
        \startitem
          \ini{Términos en la expresión algebraica $2x^2 -3x -1$:} \quad $2x^2$, $-3x$, $-1$.
        \stopitem
        \startitem
          \ini{Factores en el primer término de la expresión algebraica anterior:} \quad 2, $x^2$.
        \stopitem
        \startitem
          \ini{Términos en el primer factor de $\left(a^2b -1\right)\left(\dfrac{a^2}{3c - b^3}\right)$:} \quad $a^2b$, $-1$.
        \stopitem
        \startitem
          \ini{Factores en el primer término del denominador del segundo factor en la expresión algebraica anterior:} \quad 3, $c$.
        \stopitem
      \stopitemejem
    \stopejemplos
    \startdefinicion
      \startitemizer
        \startitem
          Una expresión algebraica que es, fundamentalmente, una suma (de términos) se llama un \obj{multinomio}.
        \stopitem
        \startitem
          Una expresión algebraica que es, fundamentalmente, una multiplicación se llama un \obj{producto}.
        \stopitem
        \startitem
          La expresión algebraica que es, fundamentalmente, una división se llama una \obj{fracción algebraica}.
        \stopitem
      \stopitemizer
    \stopdefinicion
    \startejemplos
      \startitemejem
        \startitem
          $2x^2-3x-1$ es un multinomio.
        \stopitem
        \startitem
          $\left(a^2b -1\right)\left(\dfrac{a^2}{3c - b^3}\right)$ es un producto.
        \stopitem
        \startitem
          $\dfrac{x^3 -x +11}{2x^3 +5x^2 -3x^4}$ es una fracción algebraica.
        \stopitem
      \stopitemejem
    \stopejemplos
    \youtube{\from[AE69B]}
    Podemos clasificar los multinomios de acuerdo al número de términos que contiene:\par
    \startcenteraligned
      \starttabulate[|||]
        \HL
        \NC \bf  nonomio:    \NC dos términos    \NC\NR
        \NC \bf  binomio:    \NC dos términos    \NC\NR
        \NC \bf  trinomio:   \NC tres términos   \NC\NR
        \NC \bf  tetranomio: \NC cuatro términos \NC\NR
        \NC \bf  pentanomio: \NC cinco términos  \NC\NR
        \NC \bf  hexanomio:  \NC seis términos   \NC\NR
        \HL
      \stoptabulate
    \stopcenteraligned
    \startejemplos
      \ini{Clasifique:}
      \startitemejem
        \startitem
          $\ini{-2x^3y^4}\quad$ monomio
        \stopitem
        \startitem
          $\ini{5x -3y +9}\quad$ trinomio
        \stopitem
        \startitem
          $\ini{(x-y)(x+2)-6x^2}\quad$ binomio
        \stopitem
        \startitem
          $\ini{x^2 -3y(x+1)-\dfrac{3}{y^3}-5 -3x^3y^2 + 9y}\quad$ hexanomio
        \stopitem
      \stopitemejem
    \stopejemplos
    \startdefinicion
      En un término particular
      \startitemizer
        \startitem
          el producto de sus factores constantes es un \obj{coeficiente (numérico)}.
        \stopitem
        \startitem
          el producto de sus factores variables se llama su \obj{coeficiente literal}.
        \stopitem
        \startitem
          dada el producto de varios de sus factores, al producto del resto de sus factores se le llama el \obj{coeficiente del producto dado}.
        \stopitem
        \startitem
          si el término no contiene factores variables en su denominador, su \obj{grado} es el total de la suma de los exponentes de sus variables.
        \stopitem
      \stopitemizer
    \stopdefinicion
    \startejemplos
      \startitemejem
        \startitem
          $\ini{3x^2y^3(-2z)}$
          \startitemize[packed]
            \startitem
              coeficiente numérico (c.n): $3(-2) = -6$
            \stopitem
            \startitem
              coeficiente literal (c.l): $x^2\cdot y \cdot z$
            \stopitem
            \startitem
              coeficiente de $3x^2$: $-2y^3z$
            \stopitem
            \startitem
              coefiente de de $x^2z$: $3(-2)y^3 = -6y^3$
            \stopitem
            \startitem
              coeficiente de $-2xyz$: $3xy^2$
            \stopitem
            \startitem
              grado: $2+3+1 = 6$
            \stopitem
          \stopitemize
        \stopitem

        \startitem
          $\ini{\dfrac{4x^3y^4z^2}{5}}$
          \startitemize[packed]
            \startitem
              c.n: $\dfrac{4}{5}$
            \stopitem
            \youtube{\from[AE70A]}
            \startitem
              c.l: $x^3y^4z^2$
            \stopitem
            \startitem
              coeficiente de $\dfrac{1}{5}y^4$: $4x^3z^2$
            \stopitem
            \startitem
              coeficiente de $2x^2y^4z$: $\dfrac{2xz}{5}$
            \stopitem
            \startitem
              coeficiente de $\dfrac{4}{5}xy^3$: $x^2yz^2$
            \stopitem
            \startitem
              grado: $3+4+2 = 9$
            \stopitem
          \stopitemize
        \stopitem

        \startitem
          $\ini{\dfrac{-3a^3b^5c^6}{4d^7}}$
          \startitemize[packed]
            \startitem
              c.n: $-\dfrac{3}{4}$
            \stopitem
            \startitem
              c.l: $\dfrac{a^3b^5c^6}{d^7}$
            \stopitem
            \startitem
              coeficiente de $\dfrac{3}{4}a^3b^4$: $-\dfrac{bc^6}{d^7}$
            \stopitem
            \startitem
              coeficiente de $\dfrac{a^2b^2c}{-4d^2}$: $\dfrac{3ab^3c^5}{d^5}$
            \stopitem
            \startitem
              coeficiente de $\dfrac{-3b^5c^2}{d^5}$: $\dfrac{a^3c^4}{4d^2}$
            \stopitem
            \startitem
              su grado no está definido pues contiene variables en denominador.
            \stopitem
          \stopitemize
        \stopitem
      \stopitemejem
    \stopejemplos
    \startdefinicion
      Dos términos se llaman \obj{semejantes} ssi tienen los mismos coeficientes literales.
    \stopdefinicion
    \startejemplos
      \startitemejem
        \startitem
          \ini{$3x$ y $-2y$:} no son semejantes
        \stopitem
        \startitem
          \ini{$5^2$ y $-6x^2$:} son semejantes
        \stopitem
        \startitem
          \ini{$x^3y^2$ y $4x^2y^3$:} no son semejantes
        \stopitem
        \startitem
          \ini{$\dfrac{4}{5}a^4b$ y $-3ba^4$:} son semejantes
        \stopitem
      \stopitemejem
    \stopejemplos
    \youtube{\from[AE70B]}
    \startejemplo
      \ini{Simplifique las siguientes expresiones:}
      \startitemejem
        \startitem
          $\ini{3x -2y^3 + 7x^2 -1 + 2y^3 - 6x^2} = 3x + x^2 - 1$
        \stopitem
        \startitem
          $\ini{2x^3y + 1 - \left[2y + 6\left(3x^2-2+x^3-y\right)\right]+7x^3y} = 2x^3y + 1 -\left[2y +6 \left(4x^3-2-y\right)\right]+7x^3y = 2x^3y + 1 - \left[ 2y + 24x^3 -12 -6y\right] + 7x^3y = 2x^3y + 1 - \left[-4y + 24x^3 -12\right] + 7x^3y = 2x^3y + 1 + 4y - 24x^3 + 12 + 7 x^3y = 9x^3y + 13 + 4y - 24x^3$
        \stopitem
      \stopitemejem
    \stopejemplo

    \youtube{\from[AE71A]}
    \startejemplo
      \ini{Simplifique:}
      \startejerformula
        \startalign
          \NC   \NC \ini{\left(2x-y^3+x\right) - \left{2x - \left[y^3+4-6x+\left(2-5x-1\right)\right]\right} + 4y^3} \NR
          \NC = \NC \left(3x-y^3\right) - \left{2x - \left[y^3+4-6x+\left(1-5x\right)\right]\right} + 4y^3\NR
          \NC = \NC 3x-y^3 - \left{2x - \left[y^3+4-6x+1-5x\right]\right} + 4y^3\NR
          \NC = \NC 3x-y^3 - \left{2x - \left[y^3+5-11x\right]\right} + 4y^3\NR
          \NC = \NC 3x-y^3 - \left{2x - y^3 -5 +11x\right} + 4y^3\NR
          \NC = \NC 3x-y^3 - \left{13x - y^3 -5\right} + 4y^3\NR
          \NC = \NC 3x-y^3 - 13x + y^3 +5 + 4y^3\NR
          \NC = \NC -10x +4y^3 +5\NR
        \stopalign
      \stopejerformula
    \stopejemplo

    \startdefinicion
      Decimos que \obj{evaluamos numéricamente una expresión algebraica} ssi al conocer valores específicos de sus variables los sustituimos por éstos y entonces simplificamos.
    \stopdefinicion

    \startejemplos
      \startitemejem
        \startitem
          \ini{$a + 4b - c^2 + d^2$ si $a=3,\; b=-2,\; c=5,\; d=2$}

          $= 3 + 4(-2) -5^2 + 2^3 = 3 + 4(-2) -25 + 8 = 3 + (-8) -25 + 8 = [3+(-8)] -25 + 8 = [-5+(-25)]+8 = -30 + 8 = -22$
        \stopitem
        \startitem
          \youtube{\from[AE71B]}
          \ini{$x - y \left\{z^2 - \left[ 5 -xy + \left(7-x^2 \right)\right]\right\}$ si $x=3,\; y=2,\; z=4$}

          \startejerformula
            \startalign
              \NC   \NC 3 - 2\left\{4^2 -\left[5-3\cdot 2 + \left(7 - 3^2\right)\right]\right\} \NR
              \NC = \NC 3 - 2\left\{4^2 -\left[5-3\cdot 2 + \left(7 - 9\right)\right]\right\} \NR
              \NC = \NC 3 - 2\left\{4^2 -\left[5-3\cdot 2 + \left(- 2\right)\right]\right\} \NR
              \NC = \NC 3 - 2\left\{4^2 -\left[5- 6 + \left(- 2\right)\right]\right\} \NR
              \NC = \NC 3 - 2\left\{4^2 -\left[5 + (-6) + \left(- 2\right)\right]\right\} \NR
              \NC = \NC 3 - 2\left\{4^2 -\left[-3\right]\right\} \NR
              \NC = \NC 3 - 2 \{16 -\left(-3\right)\} \NR
              \NC = \NC 3 - 2 \{19 \} \NR
              \NC = \NC 3 - 38 \NR
              \NC = \NC 35 \NR
            \stopalign
          \stopejerformula
        \stopitem
      \stopitemejem
    \stopejemplos

  \stopsection

  \startsection[title={Leyes de exponentes}]

    \startteorema{leyes de exponentes}
      Sean $x,y \in \reals$ y $m,n \in \naturalnumbers$. Entonces,
      \startitemizer
        \startitem
          $x^mx^n = x^{m+n}$
        \stopitem

        \startitem
          si $x \neq 0$, entonces

          \startformula
            \dfrac{x^m}{x^n} =
            \startmathcases
              \NC x^{m-n},            \NC si $\;m \geq n$ \NR
              \NC \dfrac{1}{x^{n-m}}, \NC si $\;m < n$    \NR
            \stopmathcases
          \stopformula
        \stopitem

        \startitem
          $\left(x^m\right)^n = x^{m\cdot n}$
        \stopitem

        \startitem
          $\left(xy\right)^n = x^ny^n$
        \stopitem

        \startitem
          si $y \neq 0, \;\; \left(\dfrac{x}{y}\right)^n = \dfrac{x^n}{y^n}$
        \stopitem
      \stopitemizer
    \stopteorema

    \startdemop
      \startitemizer
        \startitem
          $x^mx^n = \underbrace{(x \cdot x \cdot x \cdot \dots \cdot x)}_{\text{m-factores}}(\underbrace{x \cdot x \cdot x \cdot \dots \cdot x}_{\text{n-factores}}) = \underbrace{x \cdot x \cdot x \cdot \dots \cdot x \cdot x \cdot x \cdot x \cdot \dots \cdot x}_{\text{(m+n) factores}} = x^{m+n}$
        \stopitem

        \startitem
          si $m \geq n$
          \startformula
            \dfrac{x^m}{x^n} = \dfrac{\overbrace{x \cdot x \cdot x \cdot \dots \cdot x}^{\text{m-factores}}}{\underbrace{x \cdot x \cdot x \cdot \dots \cdot x}_{\text{n-factores}}} = \dfrac{\overbrace{(x \cdot x \cdot x \cdot \dots \cdot x)}^{\text{n-factores}}\overbrace{(x \cdot x \cdot x \cdot \dots \cdot x)}^{\text{(m-n)-factores}}}{\underbrace{x \cdot x \cdot x \cdot \dots \cdot x}_{\text{n-factores}}} = \underbrace{x \cdot x \cdot x \cdot \dots \cdot x}_{\text{(m-n) factores}} = x^{m-n}
          \stopformula
          \youtube{\from[AE72A]}
          si $m < n$
          \startformula
            \dfrac{x^m}{x^n} = \dfrac{\overbrace{x \cdot x \cdot x \cdot \dots \cdot x}^{\text{m-factores}}}{\underbrace{x \cdot x \cdot x \cdot \dots \cdot x}_{\text{n-factores}}} = \dfrac{\overbrace{x \cdot x \cdot x \cdot \dots \cdot x}^{\text{m-factores}}}{\underbrace{(x \cdot x \cdot x \cdot \dots \cdot x)}_{\text{m-factores}}\underbrace{(x \cdot x \cdot x \cdot \dots \cdot x)}_{\text{(n-m)-factores}}} = \dfrac{1}{\underbrace{(x \cdot x \cdot x \cdot \dots \cdot x)}_{\text{(n-m)-factores}}} = \dfrac{1}{x^{n-m}}
          \stopformula
        \stopitem

        \startitem
          Se deja como ejercicio
        \stopitem

        \startitem
          $(xy)^n = \underbrace{xy \cdot xy \cdot xy \cdot \dots \cdot xy}_{\text{n factores}} = \underbrace{(x \cdot x \cdot x \cdot \dots \cdot x)}_{\text{n factores}} \cdot \underbrace{(y \cdot y \cdot y \cdot \dots \cdot y)}_{\text{n factores}} = x^ny^n$
        \stopitem

        \startitem
          $\left(\dfrac{x}{y}\right)^n = \underbrace{\dfrac{x}{y} \cdot \dfrac{x}{y} \cdot \dots \cdot \dfrac{x}{y}}_{\text{n factores}} = \dfrac{\overbrace{x \cdot x \cdot \dots \cdot x}^{\text{n factores}}}{\underbrace{y \cdot y \cdot \dots \cdot y}_{\text{n factores}}} = \dfrac{x^n}{y^n}$
        \stopitem
      \stopitemizer
    \stopdemop

    \startdiscusion{Exponente 0 \;$x^0$}

      Consideremos el siguiente razomaniemto

      \startformula
        x^0 = x^{1+(-1)} = x^{1-1} = \dfrac{x}{x} = 1
      \stopformula

      \startdefinicion
        Sea $x \in \reals$, $x \neq 0$. Entonces,
        \startformula
          x^0 = 1.
        \stopformula
      \stopdefinicion

    \stopdiscusion

    \youtube{\from[AE72B]}
    \startdiscusion{$x^n\;$ siendo $n$ un entero negativo}
      Consideremos $x^{-n}$. Lo vamos a multiplicar por $x^n$:
      % \comentario{$x^mx^n = x^{m+n}$}

      \startformula
%         \comentario
% {inverso multiplicativo\\$ab = 1 \implies a = b^{-1} \wedge b = a^{-1}$}
        x^{-n}x^n = x^{-n+n} = x^0 = 1, \text{ si } x \neq 0
      \stopformula
      \startformula
        x^{-n} = \left(x^n\right)^1 = \left(\dfrac{x^n}{-1}\right)^{-1} = \dfrac{1}{x^n}, \text{ si } x \neq 0
      \stopformula
    \stopdiscusion

    \startdefinicion

      Sea $x \in \reals, \; x \neq 0$, y sea $n \in \naturalnumbers$. Entonces,
      \startformula
        x^{-n} = \dfrac{1}{x^n}.
      \stopformula
    \stopdefinicion

    \startteorema
      Sean $x,y \in \reals,\, x,y \neq 0$, y sea $n \in \naturalnumbers$. Entonces,
      \startitemizer
        \startitem
          $\dfrac{1}{x^{-n}} = x^n$
        \stopitem
        \startitem
          $\left(\dfrac{x}{y}\right)^{-n} = \left(\dfrac{y}{x}\right)^n$
        \stopitem
      \stopitemizer
    \stopteorema

    \startdemop
      {\sl i)} Queda como ejercicio

      {\sl ii)} $\left(\dfrac{x}{y}\right)^{-n} = \dfrac{1}{\left(\dfrac{x}{y}\right)^n} = \dfrac{1}{\dfrac{x^n}{y^n}} = \dfrac{(1)y^n}{\left(\dfrac{x^n}{y^n}\right) y^n} = \dfrac{y^n}{x^n} = \left(\dfrac{y}{x}\right)^n$
    \stopdemop

    \startdiscusion{Recopilatorio de las leyes de exponentes}
      \startitemizer[columns,joinedup]
        \startitem
          $x^m \cdot y^n = x^{m+n}$
        \stopitem

        \startitem
          $\dfrac{x^m}{x^n} =
          \startmathcases
            \NC x^{m-n},            \NC si $\;m \geq n$ \NR
            \NC \dfrac{1}{x^{n-m}}, \NC si $\;m < n$    \NR
          \stopmathcases$
        \stopitem

        \startitem
          $\left(x^m\right)^n = x^{mn}$
        \stopitem

        \startitem
          $(xy)^n = x^n y^n$
        \stopitem

        \startitem
          $\left(\dfrac{x}{y}\right)^n = \dfrac{x^n}{y^n}$
        \stopitem

        \startitem
          $x \neq 0, \, x^0=1$
        \stopitem

        \startitem
          $x^{-n} = \dfrac{1}{x^n}; \; \dfrac{1}{x^{-n}} = x^n,\, x \neq 0$
        \stopitem

        \startitem
          $\left(\dfrac{x}{y}\right)^{-n} = \left(\dfrac{y}{x}\right)^n,\, x,y \neq 0$
        \stopitem
      \stopitemizer
    \stopdiscusion

    \startejemplos
      \startitemejem
        \startitem
          \ini{$5 \cdot 3^2xy^2z^3\cdot 3^3ax^4y^5$}
          \startformula
            \startalign
              \NC = \NC 5 \cdot \left(3^2 \cdot 3^3\right)\left(x x^4\right)\left(y^2y^3\right)z^3a \NR
              \NC = \NC 5 \cdot 3^{2+3}x^{1+4}y^{2+5}z^3a \NR
              \NC = \NC 5 \cdot 3^5x^5y^7z^3a \NR
            \stopalign
          \stopformula
        \stopitem

        \startitem
          \ini{$\left(-2x^5y^3\right)^2\left(-3xy^5z\right)^3$}
          \startformula
            \startalign
              \NC = \NC (-2)^2\left(x^5\right)^2\left(y^3\right)^2(-3)^3x^3\left(^5\right)^3z^3 \NR
              \NC = \NC 4x^{5\cdot 2}y^{3\cdot 2}(-27)x^3y^{5 \cdot 3}z^3 \NR
              \NC = \NC -108x^{10}y^6x^3y^{15}z^3 \NR
              \NC = \NC -108x^{13}y^{21}z^3 \NR
            \stopalign
          \stopformula
        \stopitem

        \youtube{\from[AE73A]}
        \startitem
          \ini{$-\left(\dfrac{3^4xy^5}{3^5x^6y^2}\right)^3$}
          \startformula
            \startalign
              \NC = \NC -\left(\dfrac{y^{5-2}}{3^{5-4}x^{6-1}}\right)^3
              = -\left(\dfrac{y^3}{3x^5}\right)^3
              = -\dfrac{\left(y^3\right)^3}{\left(3x^5\right)^3}
              = -\dfrac{y^9}{3^3\left(x^5\right)^3}
              = -\dfrac{y^9}{27x^{15}} \NR
            \stopalign
          \stopformula
        \stopitem

        \startitem
          \ini{$\left(\dfrac{-2a^3b^{-2}c}{a^5b^{-9}c^{-3}}\right)^{-5}$}
          \startformula
            \startalign
              \NC = \NC \left(\dfrac{-2b^9cc^3}{a^5b^2}\right)^{-5}
              = \left(\dfrac{-2b^7c^4}{a^5}\right)^{-5}
              = \left(\dfrac{a^5}{-2b^7c^4}\right)^5
              = \dfrac{\left(a^5\right)^5}{\left(-2b^7c^4\right)^5}
              = \dfrac{\left(a^5\right)^5}{\left(-2b^7c^4\right)^5}
              \NR
              \NC = \NC \dfrac{a^{25}}{(-2)^5\left(b^7\right)^5\left(c^4\right)^5}
              = \dfrac{a^{25}}{-32b^{35}c^{20}}\NR
            \stopalign
          \stopformula
        \stopitem
      \stopitemejem
    \stopejemplos

    \youtube{\from[AE73B]}
    \startejemplos
      \startitemejem[start=5]
        \startitem
          \ini{¿Cómo se comparan los valores de $\left(3x\right)^0$ y $3x^0$, si $x \neq 0$?}

          \startformula
            (3x)^0 = 1; \quad 3 \cdot x^0 = 3 \cdot 1 = 3
          \stopformula
        \stopitem

        \startitem
          \ini{Si $x-2y-7z \neq 0$ y si $a^2 + 3b \neq 0$, evalúe la expresión \\ $(x-2y-7z)^0 - 5\left(a^2 + 3b\right)^0 + 3$}

          \startformula
            = 1 - 5 \cdot 1 + 3 = 1 - 5 + 3 = -1
          \stopformula
        \stopitem
      \stopitemejem
    \stopejemplos
    Se puede demostrar que todas estas leyes de exponentes que han sido dadas para exponentes naturales funcionan aunque éstos sean enteros (no necesariamente positivos)
    \startejemplos
      \startitemejem[start=7]
        \startitem
          \ini{Demuestre que si $x \in \reals \setminus \{\emptyset\}$, y $m,n \in \integers$, entonces $x^mx^n = x^{m+n}.$}

          Veamos solo el caso en que $m,n \in \integers^{-}$. Los otros casos se dejan como ejercicio.

          \startformula
            x^m x^n = \dfrac{1}{x^{-m}}\dfrac{1}{x^{-n}}
          \stopformula
          donde ahora $-m,-n \in \integers^{+} \equiv \naturalnumbers$
          \startformula
            = \dfrac{1 \cdot 1}{x^{-m}x^{-n}} = \dfrac{1}{x^{-m+(-n)}} = \dfrac{1}{x^{-(m+n)}} = x^{m+n}
          \stopformula
        \stopitem
      \stopitemejem
    \stopejemplos

    \startejemplos
      Determine si las siguientes igualdades son ciertas o falsas,
      \startitemejem
        \startitem
          $\ini{x^2 + x^3 = x^5} \implies $ Falsa
        \stopitem
        \startitem
          $\ini{3x^5 + 7x^5 = 10x^{10}} \implies $ Falsa
        \stopitem
        \startitem
          $\ini{x^3 x^2 = x^6} \implies $  Falsa
        \stopitem
        \startitem
          $\ini{3x^2 \cdot 2x^3 = 5x^5} \implies $  Falsa
        \stopitem
        \startitem
          $\ini{\left(x^3\right)^5 = x^8} \implies $  Falsa
        \stopitem
      \stopitemejem
    \stopejemplos
  \stopsection

  \startsection[title={Polinomios}]
    \startdefinicion
      Un \obj{polinomio en la variable $x$ de grado $n$}, donde $x \in \reals$ y $n \in \mathbb{W}$, es un multinomio en la forma
      \startformula
        a_n x^n + a_{n-1}x^{n-1} + a_{n-2}x^{n-2} + \cdots + a_1 x + a_0,
      \stopformula
      donde $a_n, a_{n-1}, a_{n-2},\dots, a_0 \in \reals$ fijos, llamados los \obj{coeficientes del polinomio}. Si el polinomio es el número cero, su grado no se define.
    \stopdefinicion

    \youtube{\from[AE74A]}
    \startsubsection[title={Características de los polinomios}]
      \startformula
        a_n x^n + a_{n-1}x^{n-1} + a_{n-2}x^{n-2} + \cdots + a_1 x + a_0,
      \stopformula

      \startitemize[n]
        \startitem
          es una suma de términos
        \stopitem
        \startitem
          la variable tiene que estar elevada a potencias enteras y no negativas
        \stopitem
        \startitem
          la variable no puede aparecer en ningún denominador, ya que $\dfrac{1}{x^n} = x^{-n}$ y, por tanto, el exponente sería negativo
        \stopitem
        \startitem
          el grado es la potencia más alta a la cual aparece elevada la variable
        \stopitem
        \startitem
          puede tener más de una variable, las cuales tienen que cumplir con las mismas condiciones con las que cumple una sola
        \stopitem
        \startitem
          si hay más de una variable, su grado es el grado más alto de entre sus términos; el grado de un polinomio $p$ se representa por el símbolo $\partial^{\circ} p$
        \stopitem
      \stopitemize

      \startejemplos
        \youtube{\from[AE74B]}
        \startitemejem
          \startitem
            \ini{$-6x^4+\dfrac{1}{5}x^3-x+7$}, es un polinomio en la variable $x$ de grado 4, con coeficientes $a_u = -6, a_3 = \dfrac{1}{5}, a_2 = 0,  a_1 = -1, a_0 = 7$.
          \stopitem
          \startitem
            \ini{$3x^5-\dfrac{6}{x^3} + 8x^2 - 3x +1$}, no es un polinomio pues su variable aparece en un denominador.
          \stopitem
          \startitem
            \ini{$-1+4y^2-6y^7-2y+5y^3$}, es unn polinomio en la variable $y$ de grado 7; sus coeficientes son: $a_7=-6,\; a_6=0,\; a_5=0,\; a_4=0,\; a_3=5,\; a_2=4,\; a_1=-2,\; a_0=-1$.
          \stopitem
          \startitem
            \ini{$-3$}, es un polinomio de grado cero; su coeficiente es $a_0 = -3$.
          \stopitem
          \startitem
            \ini{$\dfrac{1}{2}y^3 -2y^4 -\dfrac{2}{3} + 3y^2$}, polinomio en la variable $y$ de grado 4; sus coeficientes son: $a_4 = -2,\; a_3=\dfrac{1}{2},\; a_2=3,\; a_1=0,\; a_0=-\dfrac{2}{3}$.
          \stopitem
          \startitem
            \ini{$\dfrac{3x-2}{x^2-8x+1}$}, no es un polinomio. Sin embargo, tanto el numerador como el denominador son polinomios de forma independiente.
          \stopitem
          \startitem
            \ini{$p = 3x^2-2xy^2 +5y -3$}, es un polinomio en las variables $x$ e $y$, donde $\partial^{\circ}p = 3$.
          \stopitem
          \startitem
            \ini{$q = 5a -3b +ab$}, es un polinomio en las variables $a$ y $b$, donde $\partial^{\circ}q = 2$.
          \stopitem
          \startitem
            \ini{$t = -7 + 2a^3b^2 + 4ab^2 -7a^4 -3a^2b^3$}, es un polinomio en las variables $a$ y $b$, con $\partial^{\circ}t = 5$.
          \stopitem
        \stopitemejem
      \stopejemplos

    \stopsubsection

    \startsubsection[title={Operaciones con polinomios}]

      \startsubsubsection[title={Suma de polinomios}]

        \startejemplo
          \ini{Sume los polinomios: $-x^2 + x -2;\;\; 3x^2-5x+6;\;\; 4x^2-x+1;\;\; -2x^2+3$}.
          \youtube{\from[AE75A]}

          \startejerformula
            \startalign
              \NC \NC \left(-x^2 + x -2\right) + \left(3x^2-5x+6\right) + \left(4x^2-x+1\right) + \left(-2x^2+3\right) \NR
              \NC = \NC - x^2 + x -2 + 3x^2 -5x -6 + 4x^2 -x +1 -2x^2-3 \NR
              \NC = \NC 4x^2 -5x -10 \NR
            \stopalign
          \stopejerformula

          Otra forma de resolverlo

          \startcenteraligned
            \starttabulate[|mr|mr|mr|]

              \NC -x^2  \NC +x  \NC -2 \NR
              \NC 3x^2  \NC -5x \NC -6 \NR
              \NC 4x^2  \NC -x  \NC +1 \NR
              \NC -2x^2 \NC     \NC -3 \NR
              \HL
              \NC 4x^2  \NC -5x \NC -10 \NR

            \stoptabulate
          \stopcenteraligned

        \stopejemplo

        \startejemplo
          \ini{Sume los polinomios: $3x^2 -xy + 9y^2;\;\; 5xy -y^2; \;\; 7x^2 -3xy -2y^2; \;\; -4x^2 +7xy +3; \;\; 9x^2 -6y^2; \;\; 3x^2 -2xy$ y $3xy -9y^2 -1$}.

          \startcenteraligned
            \starttabulate[|mr|mr|mr|mr|]

              \NC 3x^2  \NC -xy   \NC 9y^2  \NC    \NR
              \NC       \NC 5xy   \NC -y^2  \NC    \NR
              \NC 7x^2  \NC -3xy  \NC -2y^2 \NC    \NR
              \NC -4x^2 \NC +7xy  \NC       \NC +3 \NR
              \NC 9x^2  \NC       \NC -6y^2 \NC    \NR
              \NC 3x^2  \NC -2xy  \NC       \NC    \NR
              \NC       \NC 3xy   \NC -9y^2 \NC -1 \NR
              \HL
              \NC 18x^2 \NC 9xy   \NC -9y^2 \NC +2  \NR

            \stoptabulate
          \stopcenteraligned

        \stopejemplo

      \stopsubsubsection

      \startsubsubsection[title={Resta de polinomios}]
        \startejemplo
          {\ini Reste el polinomio $-6x^2 +4x -1$ del polinomio $7x^2 +5x -3$}

          \youtube{\from[AE75B]}
          \startformula
            (7x^2 +5x -3) - (-6x^2 +4x -1) = 7x^2 +5x -3 +6x^2 -4x +1 = 13x^2 +x -2
          \stopformula

          En forma vertical

          \startcenteraligned
            \starttabulate[|mr|mr|mr|]

              \NC 7x^2  \NC +5x \NC -3 \NR
              \NC +6x^2 \NC -4x \NC +1 \NR
              \HL
              \NC 13x^2 \NC +x  \NC -2 \NR

            \stoptabulate
          \stopcenteraligned

        \stopejemplo
        \startejemplos
          {\ini Reste como se indica}

          \startitemejem
            \startitem
              \starttabulate[|mr|mr|mr|]
                \NC \ini {5x^2}  \NC \ini{-3xy} \NC \ini{+y^2} \NR
                \NC \ini {-2x^2} \NC            \NC \ini{+3y^2} \NR
                \HL
                \NC 3x^2         \NC -3xy        \NC +4y^2 \NR
              \stoptabulate
            \stopitem

            \startitem
              \ini{A $5x^2-2x-6$ réstele $3x^2+7x-7$}
              \starttabulate[|mr|mr|mr|]
                \NC 5x^2  \NC -2x \NC -6 \NR
                \NC -3x^2 \NC -7x \NC +7 \NR
                \HL
                \NC 2x^2  \NC -9x \NC +1 \NR
              \stoptabulate
            \stopitem

            \startitem
              \ini{De $2a^2 -3ab +3b^2$ reste $5a^2 +ab -4b^2$}
              \starttabulate[|mr|mr|mr|]
                \NC 2a^2  \NC -3ab \NC +3b^2 \NR
                \NC -5a^2 \NC -ab  \NC +4b^2 \NR
                \HL
                \NC -3a^2 \NC -4ab \NC +7b^2 \NR
              \stoptabulate
            \stopitem

          \stopitemejem
        \stopejemplos
      \stopsubsubsection

      \startsubsubsection[title={Multiplicación de polinomios}]

        \startdiscusion[title={Caso I: mulplicación de un monomio por un polinomio con más de un término.}]
          \startejemplos
            \startitemejem
              \startitem
                $\ini{-3(2x^2 -3y +5)} = - 3\cdot 2x^2 - (-3)3y + (-3)5 = -6x^2 -(-9y) +(-15) = -6x^2 + 9y -15$
              \stopitem
              \youtube{\from[AE76A]}
              \startitem
                $\ini{3x(-2x^2 -x -1)} = -6x^3 -3x^2 -3x$
              \stopitem

              \startitem
                $\ini{(-5x +x^2 -6y^4 -3)2x^2y^3} = -10x^3y^3 +2x^4y^3 -12x^2y^7 -6x^2y^3$
              \stopitem
            \stopitemejem
          \stopejemplos
        \stopdiscusion

        \startdiscusion[title={Caso II: Multiplicación de dos polinomios, ambos con más de un término.}]
          \startformula
            \startalign
              \NC (2x -6)\underbrace{(x^2 +5x -1)}_{c \in \reals \text{, monomio}} = (2x -6) c \NC = 2x(x^2 +5x -1) -6(x^2 +5x -1) \NR
              \NC \NC = 2x^3 + 10x^2 -2x -6x^2 -30x +6 = \NR
              \NC \NC = 2x^3 + 4x^2 -32x +6 \NR
            \stopalign
          \stopformula
          Otro método \par
          % \comentario{Es más conveniente poner el que más términos tenga arriba.}
          \startcenteraligned
            \starttabulate[|mr|mr|mr|mr|]
              \NC \NC x^2 \NC +5x \NC -1 \NR
              \NC \NC     \NC 2x  \NC -6\NR
              \HL
              \NC 2x^3 \NC +10x^2 \NC -2x  \NC    \NR
              \NC      \NC -6x^2  \NC -30x \NC +6 \NR
              \HL
              \NC 2x^3 \NC +4x^2  \NC -32x \NC +6 \NR
            \stoptabulate
          \stopcenteraligned

          \startejemplos
          \youtube{\from[AE76B]}
            \ini{Multiplique}

            \startitemejem
              \startitem
                \ini{$(3x^2 -2x +4)(-3x^2 +x)$}

                \startcenteraligned
                  \starttabulate[|mr|mr|mr|mr|]
                    \NC 3x^2  \NC -2x \NC +4 \NC \NR
                    \NC -3x^2 \NC +x  \NC    \NC \NR
                    \HL
                    \NC -9x^4 \NC +6x^3 \NC -12x^2 \NC     \NR
                    \NC       \NC 3x^3  \NC -2x^2  \NC +4x \NR
                    \HL
                    \NC -9x^4 \NC +3x^3 \NC -14x^2 \NC +4x \NR
                  \stoptabulate
                \stopcenteraligned
              \stopitem

              \startitem
                \ini{$(-x^2 +xy +4y^2)(2x^2 -6xy -y^2)$}

                \startcenteraligned
                  \starttabulate[|mr|mr|mr|mr|mr|]
                    \NC -x^2  \NC +xy    \NC +4y^2    \NC         \NC       \NR
                    \NC 2x^2  \NC -6xy   \NC -y^2     \NC         \NC       \NR
                    \HL
                    \NC -2x^4 \NC +2x^3y \NC +8x^2y^2 \NC         \NC       \NR
                    \NC       \NC  6x^3y \NC -6x^2y^2 \NC -24xy^3 \NC       \NR
                    \NC       \NC        \NC   x^2y^2 \NC   -xy^3 \NC -4y^4 \NR
                    \HL
                    \NC -2x^4 \NC +8x^3y \NC +3x^2y^2 \NC -25y^3  \NC -4y^4 \NR
                  \stoptabulate
                \stopcenteraligned
              \stopitem

            \stopitemejem
          \stopejemplos
        \stopdiscusion
      \stopsubsubsection

      \startsubsubsection[title={División de polinomios}]

        \startdiscusion[title={Caso I: Dividir entre un monomio (el divisor es un monomio)}]
          % \comentario{
          %   \startformula
          %     \dfrac{a + b}{c} = \dfrac{a}{b} + \dfrac{b}{c}
          %   \stopformula
          %   \startformula
          %     \dfrac{a - b}{c} = \dfrac{a}{b} - \dfrac{b}{c}
          %   \stopformula
          % }

          \youtube{\from[AE77A]}
          \startejemplo
            \ini{$\dfrac{12x^2y^5 -16x^8y^4 -4x^6y^7 +20x^4y^3}{-4x^2y^3}$}

            $= \dfrac{12x^2y^5}{-4x^2y^3} - \dfrac{16x^8y^2}{-4x^2y^3} - \dfrac{4x^6y^7}{-4x^2y^3} + \dfrac{20x^4y3}{-4x^2y^3} = -3y^2 -(-4x^6y) - (-x^4y^4) + (-5x^2) = -3y^2 +4x^6y +x^4y^4 -5x^2$
          \stopejemplo

          \startejemplos
            \ini{Efectúe las siguientes divisiones:}

            \startitemejem
              \startitem
                $\ini{\dfrac{20x^3y^2 - 15x^4y^5z^3 -12x^5y^4}{3x^3y^2}} = \dfrac{20}{3} -5xy^3z^3 -4x^2y^2$
              \stopitem
              \startitem
                $\ini{\dfrac{20x^4y^3 -15x^2y +4x^6y^4z}{-5x^4y^2}} = -4y + \dfrac{3}{x^2y} -\dfrac{4}{5}x^2y^2z$
              \stopitem
            \stopitemejem

          \stopejemplos
        \stopdiscusion

        \startdiscusion[title={Caso II: División de dos polinomios, ambos con más de un término.}]
          El procedimiento no discute aquí por ser muy farragoso. Solo lo aplicamos. Pero para entenderlo es necesario ver unos conceptos adicionales.

          \startdefinicion
            Dos polinomios en la variable $x$
            \startformula
              a_n x^n + a_{n-1} x^{n-1} + \cdots + a_1 x + a_0
            \stopformula
            y
            \startformula
              b_m x^m + b_{m-1} x^{m-1} + \cdots + b_1 x + b_0
            \stopformula
            se dice que \obj{son iguales} ssi $n = m$ y $a_n = b_m$, $a_{n-1} = b_{n-1}$, $\dots$, $a_1 = b_1$ y $a_0 = b_0$.
          \stopdefinicion

          \youtube{\from[AE77B]}
          \startejemplo
            Si $-8x^m + 5x^{m-1} - 9x^{m-2} + x^{m-3} - 6 = a_0 + a_1 x + a_2 x^2 + a_3 x^3 + a_4 x^4$, entonces $m=4, a_0=-6,\; a_1=1,\; a_2=-9,\; a_3=5$ y $a_4=-8$.
          \stopejemplo

          \startteorema[title={algoritmo de la división para polinomios}]
            Sean $P$ y $D$ dos polinomios de manera que $\partial^{\circ}P \geq \partial^{\circ}D$. Entonces existen dos polinomios únicos, $Q$ y $R$, de manera que $P = DQ + R$, con $\partial^{\circ}R < \partial^{\circ}D$ o $R = 0$. Llamamos a $Q$, \obj{el cociente cuando dividimos a $P$ entre $D$} , y a $R$, \obj{el residuo} que resulta de dicha división.
          \stopteorema

          \startejemplos
            \ini{Efectúe las siguientes divisiones}
            \startitemejem
              \startitem
                \ini{$\left(6x^4+7x^3+4x^2-20x-9\right)\div\left(3x^2+5x-3\right)$}
              \stopitem
            \stopitemejem
          \stopejemplos
        \stopdiscusion

      \stopsubsubsection

    \stopsubsection

    La adición (y sustracción) de números complejos se hacen mejor en forma rectangular

    Todo lo dicho aquí se extiende de una forma natural a las así llamadas derivadas unilaterales de orden superior, lo que el lector puede hacer por su cuenta sin  gran trabajo.

  \stopsection

\stopchapter

\stopcomponent
