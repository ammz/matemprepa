% Created 2018-11-11 dom 23:52
% Intended LaTeX compiler: pdflatex
\documentclass[11pt]{article}
\usepackage[utf8]{inputenc}
\usepackage[T1]{fontenc}
\usepackage{graphicx}
\usepackage{grffile}
\usepackage{longtable}
\usepackage{wrapfig}
\usepackage{rotating}
\usepackage[normalem]{ulem}
\usepackage{amsmath}
\usepackage{textcomp}
\usepackage{amssymb}
\usepackage{capt-of}
\usepackage{hyperref}
\usepackage{amsthm}\theoremstyle{definition}\newtheorem{definition}{Definition}[section]
\author{Antonio Moreno}
\date{\today}
\title{}
\hypersetup{
 pdfauthor={Antonio Moreno},
 pdftitle={},
 pdfkeywords={},
 pdfsubject={},
 pdfcreator={Emacs 26.1 (Org mode 9.1.14)}, 
 pdflang={English}}
\begin{document}

\tableofcontents


\section{Lógica}
\label{sec:orgdf41137}
La lógica se encarga del estudio de los argumentos o razonamientos válidos.

\begin{abstract}
We demonstrate how to solve the Syracuse problem.
\end{abstract}


\begin{definition}
Una oración que establece un hecho que, en algún momento, puede ser clasificado cierto o falso, se llama una afirmación o enunciado.
\end{definition}
\end{document}